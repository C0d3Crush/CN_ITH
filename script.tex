% !TeX TS-program = pdflatex -shell-escape


\documentclass{article}
\usepackage{cellspace}
\setlength\cellspacetoplimit{4pt} % Adjust the length as needed
\setlength\cellspacebottomlimit{4pt} % Adjust the length as needed

\usepackage{amssymb}
\usepackage{tikz}
\usepackage{booktabs}
\usepackage{dsfont}
\usepackage{caption}
\usepackage{amsmath}
\usepackage{amsthm}
\usepackage{amssymb}
\usepackage{amsfonts}
\usepackage{tcolorbox} % for colored box around pasage
\usepackage{graphicx}
\graphicspath{{img/}}
\usepackage[ngerman]{babel}
\usepackage[margin=2.5cm,headheight=22.28003pt,top=2.5cm]{geometry}
\usepackage{mathptmx}
\usepackage{setspace}
\usepackage{lipsum} % this package is used to create dummy text.
\usepackage{enumitem}
\usepackage[utf8]{inputenc}
\usepackage[margin=2.5cm,headheight=22.28003pt,top=2.5cm]{geometry}
\usepackage{bbm}
\usepackage[margin=2.5cm,headheight=22.28003pt,top=2.5cm]{geometry}
\usepackage{xcolor}
\usepackage{tikz}
\usepackage[margin=2.5cm,headheight=22.28003pt,top=2.5cm]{geometry}
\usepackage{mathptmx}
\usepackage{setspace}
\usepackage{amsmath}
\usepackage{mathptmx}
\usepackage{graphicx} % for graphics
\usepackage{lipsum} % for sample text
\usepackage[margin=2.5cm,headheight=22.28003pt,top=2.5cm]{geometry}
\usepackage{mathptmx}
\usepackage{setspace}
\usepackage{amsmath}
\usepackage{mathptmx}
\usepackage{graphicx} % for graphics
\usepackage{lipsum} % for sample text
\usepackage{amsthm}
\usepackage[ngerman]{babel}
%\usepackage{fontspec}
\usepackage{hyperref}
\usepackage{fancyhdr}
\usepackage{titlesec}
\usepackage[T1]{fontenc}
\usepackage{lmodern}
\usepackage{xcolor}
\usepackage{graphicx}
\usepackage{svg}



%%%%%%%%% DARK THEME %%%%%%%%%%%%
% Define the dark mode colors
%\definecolor{background}{HTML}{212121} % Dark gray background
%\definecolor{textcolor}{HTML}{FFFFFF} % White text
%\definecolor{accentcolor}{HTML}{00BCD4} % Teal accent color
%\definecolor{sectioncolor}{HTML}{FF4081} % Soft pink section headings

% Set the dark mode theme
%\pagecolor{background} % Set the background color
%\color{textcolor} % Set the text color

% Set the color for section headings
%\usepackage{titlesec}
%\titleformat{\section}{\normalfont\Large\bfseries\color{sectioncolor}}{\thesection}{1em}{}
%%%%%%%%% DARK THEME %%%%%%%%%%%

% Define the dark mode colors
\definecolor{background}{HTML}{212121} % Dark gray background
\definecolor{textcolor}{HTML}{FFFFFF} % White text
\definecolor{textcolorBlack}{HTML}{FFFFFF} % White text
\definecolor{sectioncolor}{HTML}{FF4081} % Soft pink section headings
\definecolor{accentcolor}{HTML}{DC143C} % Teal accent color
\colorlet{linkcolor}{blue} % Default hyperlink color in light mode
% Anpassung des Caption-Stils
\DeclareCaptionFormat{white}{\color{white}#1#2#3\par}


% Default mode (light mode)
\pagecolor{white} % Set the background color
\color{black} % Set the text color
\usepackage{titlesec}
\titleformat{\section}{\normalfont\Large\bfseries}{}{0em}{}

% Dark mode toggle command
\newif\ifdarkmode
\darkmodefalse % Set default mode to light mode

\makeatletter
\newcommand{\toggleDarkMode}{%
    \ifdarkmode
        % Dark mode
        \captionsetup{format=white}
        \pagecolor{background} % Set the background color
        \color{textcolor} % Set the text color
        \colorlet{linkcolor}{white} % Set link color to white in dark mode
        \titleformat{\section}{\normalfont\Large\bfseries\color{sectioncolor}}{\thesection}{1em}{}
        \fancyhead[L]{\textcolor{textcolor}{\leftmark}} % Set header color in dark mode
        \renewcommand{\headrule}{\textcolor{textcolor}{\hrule}} % Set header rule color in dark mode
        % TikZ settings for dark mode
        \tikzset{
            text=white,
            accentcolor=accentcolor,
            sectioncolor=sectioncolor,
            linkcolor=white,
        }
    \else
        \definecolor{textcolor}{HTML}{000000} % White text

        % Light mode
        \colorlet{tikztextcolor}{black} % Use the default text color in light mode
        \colorlet{tikzaccentcolor}{red} % Use the default accent color in light mode    
        \pagecolor{white} % Set the background color
        \color{black} % Set the text color
        \titleformat{\section}{\normalfont\Large\bfseries}{}{0em}{}
        \colorlet{linkcolor}{blue} % Set link color to blue in light mode
        \fancyhead[L]{\leftmark} % Set header color in light mode
        \renewcommand{\headrule}{} % Remove header rule in light mode
        % TikZ settings for light mode
        \tikzset{
            text=black,
            accentcolor=accentcolor,
            sectioncolor=sectioncolor,
            linkcolor=linkcolor,
        }
        \tikzset{
        every picture/.style={text=black}, % Set default text color to black
        every node/.style={text=black}, % Set default text color for nodes to black
}
    \fi
    \hypersetup{linkcolor=linkcolor, urlcolor=linkcolor}
    %\color{tikztextcolor} % Set the text color
}
\makeatother
\pagestyle{fancy}
\fancyhf{}
\lhead{\leftmark}
\rhead{\thepage}

\hypersetup{
    colorlinks=true,
    %linkcolor=blue,
    %linkcolor=white,
    urlcolor=blue,
    citecolor=blue,
    linktoc=all
}
\newtheorem{proposition}{Proposition}

\usetikzlibrary{tikzmark}

\usetikzlibrary{automata, positioning} % Hinzufügen der benötigten TikZ-Bibliotheken
\usetikzlibrary{arrows, positioning, calc}
\newtheorem{newproposition}{Proposition}[subsection] % Propositionen werden nach Subsections nummeriert


% Funktion für ein Kapitel
\newcommand{\mychapter}[1]{%
  \chapter{#1}%
  \label{sec:\theHchapter}%
}

% Funktion für eine Section
\newcommand{\mysection}[1]{%
  \section{#1}%
  \label{sec:\thesection}%
}

% Funktion für eine Subsection
\newcommand{\mysubsection}[1]{%
  \subsection{#1}%
  \label{subsec:\thesubsection}%
}

% Funktion für eine Subsubsection
\newcommand{\mysubsubsection}[1]{%
  \subsubsection{#1}%
  \label{subsubsec:\thesubsubsection}%
}

\newcommand{\coversection}[3]{
  \clearpage % Start the section on a new page
  \thispagestyle{empty} % No page number on the section page
  
  \vspace*{2cm} % Vertical spacing
  
  \begin{center}
    %\includegraphics[width=0.5\linewidth]{#1} % Adjust the width as needed
  \end{center}
  
  \vspace{2cm} % Vertical spacing
  
  \mysection{\textbf{#2}} % Section title
  
  \vspace{1cm} % Vertical spacing
  
  \begin{flushright}
    \textit{\small #3} % Quote
  \end{flushright}  
  \clearpage % Add an empty page after the section page
}

\newcommand{\createDiagram}[2]{%
  \begin{figure}[htp]
    \centering
    #2
    \caption{#1}
    \label{fig:#1}
  \end{figure}%
}

\begin{document}

\toggleDarkMode % Set the mode based on \darkmodetrue or \darkmodefalse

\begin{titlepage}
  
    \centering
    \vspace*{2cm}
  
    % Uni Logo
    %\includegraphics[width=0.3\textwidth]{unilogo.png}
    \vspace{1cm}
  
    % Course and University Information
    \textsc{\Large Einführung in die Theoretische Informatik - Script}\\[1.5cm]
    
    \title{Einführung in die Theoretische Informatik}
    \author{Lukas Dzielski\thanks{Universität Heidelberg}}
    
    \texttt{\large https://github.com/C0d3Crush/ITH-Script}\\
    \texttt{\large Lukas.Dzielski@stud.uni-heidelberg.de}\\[2cm]

    % Git Repository
    \includegraphics[width=0.2\textwidth]{qrcode.png}\\[10cm]

  
    % Date
    {\large \today}\\[2cm]
  
    \vfill
  
\end{titlepage}

\newpage

\tableofcontents
\footnote{Überschriften mit (*) Ed. sug.}

\newpage

\documentclass[a4paper,11pt]{article}

%\setlength{\headheight}{22.62503pt}

% Use packages to set margins, fonts, and spacing
\usepackage[margin=2.5cm,headheight=22.28003pt,top=2.5cm]{geometry}
\usepackage{amssymb}
\usepackage{mathptmx}
\usepackage{setspace}
\usepackage{amsmath}
\usepackage{mathptmx}
\usepackage{graphicx} % for graphics
\usepackage{lipsum} % for sample text
\usepackage{amssymb}
\usepackage{xcolor}
\usepackage{tikz}
\usepackage{dsfont}
\usepackage{amssymb}
\usepackage{tikz}
\usepackage{caption}
\usepackage{amsthm}
\usepackage[ngerman]{babel}

\newtheorem{proposition}{Proposition}[subsection] % Propositionen werden nach Subsections nummeriert

\usetikzlibrary{tikzmark}



%\usepackage{fontspec}
%\setmainfont{TeX-Gyre-Schola/texgyreschola-regular.otf}
\newcommand{\coversection}[3]{
  \newpage
  \thispagestyle{empty}
  \begin{center}
    \vspace*{\fill}
    \includegraphics[width=0.5\linewidth]{#1}
    \vspace*{0.5cm} % Anpassung des Abstands
    \par
    \Large\textbf{#2}
    \par\vspace{0.5cm} % Anpassung des Abstands
    \begin{quote}
      \itshape\small\raggedleft #3
    \end{quote}
    \par\vspace{\fill}
  \end{center}
  \newpage
}

\begin{document}
\coversection{predic.png}{Berechenbarkeit}{Predictability is like knowing the path a river takes. The river starts at its source and flows down to the sea. Along the way, it may turn, twist, and divide, but it always follows the path of least resistance due to gravity. Knowing the terrain allows us to predict where the river will go.\\ \hspace*{\fill} - ChatGPT}

\section{Berechenbarkeit} \textbf{Konvention: } Sprechen wir von einer $e \in \mathbb{N}_0$ oder $(e_1, \cdots, e_n) \in \mathbb{N}_0^n$ wobei $n \in \mathbb{N}$ als Eingabe für eine TM oder Ausgabe einer TM, so bedetet dies, dass die Eingabe bzw. Ausgabe $bin(e)$ bzw $(bin(e_1), \cdots, bin(e_n))$ ist. Dies erlaubt es über partiell berechnenbare Funktionene $\Phi: \mathbb{N}_0^n \leadsto \mathbb{N}_0$ wobei $n\in \mathbb{N}$ zu sprechen und $L \subseteq \mathbb{N}_0$ als Sprache über $\{0, 1\}$ aufzufassen.

\subsection{Definition (Code)} Wir betrachten die Funktion code (mit geeignetem Definitionsbereich) und Zielmenge $\{0, 1\}^*$, für die folgendes gilt. Zunächst gelte \[code(L) = 10 \qquad code (S) = 00 \qquad code(R) = 01\] Für eine Instruktion $ I = (q, a, q', a', B) \in \mathbb{N}_0 \times \{0, 1\} \to \mathbb{N}_0 \times \{0, 1\} \times \{L, S, R\}$ einer normierten TM sei \[code (I) = 0^{|bin(q)|} 1 bin (q) a 0^{|bin(q')|} 1 bin (q') a' code (B)\] Für eine endliche Menge $ \Delta \subseteq \mathbb{N}_0 \times \{0, 1\} \to \mathbb{N}_0 \times \{0, 1\}\times \{L, S, R\}$ von Instruktionen einer normierten TM und $i \in [|\Delta|]$ sein $code_i(\Delta)$ dann ein längenlexikographische Ordnung i-te Wort in $\{code(I): I \in \Delta\}$ und sei \[ code (\Delta) = code_1(\Delta), \cdots, code_{|\Delta|}(\Delta)\] Für eine normierte TM $M = (\{0, \cdots, n\}, \{0, 1\}, \{\Box, 0, 1\}, \Delta, 0, \{0\})$ sei \[ code (M) = 0^{|bin(n)|} 1 bin (n) code (\Delta)\] der \textbf{Code} von $M$. Relevant ist hierbei dass es eine geeignete effektive Codierung von Turingmachinen durch Binärwörter gibt, so dass folgendes gilt 
\begin{itemize}
  \item Jede normierte TM hat einen Code 
  \item Keine zwei verschiedene normierten TMs haben den gleichen Code.
  \item Die Sprache der Codes von Turingmachinen ist entscheidbar
  \item Codes können eine geeignete Repräsentation der durch sie codierten TMs umgewandelt werden, die es insbesondere erlauben die codierten TMs effekiv zu simulieren.
  \item geignete Repräsentationen von TMs können effektiv in ihre Codes umgewandet werden.
\end{itemize}

\subsection{Definition (standardaufzählung)} Sei $\hat{w_0}, \hat{w_1}, \cdots$ die Aufzählung aller Codes normierter TMs in längenlexikographischer Ordnung. Für $e \in \mathbb{N}_0$ sei $M_e$ die durch $\hat{w_e}$ codierte TM und für $n \in \mathbb{N}$ sei $\Phi_e^n : \mathbb{N}_0^n \rightarrow \mathbb{N}_0$ die von $M_e$ berechnete n-äre partielle Funktion. Für $n \in \mathbb{N}$ heißt die Folge $(\Phi_e^n)_e\in \mathbb{N}$ \textbf{standardaufzählung} der n-ären partiell berechenbaren Funktion. Für $n \in \mathbb{N} $ und eine partiell berechenbare n-äre Funktion $\varphi: \mathbb{N}_0^n \rightarrow \mathbb{N}_0$ heißt jede zahl $e \in \mathbb{N}_0$ mit $\Phi_e^n = \varphi$ \textbf{Index} von $\varphi$.

\subsection*{Konvention: } Ergibt sich n aus dem Kontext, so schreiben wir auch $\Phi_e$ statt $\Phi_e^n$

\subsection{Bemerkung} Für $n \in \mathbb{N}$ und eine partielle berechnbare n-äre partielle Funktion $\Phi : \mathbb{N}_0^n \rightarrow \mathbb{N}_0$ gibt es unendlich viele Indizes von $\varphi$.

\subsection{Definition (U)} Es bezeichnet U die normierte TM, die bei Eingabe $(e, x_1, \cdots, x_n) \in \mathbb{N}_0^{n+1}$ wobei $n \in \mathbb{N}$ die normierte TM $\mathcal{M}_e$ bei Eingabe $(x_1, \cdots, x_n)$ simuliert und falls diese terminiert die Ausgabe der Simulierten ausgibt.

\subsection{Definition (Universell)} Eine DTM U heißt \textbf{Universell}, wenn es für alle $n \in \mathbb{N}$ und alle partiell berechenbaren Funktionen $\varphi : \mathbb{N}_0^n \leadsto \mathbb{N}_0$ eine $e \in \mathbb{N}$, so dass \[U(e, x_1, \cdots, x_n) = \varphi(x_1, \cdots, x_n)\] $\forall x_1, \cdots, x_n \in \mathbb{N}_0$ gilt.

\subsection{Bemerkung} \sloppy Die TM U ist universell, denn für $e \in \mathbb{N}_0$, $n \in \mathbb{N}$ und $x_1, \cdots, x_n \in \mathbb{N}_0$ gilt \[U(e,x_1, \cdots, x_n) = \Phi_e(x_1, \cdots, x_n)\]\\ \[(x, y) \mapsto x^y\] \[y \mapsto 2^y\] \[(x_1, \cdots, x_m, y_1, \cdots, y_n) \mapsto \varphi(x_1, \cdots, x_m, y_1, \cdots, y_n) partiell berechenbar\] \[\leadsto (y_1,\cdots, y_m) \mapsto \varphi(x_1, \cdots, x_m, y_1, \cdots, y_n) partiell berechenbar \]

\subsection{Satz ($s_n^m$ - Theorem)} $\forall m, n \in \mathbb{N}$ existiert eine berechenbare Funktion $s_n^m : \mathbb{N}_0^{m+1} \to \mathbb{N}_0$ mit \[\Phi_e^{m+1}(x_1\cdots, x_m, y_1, \cdots, y_n) = \Phi_{s_n^m(e, x_1, \cdots, x_m)}^n (y_1, \cdots, y_n)\] $\forall e, x_1, \cdots, x_m, y_1, \cdots, y_n \in \mathbb{N}_0$

\begin{proof}Fixiere $m \in \mathbb{N}$. Betrachte die DTM S , die bei Eingabe $(e, x_1, \cdots, x_m) \in \mathbb{N}_0^{m+1}$ wie folgt vorfährt.
\begin{itemize}
  \item Zunächst bestimmt S den Code von $\mathcal{M}_e$ 
  \item der Code von $\mathcal{M_e}$wird dann in einen Code einer normierten TM $\mathcal{M}$ umgewandet, die zunächst $x_1\Box\cdots\Box x_m\Box$ neben die Eingabe schreibt, dan den Kopf auf das erste Feld des beschriebenen Bandteilsbewegt und dann wie $\mathcal{M_e}$ arbeitet.
  \item Es wird bestimmt an welcher Stelle der Standardaufzählung der Code von auftaucht und diese Stelle wird ausgegeben.
\end{itemize}
Sei $s_n^m$ die von S berechnete $(m+1)$-äre partielle Funktion. Dann ist $s_n^m$ eine Funktion wie gewünscht. Es gibt überabzählbar viele Binärsprachen, denn: Betrachte Aufzählung von Binärsprachen $L_1, L_2, \cdots$ 
\begin{table}[ht]
  \centering
  \renewcommand{\arraystretch}{2} % Adjust the value to increase or decrease the cell size
  \begin{tabular}{c c c c c}
    \tikzmarknode{L0-0}{$\mathds{1}_{L_0}(0)$} & $\mathds{1}_{L_0}(1)$ & $\mathds{1}_{L_0}(2)$ & $\mathds{1}_{L_0}(3)$ \\
    $\mathds{1}_{L_1}(0)$ & $\mathds{1}_{L_1}(1)$ & $\mathds{1}_{L_1}(2)$ & $\mathds{1}_{L_1}(3)$ \\
    $\mathds{1}_{L_2}(0)$ & $\mathds{1}_{L_2}(1)$ & \tikzmarknode{L2-2}{$\mathds{1}_{L_2}(2)$} & $\mathds{1}_{L_2}(3)$ \\
  \end{tabular}
  \captionsetup{labelformat=empty, justification=centering, skip=10pt}
  \caption{Standardaufzählung}
  
  \begin{tikzpicture}[overlay, remember picture, blue, >=stealth]
    \draw [->, thick] ([yshift=1ex]L0-0.south) -- ([yshift=-1ex]L2-2.north);
  \end{tikzpicture}
\end{table}

\begin{tikzpicture}[overlay, remember picture, blue, >=stealth]
  \draw [->, thick] ([yshift=1ex]L0-0.south) -- ([yshift=-1ex]L2-2.north);
\end{tikzpicture}

$L$ mit $\mathds{1}_L(i)$ = 
$\begin{cases}
    0, & \text{wenn } \mathds{1}_{L_i}(i) = 1 \\
    1, & \text{wenn } \mathds{1}_{L_i}(i) = 0 \\
\end{cases}$
\end{proof}

\subsection{Definition (diagonales Halteproblem)} Die Menge $H_{diag} := \{e \in \mathbb{N}_0 : \Phi_e (e) \downarrow\}$ heißt \textbf{diagonales Halteproblem}.

\begin{proposition}
  Das diagonale Halteproblem ist rekursiv aufzählbar.
\end{proposition}
\begin{proof}
  Die DTM, die bei Eingabe $e \in \mathbb{N}_0$ wie $U$ bei Eingabe $(e, e)$ arbeitet, aber bei terminieren $1$ statt der Ausgabe von $U$ ausgibt berechnet die partielle charachteristische Funktion von $H_{diag}$. Die partielle Funktion $x_{H_{diag}}$ ist also partiell berechenbar. Die partielle Funktion $x_{H_{diag}^c}$ ist nicht partiell berechenbar, dann: Betrachte Standardaufzählung
\end {proof}
  
\begin{table}[ht]
  \centering
  \renewcommand{\arraystretch}{2} % Adjust the value to increase or decrease the cell size
  \begin{tabular}{c c c c c}
    \tikzmarknode{L0-0}{$\Phi_{L_0}(0)$} & $\Phi_{L_0}(1)$ & $\Phi_{L_0}(2)$ & $\Phi_{L_0}(3)$ \\
    $\Phi_{L_1}(0)$ & $\Phi_{L_1}(1)$ & $\Phi_{L_1}(2)$ & $\Phi_{L_1}(3)$ \\
    $\Phi_{L_2}(0)$ & $\Phi_{L_2}(1)$ & \tikzmarknode{L2-2}{$\Phi_{L_2}(2)$} & $\Phi_{L_2}(3)$ \\
  \end{tabular}
  \captionsetup{labelformat=empty, justification=centering, skip=10pt}
  \caption{Standardaufzählung}
\end{table}

\begin{tikzpicture}[overlay, remember picture, blue, >=stealth]
  \draw [->, thick] ([yshift=1ex]L0-0.south) -- ([yshift=-1ex]L2-2.north);
\end{tikzpicture}

$\varphi$ mit $\varphi(i)$ = 
$\begin{cases}
    \uparrow, & \text{wenn } \Phi_i(i) \downarrow\\
    \downarrow, & \text{wenn } \Phi_i(i) \uparrow \\
\end{cases}$
Wird nicht aufgezählt.

\subsection{Satz} Das diagonale Halteproblem ist nicht entscheidbar. 
\begin{proof}
  Angenommen $H_{diag}$ wäre entscheidbar. Dann wäre die partielle charakteristische Funktion $\varphi$ von $H_{diag}^c = \mathbb{N}_0 / H_{diag}$ partiell berechenbar, es gäbe also ein Index $e \in \mathbb{N}_0$ von $\varphi$. Es folge \[e \in H_{diag}^c \Leftrightarrow \varphi(e) \downarrow \Leftrightarrow \Phi_e(e) \downarrow \Leftrightarrow e \in H_{diag} \Leftrightarrow e \not \in H_{diag}^c\] Die ist ein Wiederspruch.
\end{proof}

\subsection{m-Reduktion} Für eine Sprache $A$ über einem Alphabet $\Sigma$ und eine Sprache $B$ über einem Alphabet $\Gamma$ ist A genau dann \textbf{many-one-reduzierbar}, auch \textbf{m-reduzierbar}, auf $B$, kurz $A \leq_m B$, wenn es eine berechebare Funktion. $f: \Sigma^* \to \Gamma^*$ gibt so dass \[w \in A \Leftrightarrow f(w)\in B\] $\forall w \in \Sigma^*$ gilt. Gelten $A \leq_m B$ und $B \leq_{m} A$, so sind $A$ und $B$ \textbf{many-one-äquivalent} auch \textbf{m-äquivalent}, kurz $A =_m B$.

\subsection{Bemerkung} 
\begin{itemize}
  \item [(i)] $\leq_m$ ist transitiv.
  \item [(ii)] Gilt $A \leq_m B$ für Sprachen $A$ und $B$ und ist $B$ entscheidbar, so ist auch $A$ entscheidbar.
  \item [(iii)] Alle entscheidbaren Sprachen L mit $\varnothing \not = L \not = \mathbb{N}_0$ und m-äquivalent.
\end{itemize}

\subsection{Satz} Das \textbf{initiale Halteproblem} $H_{init} = {e \in \mathbb{N}_0 = \Phi_e(0) \downarrow}$ ist nicht entscheidbar.

\subsubsection*{Idee: } suche $f:\mathbb{N}_0 \to \mathbb{N}_0$ mit $\Phi_e(e)\downarrow \Leftrightarrow \Phi_{f(e)}(0)\downarrow$ Wähle $f$ so dass $\Phi_{f(e)}(x) = \Phi_e(e) \forall x \in \mathbb{N}_0$

\begin{proof}
  Sei $\psi : \mathbb{N}_0^2 \leadsto \mathbb{N}_0$ mit $\psi (e, x) = \Phi_e(e) \forall e, x \in \mathbb{N}_0$. Dann ist $\psi$ partiell berechenbar. Sei $e_0$ ein Index von $\psi$ und $s:\mathbb{N}_0^2 \to \mathbb{N}_0$ gilt. \\Sei $f: \mathbb{N}_0 \to \mathbb{N}_0$ mit $f(e) = s(e_0, e) \forall e /in \mathbb{N}_0$.\\ Dann ist f berechenbar. \\ $\forall e \in \mathbb{N}_0$ gilt. \[e \in H_{diag} \Leftrightarrow \Phi_e(e) \downarrow \Leftrightarrow \psi(e, 0) \downarrow \Leftrightarrow \Phi_{e_0}(e, 0) \downarrow \Leftrightarrow \Phi_s (e_0, e)(0)\downarrow \Leftrightarrow \Phi_{f(e)} ...\]  
\end{proof}

\end{document}

\coversection{Turingmachine/turing.png}{Turingmachine}{A Turing machine is like a wise old person, sitting at an endless table, playing a complex game. They have a magical pen that reads and writes on the game board. They follow strict rules, do not move from their spot, but the table mysteriously moves back and forth. Their concentration is deep and calm as they perform a complex ballet of reading, writing, and state-changing.\\ \hspace*{\fill} - ChatGPT}Wir Betrachte das folgende, sehr bekannt, berechnunsmodell. Anschaulich lässt es sich wie folht beschreiben.
\begin{itemize}
  \renewcommand\labelitemi{-}
  \item Es gibt einen "Speicher" \(\leadsto\)  k unendlich lange Arrays(\textbf{Bänder})
  \item Es gibt einen "Arbeitsspeicher" \(\leadsto\) eine endliche Menge von Zusänden, die die Machine einnehmen kann
  \item Für jedes Band gibt es einen Schreib- und Lesekopf 
  \item Jeder Schritt ist wie folgt:\\ Abhängig von Zustand und gelesenene Symbol, Schreiben die Küpfe genau ein Symbol, bewegen sich nun maximal eine Position und der Zustand der Machine wird geändert.
  \item Stellt die Machine ihhr schrittweises Arbeiten ein, so wird die Ausgabe entweder den Zustand entnommen oder von einem der Bänder in geeigneter Weise abgelesen.
\end{itemize}

\createDiagram{Turingmachine}
{
  \begin{tikzpicture}[every node/.style={minimum size=1cm, font=\bfseries}]
  % Zustandskasten
  \node[draw, fill=blue] (q) at (-1,0) {q};
  \node[right=0.5cm] at (q.mid) {Zustände};

  % Bänder
  \foreach \y/\xpos in {1/-3, 2/1} {
      % Band
      \draw (-4,-\y) rectangle (4,-\y-1);
      \foreach \x in {-3.5,-2.5,...,3.5} {
          \draw (\x,-\y) -- (\x,-\y-1);
      }
      % Lesekopf
      \fill[blue] (\xpos+0.5,-\y) rectangle (\xpos+1+0.5,-\y-1);
      % Verbindung zum Zustandskasten
      \draw[->] (q) -- (\xpos+1,-\y-0.5);
  }
  \node[below=0.5cm] at (0.2,-3) {Bänder};

  \end{tikzpicture}
}

\mysubsection{Definition (Turingmachine, Alan Tuing, 1936)} 
  Sei \(k \in \mathbb{N}\) eine \textbf{k-Band-Turingmachine}m kurz k-TM, ist ein Tupe \(M = (Q, \Sigma, \varGamma, \Delta, s, F )\). Dabei ist:
  \begin{itemize}
    \item Q eine endliche Menge, \textbf{Zustandmenge}
    \item \(\Sigma\) das \textbf{Eingabealphabet}, ein Alphabet \(\Box \not \in \Sigma\)
    \item \(\varGamma\) das \textbf{Bandaphabet}, ein Alphabet mit \(\Sigma \subseteq \varGamma\) und \(\Box \in \varGamma / \Sigma\) 
    \item \(\Delta \subseteq Q \times \varGamma^{k} \rightarrow \subseteq Q \times \varGamma^{k} \times {L, S, R}^{k}\) die \textbf{Übergangsrelation}
    \item \(s \in Q\) der \textbf{Startzustand}
    \item \(F \subseteq Q\) die Menge der \textbf{akzeptierenden Zustände} 
  \end{itemize}
  \noindent Das Symbol \(\Box\) heißt \textbf{Blank}. Die Elemente von \(\Delta\) heißen \textbf{instruktionen}. Für eine Instruktion \((q_{1}, a_{1}, \cdots, a_{k}, q', a'_{1}, \cdots, a'_{k}, B_{1}, \cdots, B_{k})\) \textbf{Anweisungteil}. Die TM M ist eine \textbf{deterministische k-Band Turingmachine}, kurz k-DTM, wenn es \(\forall b \in Q \times \varGamma^{k}\) höchstens eine Instruktion \(i \in \Delta\) mit Bedingungsteil b.

\mysubsection{Definition (Konfiguration)} 
  Sei \(M = (Q, \Sigma, \Gamma, \Delta, s, F)\) eine k-TM. Eine \textbf{Konfigration} von M ist ein Tupel 
  \[
    C = (q, w_{1}, \cdots, w_{k}, p_{1}, \cdots, p_{k}) \in Q \times (p^{*})^{k} \times \mathbb{N}^{k}
  \] 
  Die \textbf{Startkonfiguration} von M zur Eingabe \((u_{1}, \cdots, u_{n}) \in (\Sigma^{*})^{n}\), wobei \(n \in \mathbb{N}\), ist die Konfiguration 
  \[
    Start_{M}(u_{1}, \cdots, u_{n}) = (s, u_{1} \Box u_{2} \Box \cdots \Box u_{n}, \Box, \cdots, 1, \cdots, 1)
  \] 
  Die Konfiguration C ist eine \textbf{Stoppkonfigration} von M, wenn es keine Instruktion \(i \in \Delta\) mit Bedingungsteil \((q, w_{1}(p_{1}), \cdots, w_{k}(p_{k}))\) gibt.

\mysubsection{Definition (Nachfolgekonfiguration)} 
  Sei \(M = (Q, \Sigma, \Gamma, \Delta, s, F)\) eine k-DTM. Für Konfiguration \(C = q_{1}, w_{1}, \cdots, w_{k}, p_{1},\cdots, p_{k}\) und \(C' = q'_{1}, w'_{1}, \cdots, w'_{k}, p'_{1},\cdots, p'_{k}\) von M ist die Konfigration C' Nachfolgekonfiguration von C, wenn es eine Instruktion 
  \[
    (q, w_{1}(p_{1}), \cdots, w_{k}(p_{k}), a_{1}', a_{k}', B_{1}, \cdots, B_{k}) \in \Delta
  \]
  gibt, sodass 
  \begin{equation*}
    w_{i}' = 
    \begin{cases}
      \Box a_{i}' w_{i}(2) \cdots w_{i}(|w_{i}|), & \text{falls}\ p_{i} = 1 \text{und}  B_{i} = L \\
      w_{i} \cdots w_{i}(|w_{i}| - 1) a_{i}' \Box, & \text{falls}\ p{i} = |w_{i}| \text{und} B_{i} = R \\
      w_{i} \cdots w_{i}(p_{i}-1) a_{i}' w_{i}(p_{i} + 1) \cdots w_{i}(|w_{i}|), & \text{sonst} \\
    \end{cases}
  \end{equation*}
  und 
  \begin{equation*}
    p_{i}' = 
    \begin{cases}
      1, & \text{falls}\ p_{i} = 1 \text{ und } B_{i} = L\\
      p_{i} - 1, & \text{falls}\ p_{i} \geq 2 \text{ und } B_{i} = L\\
      p_{i}, & \text{falls}\ B_{i} = S\\
      p_{i} + 1, & \text{falls}\ B_{i} = R\\
    \end{cases}
  \end{equation*}
  \(\forall i \in [k]\) gelten. \\ Es bezeichnen \(\rightarrow M\)  die Relation auf der Menge der Konfiguration von M, sodass \(C \rightarrow_{M} C'\) falls C, C' Konfig von M sind wobei C' eine Nachfolgekonfiguration von C ist.

\mysubsection{Definition (Rechnung)} 
  Sei \(M = (Q, \Sigma, \Gamma, \Delta, s, F)\) eine k-DTM. Eine \textbf{endliche partielle Rechnung} von M ist eine endliche Folge \(C_{1}, \cdots, C_{n}\) von Konfig von M mit \(C_{i} \rightarrow_{M} C_{i+1} \forall i \in [n-1]\). Eine \textbf{unendliche partielle Rechnung} von M ist eine unendliche Folge \(C_{1}, C_{2}, \cdots\) von Konfigration von M mit \(C_{1} \rightarrow_{M} C_{1+1} \forall i \in \mathbb{N}\). Eine \textbf{Rechnung von M zur Eingabe } \((w_{1}, \cdots, w_{n}) \in (\Sigma^*)^n\) (mit \(n \in \mathbb{N}\)) ist eine endliche partielle Rechnung \(start_M = C_1, \cdots, C_m\) bei der \(C_m\) eine Stoppkonfiguration von M oder eine unendliche partielle rechnung \(start_M(w_1, \cdots, w_n) = C_1, C_2, \cdots\)

\mysubsection{Bemerkung} 
  Ist M eine k-DTM, so gilt es \(\forall n \in \mathbb{N}\) und \((w_1, \cdots, w_n) \in (\Sigma^*)^n\) genau eine Rechnung zur Eingabe \((w_1, \cdots, w_n)\).

\mysubsection{Definition (total)} 
  Eine k-DTM \(M = (Q, \Sigma, \Gamma, \Delta, s, F)\) \textbf{terminiert} bei Eingabe \((w_1, \cdots, w_n) \in (\Sigma^*)^n\) wenn die Rechnung von M zur Eingabe \((w_1, \cdots, w_n)\) endlich ist. Eine k-TM \(M = (Q, \Sigma, \Gamma, \Delta, s, F)\) ist \textbf{total}, wenn \(\forall n \in \mathbb{N}\) und \((w_1, \cdots, w_n) \in (\Sigma^*)^n\) alle Rechnungen von M zur Eingabe \((w_1, \cdots, w_n)\) endlich sind.

\mysubsection{Definition (akzeptierte Sprache)} 
  Sei \(M = (Q, \Sigma, \Gamma, \Delta, s, F)\) eine k-TM. Eine Stoppkonfiguration \((q, w_1, \cdots, w_k, p_1, \cdots, p_k)\) von M ist \textbf{akzeptierend}, wenn \(q \in F\). Die \textbf{akzeptierte Sprache L(M)} von M ist die Sprache über dem Alphabet \(\Sigma\) so dass \(w \in L(M)\) gilt, wenn es eien endliche Rechnung \(C_1, \cdots, C_n\) von M zur Eingabe w gibt, bei der \(C_n\) eine akzeptierende Stoppkonfigration von M ist. 

\paragraph*{Hinweis: } 
  Für nicht deterministische TM heißt das insbesondere, dass es für die Wörter w in der  akzeptierten Sprache nur mindestend \textbf{eine} im einer akzeptierten Stoppkonfigration endende endliche Rechnung zur Eingabe w geben muss. Für Wörter w, die nicht in L(M) sind, sind \textbf{alle} rechnungen von M zur Eingabe am Ende nicht in einer akzeptierten Stoppkonfigration oder unendlich.

\mysubsection{Definition(entscheidbar)}
  Eine Sprache L ist genau dann \textbf{entscheidbar}, wenn es eien totale k-TM M mit L(M) = L gibt. Wir schreiben \textbf{REC} für die Klasse der entscheidbaren Sprachen. Der Begriff entscheidbar für Sprachen ergibt sich hier daraus, dass effektiv entschieden werden kann ob eine gegebene Eingabe in der Sprache liegt oder nicht. Insbesondere steht? dies voraus, dass Eingabe, die nicht in der Sprache liegen effektiv als nicht in der Sprache liegend erkannt werden. 
\paragraph*{Begriff: } 
  effektiv \(\leadsto\) eine TM erlefigt dies in endicher Zeit. Da sich der durch TM formatierte  Berechenbarkeitsbegriff, also die Formalisierung dessen was effektiv durchführbar ist, auch äquivalent durch rekursive Funktion definieren lässt, weden entscheidbare Sprachen auch als rekuriv bezeichnet.

\mysubsection{Definition(rekursiv aufzählbar)} 
  Eine Sprache L ist genau dann \textbf{rekursiv aufzähbar}, wenn es eine k-TM mit akzeptierten Sprache L gibt. Wir schreiben \textbf{RE} für die Klasse der rekursiv aufzählbaren Sprachen. Die Aufzählbarkeit leitet sich daraus ab, dass es für eine rekuriv aufzählbare Sprache L über einem Alphabet \(\Sigma\) möglich ist effektive Verfahren anzugeben ,die die Wörter von L aufzählen, also dass eine endlich oder unendliche Aufzählung von \(A = w_1, w_2, \cdots \) mit \( {w_1, w_2, \cdots} = L\) existiert.

\paragraph*{Bemerkung vom Author: }
  Rekursiv aufzählbar" ist ein Begriff der verwendet wird um eine Menge zu beschreiben, die wir mit einem Computerprogramm oder Algorithmus "auflisten" können. Stellen Sie sich vor, Sie haben eine Box mit nummerierten Bällen, und Sie haben ein Programm, das Bälle aus der Box zieht. Wenn Sie sicherstellen können, dass Sie jeden Ball in der Box mindestens einmal ziehen, egal wie lange es dauert, dann ist die Menge der Bälle in der Box "rekursiv aufzählbar

\paragraph*{Bemerkung vom Author: }
  Wenn wir sagen, dass eine Sprache "rekursiv aufzählbar" ist, bedeutet das, dass es einen Algorithmus oder ein Computerprogramm gibt, das alle Wörter in dieser Sprache "auflisten" kann. Es könnte einige Wörter mehrmals auflisten und es könnte eine sehr lange Zeit dauern, aber es würde schließlich jedes Wort in der Sprache "treffen". Eine "k-TM" ist eine Art von Maschine, die wir in der theoretischen Informatik verwenden, um diese Art von Aufzählung zu machen. Wenn es eine k-TM gibt, die eine Sprache akzeptiert, bedeutet das, dass die Sprache rekursiv aufzählbar ist.

\mysubsection{Bemerkung} 
  Jede entscheidbare Sprache ist rekursiv aufzähbar.

\mysubsection{Bemerkung} 
  Alle endlichen Sprachen sind entscheidbar.

\mysubsection{Bemerkung} 
  Eine Sprache L über einem Alphabet \(\Sigma\) ist genau dann entscheidbar, wenn L und \(L^c :=(\Sigma^*)/L\) rekursiv aufzähbar sind.

\mysubsection{Definition (Ausgabe)} 
  Sei \(M = (Q, \Sigma, \Gamma, \Delta, s, F)\) eine k-TM und \(C = (q, w_1, \cdots, w_k, p_1, \cdots, p_k)\) eine Konfigration von M.Die Ausgabe \(out_M(C)\) von M bei Konfiguration C ist das längste Präfix w, das aus den Symbolen der Bänder von M besteht und den folgenden Bedingungen genügt: \(w \in (\Gamma / {\Box})^*\), \( w_1(p_1), \cdots, w_1(|w_1|)\) sind Präfixe von w.\footnote{Ed. sug. text}


\mysubsection{Definition (berechnete Funktion)} 
  Sei \(M = (Q, \Sigma, \Gamma, \Delta, s, F)\) eine k-DTM und \(n \in \mathbb{N}\). Die von M berechnete \textbf{n-äre partielle Funktion} \(\Phi_M\) ist die partielle Funktion \(\Phi_M : (\Sigma^*)^n \leadsto (\Gamma / {\Box})^*\), so dass \(\forall (w_1, \cdots, w_n) \in (\Sigma^*)^n\) folgendes gilt:
  \begin{enumerate}
    \item Ist die rechnung von M zur Eingabe \((w_1, \cdots, w_n)\) die endliche Rechnung \(C_1, \cdots, C_M\), so gilt \(\Phi_M(w_1, \cdots, w_n) = out_M(C_M)\).
    \item Ist die Rechnung von M zur Eingabe \((w_1, \cdots, w_n)\) unendlich, so gilt \(\Phi_M(w_1, \cdots, w_n)\uparrow\) 
  \end{enumerate}
  Für \(w_1, \cdots, w_n \in \Sigma^*\) schreiben wir statt \(\Phi_M(w_1, \cdots, w_n)\) auch \(M(w_1, \cdots, w_n)\).

\mysubsection{Definition (partiell berechenbar)}
  Für Alphabet \(\Sigma, \Gamma\) und eine partielle Funktion \(\Phi : \Sigma^* \leadsto \Gamma^*\) ist \(\Phi\) \textbf{partiell berechenbar}, wenn es eine \(k \in \mathbb{N}\) gibt und eine k-DTM M mit \(\Phi_M = \Phi\) gibt. Ist \(\Phi\) total und partiell berechenbar, so ist \(\Phi\) berechenbar. Wir schreiben \textbf{RF} für die Klasse der partiellen Funktionen.\\Mittels der Induktivität von \(\mathbb{N}_0\) und\( {0, 1}^*\) können so auch partielle Funktionen, die von oder nach \(\mathbb{N}_0\) abbilden als (partielle) berechenbare Funktion bezeichnent werden. Beispielsweise ist eine partielle Funktion \(\Phi : \mathbb{N}_0 \leadsto \mathbb{N}_0\) dennoch genau dann partiell berechenbar, wenn die partielle Funktion bin \(\circ \Phi \circ bin^{-1}\) partiell berechenbar ist. Gewissermaßen verfügen die hier definierten TM über zewi Ausgabemechanismen. Die Ausgabeim engeren Sinne in Definition 2.13 und das Ablesen vn Akzeptanz anhand des schließlich erreichten Zustands in Definition 2.7. Im Sinne der folgenden Bemerkung wäre der zweiten Fall nicht notwendig, allerdings ist dies ein wichtiger spezialfall.

\mysubsection{Definition (charackteristische Funktion, partielle charachteristische Funktion)} 
  Sei L eine Sprache über dem Alphabet \(\Sigma\)
  \begin{itemize}
    \item [(i)] Die \textbf{charackteristische Funktion} von L als Sprache über \(\Sigma\) ist die Funktion \(\mathbbm{1}_L : \Sigma \rightarrow \{0, 1\}\) mit \(\mathbbm{1}_L = 1 \forall w \in L\) und \(\mathbbm{1}_L (u) = 0 \forall w \in \Sigma^* / L\).
    \item [(ii)] Die \textbf{ partielle charackteristische Funktion} von L als Sprache über \(\Sigma\) ist die partielle Funktion \(x_L : \Sigma^* \leadsto \{1\} mit x_L(w) = 1 \forall w \in L und x_L(w) \uparrow  \forall w \in \Sigma^* / L\). 
  \end{itemize}

\mysubsection{Bemerkung} 
  Sei L eine Sprache über einem Alphabet \(\Sigma\). 
  \begin{itemize}
    \item [(i)] L ist genau dann entscheidbar, wenn \(\mathbbm{1}_L\) berechenbar ist.
    \item [(ii)] L ist genau dann rekursiv aufzähbar, wenn \(x_L\) partiell berechenbar ist.
  \end{itemize}

\newpage

\mysubsection{Bemerkung (normiert)} 
  Eine 1-DTM \(M = (Q, \Sigma, \Gamma, \Delta, s, F)\) heißt \textbf{normiert}, wenn \(Q = {0,\cdots, n}\) für eine \(n \in \mathbb{N}_{0}, \Sigma = {0, 1}, \Gamma = {\Box, 0, 1}, s = 0, F = {s}\). Alle TMs mit Eingabealphabet {0,1} lassen sich mit folgenden Schritten in eine normierte TM mit gleicher erkannter Sprache und gleicher berechneter Funktion umwandeln. 
\paragraph*{Von Nichtdeterminismus zu Determinismus:} 
  Eine DTM kann die Rechnungen einer nichtdeterministischen TM parallel im Sinne von abwechend schrittweise durchführen um schließlich das Verhalten der simulierten TM zu ??. Dies entspricht einer \textbf{Breitensuche im Rechnungsbaum}.
  \createDiagram{Breitensuche}
  {}

  \begin{tikzpicture}
    [
      level distance=1.5cm,  level 1/.style={sibling distance=3cm},  
      level 2/.style={sibling distance=1.5cm},  
      every node/.style={align=center, text=textcolor}
    ] 
    \node (0) {\(wort_M(w)\)}
      child 
      {
        node (1) {\(C_1\)}
        child {node (3) {\(C_3\)}}
        child {node (4) {\(C_4\)}}
      }
      child 
      {
        node (2) {\(C_2\)}
        child {node (5) {\(C_5\)}}
        child {node (6) {\(C_6\)}}
      };

    \path[->,red,thick] 
                      (0) edge (1)
                      (1) edge (2)
                      (2) edge (3)
                      (3) edge (4)
                      (4) edge (5)
                      (5) edge (6);
  \end{tikzpicture}

  \paragraph{Von mehreren Bändern zu einem Band}: Intuitiv können k Bänder auf ein Band simuliert werden, indem die Felder des einen Bandes in k-teilfelder unterteilt werden, die jeweils die gleiche Bandalphabetbuchstaben wie zufor als Beschreibung zulassen und es zudem erlaubt zu markieren, dass der simulierte Kopf des simulierten Bandes dort steht. Eine dieser Idee folgende Konstruktion wird als \textbf{Spurentechnik} bezeichnet. Formal: Übergang vom Bandalphabet \(\Gamma\) zu 
  \[
    ((\Gamma \cup{\underline{a} : a \in \Gamma})^{k}/{\Box}^{k}) \cup {\Box}
  \] 
  wobei \(\underline{a} \not \in \Gamma für a \in \Gamma\). Hierbei bedeutet \(\underline{a}\), dass das simulierte Feld mit a beschriftet ist und dass dort der simulierte Kopf steht. Weiter spielt \(\Box\) die Rolle des k-Tupels \((\Box, \cdots, \Box)\) um der Tatsache gerecht zu werden, dan alle Felderzu Begin mit \(\Box\) beschriftet sind.
  
  \createDiagram{Spurentechnik}
  {
    \begin{tikzpicture}[cell/.style={rectangle, draw=black, minimum size=1cm}, node distance=0cm]

      % Erstes Band
      \node[cell] (cell11) {...};
      \node[cell, right=of cell11, draw=red] (cell12) {0};
      \node[cell, right=of cell12, draw=accentcolor] (cell13) {0};
      \node[cell, right=of cell13, draw=accentcolor] (cell14) {1};
      \node[cell, right=of cell14, draw=accentcolor] (cell15) {...};
      
      % Zweites Band
      \node[cell, below=0.5cm of cell11, draw=accentcolor] (cell21) {...};
      \node[cell, right=of cell21, draw=accentcolor] (cell22) {1};
      \node[cell, right=of cell22, draw=accentcolor] (cell23) {0};
      \node[cell, right=of cell23, draw=red] (cell24) {0};
      \node[cell, right=of cell24, draw=accentcolor] (cell25) {...};
      
      % Drittes Band
      \node[cell, below=0.5cm of cell21, draw=accentcolor] (cell31) {...};
      \node[cell, right=of cell31, draw=accentcolor] (cell32) {0};
      \node[cell, right=of cell32, draw=red] (cell33) {1};
      \node[cell, right=of cell33, draw=accentcolor] (cell34) {1};
      \node[cell, right=of cell34, draw=accentcolor] (cell35) {...};
      
      % Vertikales Band
      \node[cell, right=5cm of cell22, minimum height=4.13cm, align=center, draw=accentcolor] (cell41) {...};
      \node[cell, right=0cm of cell41, minimum height=4.13cm, align=center, draw=red] (cell42) {1 \\\\\\ 0 \\\\\\ 1};
      \node[cell, right=0cm of cell42, minimum height=4.13cm, align=center, draw=accentcolor] (cell43) {1 \\\\\\ 0 \\\\\\ 1};
      \node[cell, right=0cm of cell43, minimum height=4.13cm, align=center, draw=accentcolor] (cell44) {1 \\\\\\ 0 \\\\\\ 1};
      \node[cell, right=0cm of cell44, minimum height=4.13cm, align=center, draw=accentcolor] (cell45) {...};


      % Pfeil
      \draw[->, very thick] ([yshift=0.25cm]cell12.north) -- (cell12.north);

      % Pfeil
      \draw[->, very thick] ([yshift=0.25cm]cell24.north) -- (cell24.north);

      % Pfeil
      \draw[->, very thick] ([yshift=0.25cm]cell33.north) -- (cell33.north);

      % Pfeil
      \draw[->, very thick] ([yshift=0.25cm]cell42.north) -- (cell42.north);

      % Buchstabe am Pfeil
      \node[above=0.2cm of cell12.north] {\(\varphi\)};

      % Buchstabe am Pfeil
      \node[above=0.2cm of cell42.north] {q};

      % Pfeil von links nach rechts über den Bändern
      \draw[->, very thick, black] (cell25.east) -- (cell41.west);

    \end{tikzpicture}
  }

  \newpage
    
  \paragraph{Von beliebigen bandalphabet zu \(\{\Box, 0, 1\}\)}: Andere bandalphabete können bei einem \textbf{Alphabetwechel} zum Bandalphabet \(\{\Box, 0, 1\}\) simuliert werden um ein Symbol des vorherigen Bandlaphabets durch ein Binärwort zu beschreiben. Die TM liest stets nur ein Feld, es wird dabei also nötig sein die Zustandsmenge so zu erweitern, dass angrenzende Felder im Zustand gespeichert weden können.
  
  \usetikzlibrary{arrows, positioning, calc}
  \createDiagram{Alphabetwechel}
  {}
    \begin{tikzpicture}[cell/.style={rectangle, draw=black, minimum size=1cm}, arrow/.style={->, >=stealth, thick, shorten <=1pt, shorten >=1pt}, dashedline/.style={dashed, shorten <=1pt, shorten >=1pt}]

    % Oberes Band
    \foreach \i/\label in {1/A,2/B,3/C,4/D,5/E,6/F,7/G} 
    {
      \node[cell] (ucell\i) at (\i, 0) {\label};
    }
    
    % Unteres Band
    \foreach \i/\label in {0/1,1/2,2/3,3/4,4/5,5/6,6/7,7/8,8/9,9/10,10/11,11/12,12/13,13/14} 
    {
      \node[cell] (lcell\i) at (\i/2+0.25, -3.5) {\label};
    }
    
    % Verbindungslinien
    \foreach \i in {1,...,7} 
    {
      \pgfmathtruncatemacro\j{2*\i-2}
      \pgfmathtruncatemacro\k{2*\i-1}
      \draw[dashedline] (ucell\i.south west) -- (lcell\k.north west);
    }
    
    % Pfeil auf die erste Zelle
    \draw[arrow] (1,2) -- (ucell1);

    \end{tikzpicture}

\mysubsection{Bemerkung} 
  Sei \(L \subseteq \{0, 1\}^*\) eine Sprache und sei \(\Phi : \{0, 1\}^* \leadsto \{0, 1\}^*\) eine partielle Funktion.
  \begin{itemize}
    \item [(i)] L ist genau dann entscheidbar, wenn L akzeptierte Sprache einer totalen normierten TM ist. 
    \item [(ii)] L ist genau dann rekursiv aufzähbar, wenn L akzeptierte Sprache einer normierten TM ist.
    \item [(iii)] \(\Phi\) ist genau dann partiell berechenbar, wenn \(\Phi\) berechnete Funktion einer normierten TM ist.
  \end{itemize}

\paragraph*{Church- Turing- These} Berechenbarkeit auf eienr Turingmachine entspricht intuitiver Berechenbarkeit.

\coversection{Berechenbarkeit/predic.png}{Berechenbarkeit}{Predictability is like knowing the path a river takes. The river starts at its source and flows down to the sea. Along the way, it may turn, twist, and divide, but it always follows the path of least resistance due to gravity. Knowing the terrain allows us to predict where the river will go.\\ \hspace*{\fill} - ChatGPT}
\textbf{Konvention: } Sprechen wir von einer $e \in \mathbb{N}_0$ oder $(e_1, \cdots, e_n) \in \mathbb{N}_0^n$ wobei $n \in \mathbb{N}$ als Eingabe für eine TM oder Ausgabe einer TM, so bedetet dies, dass die Eingabe bzw. Ausgabe $bin(e)$ bzw $(bin(e_1), \cdots, bin(e_n))$ ist. Dies erlaubt es über partiell berechnenbare Funktionene $\Phi: \mathbb{N}_0^n \leadsto \mathbb{N}_0$ wobei $n\in \mathbb{N}$ zu sprechen und $L \subseteq \mathbb{N}_0$ als Sprache über $\{0, 1\}$ aufzufassen.

\mysubsection{Definition (Code)} Wir betrachten die Funktion code (mit geeignetem Definitionsbereich) und Zielmenge $\{0, 1\}^*$, für die folgendes gilt. Zunächst gelte \[code(L) = 10 \qquad code (S) = 00 \qquad code(R) = 01\] Für eine Instruktion $ I = (q, a, q', a', B) \in \mathbb{N}_0 \times \{0, 1\} \to \mathbb{N}_0 \times \{0, 1\} \times \{L, S, R\}$ einer normierten TM sei \[code (I) = 0^{|bin(q)|} 1 bin (q) a 0^{|bin(q')|} 1 bin (q') a' code (B)\] Für eine endliche Menge $ \Delta \subseteq \mathbb{N}_0 \times \{0, 1\} \to \mathbb{N}_0 \times \{0, 1\}\times \{L, S, R\}$ von Instruktionen einer normierten TM und $i \in [|\Delta|]$ sein $code_i(\Delta)$ dann ein längenlexikographische Ordnung i-te Wort in $\{code(I): I \in \Delta\}$ und sei \[ code (\Delta) = code_1(\Delta), \cdots, code_{|\Delta|}(\Delta)\] Für eine normierte TM $M = (\{0, \cdots, n\}, \{0, 1\}, \{\Box, 0, 1\}, \Delta, 0, \{0\})$ sei \[ code (M) = 0^{|bin(n)|} 1 bin (n) code (\Delta)\] der \textbf{Code} von $M$. Relevant ist hierbei dass es eine geeignete effektive Codierung von Turingmachinen durch Binärwörter gibt, so dass folgendes gilt 
\begin{itemize}
  \item Jede normierte TM hat einen Code 
  \item Keine zwei verschiedene normierten TMs haben den gleichen Code.
  \item Die Sprache der Codes von Turingmachinen ist entscheidbar
  \item Codes können eine geeignete Repräsentation der durch sie codierten TMs umgewandelt werden, die es insbesondere erlauben die codierten TMs effekiv zu simulieren.
  \item geignete Repräsentationen von TMs können effektiv in ihre Codes umgewandet werden.
\end{itemize}

\mysubsection{Definition (standardaufzählung)} Sei $\hat{w_0}, \hat{w_1}, \cdots$ die Aufzählung aller Codes normierter TMs in längenlexikographischer Ordnung. Für $e \in \mathbb{N}_0$ sei $M_e$ die durch $\hat{w_e}$ codierte TM und für $n \in \mathbb{N}$ sei $\Phi_e^n : \mathbb{N}_0^n \rightarrow \mathbb{N}_0$ die von $M_e$ berechnete n-äre partielle Funktion. Für $n \in \mathbb{N}$ heißt die Folge $(\Phi_e^n)$ mit $e\in \mathbb{N}$ \textbf{standardaufzählung} der n-ären partiell berechenbaren Funktion. Für $n \in \mathbb{N} $ und eine partiell berechenbare n-äre Funktion $\varphi: \mathbb{N}_0^n \rightarrow \mathbb{N}_0$ heißt jede zahl $e \in \mathbb{N}_0$ mit $\Phi_e^n = \varphi$ \textbf{Index} von $\varphi$.

\textbf{Konvention: } Ergibt sich n aus dem Kontext, so schreiben wir auch $\Phi_e$ statt $\Phi_e^n$

\mysubsection{Bemerkung} Für $n \in \mathbb{N}$ und eine partielle berechnbare n-äre partielle Funktion $\Phi : \mathbb{N}_0^n \rightarrow \mathbb{N}_0$ gibt es unendlich viele Indizes von $\varphi$.

\mysubsection{Definition (U)} Es bezeichnet U die normierte TM, die bei Eingabe $(e, x_1, \cdots, x_n) \in \mathbb{N}_0^{n+1}$ wobei $n \in \mathbb{N}$ die normierte TM $\mathcal{M}_e$ bei Eingabe $(x_1, \cdots, x_n)$ simuliert und falls diese terminiert die Ausgabe der Simulierten ausgibt.

\mysubsection{Definition (Universell)} Eine DTM U heißt \textbf{Universell}, wenn es für alle $n \in \mathbb{N}$ und alle partiell berechenbaren Funktionen $\varphi : \mathbb{N}_0^n \leadsto \mathbb{N}_0$ eine $e \in \mathbb{N}$, so dass \[U(e, x_1, \cdots, x_n) = \varphi(x_1, \cdots, x_n)\] $\forall x_1, \cdots, x_n \in \mathbb{N}_0$ gilt.

\mysubsection{Bemerkung} \sloppy Die TM U ist universell, denn für $e \in \mathbb{N}_0$, $n \in \mathbb{N}$ und $x_1, \cdots, x_n \in \mathbb{N}_0$ gilt \[U(e,x_1, \cdots, x_n) = \Phi_e(x_1, \cdots, x_n)\]\\ \[(x, y) \mapsto x^y\] \[y \mapsto 2^y\] \[(x_1, \cdots, x_m, y_1, \cdots, y_n) \mapsto \varphi(x_1, \cdots, x_m, y_1, \cdots, y_n) partiell berechenbar\] \[\leadsto (y_1,\cdots, y_m) \mapsto \varphi(x_1, \cdots, x_m, y_1, \cdots, y_n) partiell berechenbar \]

\mysubsection{Satz ($s_n^m$ - Theorem)} 
$\forall m, n \in \mathbb{N}$ existiert eine berechenbare Funktion $s_n^m : \mathbb{N}_0^{m+1} \to \mathbb{N}_0$  mit \[\Phi_e^{m+1}(x_1\cdots, x_m, y_1, \cdots, y_n) = \Phi_{s_n^m(e, x_1, \cdots, x_m)}^n (y_1, \cdots, y_n)\] $\forall e, x_1, \cdots, x_m, y_1, \cdots, y_n \in \mathbb{N}_0$

\begin{proof}Fixiere $m \in \mathbb{N}$. Betrachte die DTM S , die bei Eingabe $(e, x_1, \cdots, x_m) \in \mathbb{N}_0^{m+1}$ wie folgt vorfährt.
\begin{itemize}
  \item Zunächst bestimmt S den Code von $\mathcal{M}_e$ 
  \item der Code von $\mathcal{M_e}$wird dann in einen Code einer normierten TM $\mathcal{M}$ umgewandet, die zunächst $x_1\Box\cdots\Box x_m\Box$ neben die Eingabe schreibt, dan den Kopf auf das erste Feld des beschriebenen Bandteilsbewegt und dann wie $\mathcal{M_e}$ arbeitet.
  \item Es wird bestimmt an welcher Stelle der Standardaufzählung der Code von auftaucht und diese Stelle wird ausgegeben.
\end{itemize}
Sei $s_n^m$ die von S berechnete $(m+1)$-äre partielle Funktion. Dann ist $s_n^m$ eine Funktion wie gewünscht. Es gibt überabzählbar viele Binärsprachen, denn: Betrachte Aufzählung von Binärsprachen $L_1, L_2, \cdots$ 
\begin{table}[ht]
  \centering
  \renewcommand{\arraystretch}{2} % Adjust the value to increase or decrease the cell size
  \begin{tabular}{c c c c c}
    \tikzmarknode{L0-0}{$\mathds{1}_{L_0}(0)$} & $\mathds{1}_{L_0}(1)$ & $\mathds{1}_{L_0}(2)$ & $\mathds{1}_{L_0}(3)$ \\
    $\mathds{1}_{L_1}(0)$ & $\mathds{1}_{L_1}(1)$ & $\mathds{1}_{L_1}(2)$ & $\mathds{1}_{L_1}(3)$ \\
    $\mathds{1}_{L_2}(0)$ & $\mathds{1}_{L_2}(1)$ & \tikzmarknode{L2-2}{$\mathds{1}_{L_2}(2)$} & $\mathds{1}_{L_2}(3)$ \\
  \end{tabular}
  \captionsetup{labelformat=empty, justification=centering, skip=10pt}
  \caption{Standardaufzählung}
  
  \begin{tikzpicture}[overlay, remember picture, red, >=stealth]
    \draw [->, thick] ([yshift=1ex]L0-0.south) -- ([yshift=-1ex]L2-2.north);
  \end{tikzpicture}
\end{table}

$L$ mit $\mathds{1}_L(i)$ = 
$\begin{cases}
    0, & \text{wenn } \mathds{1}_{L_i}(i) = 1 \\
    1, & \text{wenn } \mathds{1}_{L_i}(i) = 0 \\
\end{cases}$
\end{proof}

\mysubsection{Definition (diagonales Halteproblem)} Die Menge $H_{diag} := \{e \in \mathbb{N}_0 : \Phi_e (e) \downarrow\}$ heißt \textbf{diagonales Halteproblem}.

\mysubsection{Proposition} Das diagonale Halteproblem ist rekursiv aufzählbar.
\begin{proof}
  Die DTM, die bei Eingabe $e \in \mathbb{N}_0$ wie $U$ bei Eingabe $(e, e)$ arbeitet, aber bei terminieren $1$ statt der Ausgabe von $U$ ausgibt berechnet die partielle charachteristische Funktion von $H_{diag}$. Die partielle Funktion $x_{H_{diag}}$ ist also partiell berechenbar. Die partielle Funktion $x_{H_{diag}^c}$ ist nicht partiell berechenbar, dann: Betrachte Standardaufzählung
\end {proof}
  
\begin{table}[ht]
  \centering
  \renewcommand{\arraystretch}{2} % Adjust the value to increase or decrease the cell size
  \begin{tabular}{c c c c c}
    \tikzmarknode{L0-0}{$\Phi_{L_0}(0)$} & $\Phi_{L_0}(1)$ & $\Phi_{L_0}(2)$ & $\Phi_{L_0}(3)$ \\
    $\Phi_{L_1}(0)$ & $\Phi_{L_1}(1)$ & $\Phi_{L_1}(2)$ & $\Phi_{L_1}(3)$ \\
    $\Phi_{L_2}(0)$ & $\Phi_{L_2}(1)$ & \tikzmarknode{L2-2}{$\Phi_{L_2}(2)$} & $\Phi_{L_2}(3)$ \\
  \end{tabular}
  \captionsetup{labelformat=empty, justification=centering, skip=10pt}
  \caption{Standardaufzählung}
\end{table}

\begin{tikzpicture}[overlay, remember picture, green, >=stealth]
  \draw [->, thick] ([yshift=1ex]L0-0.south) -- ([yshift=-1ex]L2-2.north);
\end{tikzpicture}

$\varphi$ mit $\varphi(i)$ = 
$\begin{cases}
    \uparrow, & \text{wenn } \Phi_i(i) \downarrow\\
    \downarrow, & \text{wenn } \Phi_i(i) \uparrow \\
\end{cases}$
Wird nicht aufgezählt.

\mysubsection{Satz} Das diagonale Halteproblem ist nicht entscheidbar. 
\begin{proof}
  Angenommen $H_{diag}$ wäre entscheidbar. Dann wäre die partielle charakteristische Funktion $\varphi$ von $H_{diag}^c = \mathbb{N}_0 / H_{diag}$ partiell berechenbar, es gäbe also ein Index $e \in \mathbb{N}_0$ von $\varphi$. Es folge \[e \in H_{diag}^c \Leftrightarrow \varphi(e) \downarrow \Leftrightarrow \Phi_e(e) \downarrow \Leftrightarrow e \in H_{diag} \Leftrightarrow e \not \in H_{diag}^c\] Die ist ein Wiederspruch.
\end{proof}

\mysubsection{m-Reduktion} Für eine Sprache $A$ über einem Alphabet $\Sigma$ und eine Sprache $B$ über einem Alphabet $\Gamma$ ist A genau dann \textbf{many-one-reduzierbar}, auch \textbf{m-reduzierbar}, auf $B$, kurz $A \leq_m B$, wenn es eine berechebare Funktion. $f: \Sigma^* \to \Gamma^*$ gibt so dass \[w \in A \Leftrightarrow f(w)\in B\] $\forall w \in \Sigma^*$ gilt. Gelten $A \leq_m B$ und $B \leq_{m} A$, so sind $A$ und $B$ \textbf{many-one-äquivalent} auch \textbf{m-äquivalent}, kurz $A =_m B$.

\mysubsection{Bemerkung} 
\begin{itemize}
  \item [(i)] $\leq_m$ ist transitiv.
  \item [(ii)] Gilt $A \leq_m B$ für Sprachen $A$ und $B$ und ist $B$ entscheidbar, so ist auch $A$ entscheidbar.
  \item [(iii)] Alle entscheidbaren Sprachen L mit $\varnothing \not = L \not = \mathbb{N}_0$ und m-äquivalent.
\end{itemize}

\mysubsection{Satz} Das \textbf{initiale Halteproblem} $H_{init} = {e \in \mathbb{N}_0 = \Phi_e(0) \downarrow}$ ist nicht entscheidbar.

\subsubsection*{Idee: } suche $f:\mathbb{N}_0 \to \mathbb{N}_0$ mit $\Phi_e(e)\downarrow \Leftrightarrow \Phi_{f(e)}(0)\downarrow$ Wähle $f$ so dass $\Phi_{f(e)}(x) = \Phi_e(e)$ $\forall x \in \mathbb{N}_0$

\begin{proof}
  Sei $\psi : \mathbb{N}_0^2 \leadsto \mathbb{N}_0$ mit $\psi (e, x) = \Phi_e(e) \forall e, x \in \mathbb{N}_0$. Dann ist $\psi$ partiell berechenbar. Sei $e_0$ ein Index von $\psi$ und $s:\mathbb{N}_0^2 \to \mathbb{N}_0$ gilt. Sei $f: \mathbb{N}_0 \to \mathbb{N}_0$ mit $f(e) = s(e_0, e) \forall e /in \mathbb{N}_0$. Dann ist f berechenbar. $\forall e \in \mathbb{N}_0$ gilt. \[e \in H_{diag} \Leftrightarrow \Phi_e(e) \downarrow \Leftrightarrow \psi(e, 0) \downarrow \Leftrightarrow \Phi_{e_0}(e, 0) \downarrow \Leftrightarrow \Phi_s (e_0, e)(0)\downarrow \Leftrightarrow \Phi_{f(e)} (0)\downarrow \Leftrightarrow f(e) \in H_{init}\] Es gilt also $H_{diag} \leq_{m} H_{init}$, da $H_{diag}$ nicht entscheidbar ist, ist damit $H_{init}$ nicht entscheidbar.
\end{proof}

\textbf{Dominosteinspiel!}
\subsubsection*{Gegeben: } Endlich viele typen von Spielsteinen mit jeweils zwei beschrifteten Feldern: "oberes Feld, unteres Feld". Beschritungen sind nichtleere Wörter über einem Alphabet. Spielsteine sind vom gleichen Typ, wenn die beiden oberen Felder gleich beschriftet sind und die beiden unteren Felder gleich beschriftet sind. Es gibt von jedem Typ beliebig viele steine. 

\subsubsection*{Gesucht: }Können ein oder mehrere (aber endlich viele) Spielsteine so nebeneinander gelegt werden, dass sich oben und unten von links nach rechts gelesen das gleiche Wort ergibt?
\begin{center}
  \begin{tikzpicture}
    % Define styles for Dominoes
    \tikzstyle{domino}=[rectangle, draw, minimum width=1cm, minimum height=2cm, inner sep=0pt]
  
    % Domino (0111, 0)
    \node[domino] (d1) at (0,0) {};
    \node at ([yshift=4mm]d1.center) {0111};
    \node at ([yshift=-4mm]d1.center) {0};
  
    % Domino (1, 01)
    \node[domino] (d2) at (1.5,0) {};
    \node at ([yshift=4mm]d2.center) {1};
    \node at ([yshift=-4mm]d2.center) {01};
  
    % Domino (0, 1)
    \node[domino] (d3) at (3,0) {};
    \node at ([yshift=4mm]d3.center) {0};
    \node at ([yshift=-4mm]d3.center) {1};
  
    % Domino (0, 000)
    \node[domino] (d4) at (4.5,0) {};
    \node at ([yshift=4mm]d4.center) {0};
    \node at ([yshift=-4mm]d4.center) {000};
  
    % Domino (1, 011)
    \node[domino] (d5) at (6,0) {};
    \node at ([yshift=4mm]d5.center) {1};
    \node at ([yshift=-4mm]d5.center) {011};
  
    \end{tikzpicture}
    \vspace{1cm}\\
    \begin{tikzpicture}

      % Define styles for Dominoes
      \tikzstyle{domino}=[rectangle, draw, minimum width=1cm, minimum height=2cm, inner sep=0pt]
      
      % Domino (0111, 0)
      \node[domino] (d1) at (0,0) {};
      \node at ([yshift=4mm]d1.center) {0111};
      \node at ([yshift=-4mm]d1.center) {0};
      
      % Domino (0, 1)
      \node[domino] (d2) at (1,0) {};
      \node at ([yshift=4mm]d2.center) {0};
      \node at ([yshift=-4mm]d2.center) {1};
      
      % Domino (0, 1)
      \node[domino] (d3) at (2,0) {};
      \node at ([yshift=4mm]d3.center) {0};
      \node at ([yshift=-4mm]d3.center) {1};
      
      % Domino (0, 1)
      \node[domino] (d4) at (3,0) {};
      \node at ([yshift=4mm]d4.center) {0};
      \node at ([yshift=-4mm]d4.center) {1};
      
      % Domino (0, 000)
      \node[domino] (d5) at (4,0) {};
      \node at ([yshift=4mm]d5.center) {0};
      \node at ([yshift=-4mm]d5.center) {000};
      
      % Domino (1, 01)
      \node[domino] (d6) at (5,0) {};
      \node at ([yshift=4mm]d6.center) {1};
      \node at ([yshift=-4mm]d6.center) {01};
      
      % Arrow pointing to the leftmost domino
      \draw[->, thick] (-1.5,0) -- (d1.west);

      \end{tikzpicture}
        
  \end{center}

  \mysubsection{Definition (Postsches Korrespondenzproblem, Emil Port, 1946)} Für ein Alphabet $\Sigma$ sei eine Instanz des Postschen Korrespondenzproblems über $\Sigma$ eine endliche Teilmenge $I \subseteq (\Sigma^+)^2$. Eine Lösung für eine solche Instanz ist eine endliche Folge $(u_1, v_1), \cdots, (u_n, v_n)$ von Paaren in $I$ mit $n \geq 1$,so dass \[u_1 \cdots u_n = v_1 \cdots v_n\] Gibt es eine Lösung für eine instanz des Postschen Korrespondenzproblems, so heißt diese Instanz lösbar. Das \textbf{Postsche Korrespondenzproblem} über einem Alphabet $\Sigma$, kurz $PCP_{\Sigma}$ ist die Menge aller lösbaren Instanzen des Postschen Korrespondenzproblems über $\Sigma$.\\\\ Für ein Alphabet $\Sigma$ sei eine Instanz des modifizierten Postschen Korrespondenzproblems über $\Sigma$ ein Paar $(p, I)$, wobei $I\subseteq (\Sigma^+)^2$ eine endliche Teilmenge und $p\in I$ ein Paar von Wörtern ist. Eine Lösung für eine solche Instanz ist eine endlcihe Folge $(u_1, v_1), \cdots, (u_n, v_n)$ von Paaren ist $I$, so dass \[p = (u_1, v_1) und u_1\cdots u_n = v_1\cdots v_n\] Gibt es eine Lösung für eine Instanz des modifizierten Postschen Korrespondenzproblems so heißt diese Instanz lösbar. Das \textbf{modifizierte Postsche Korrespondenzproblem} über einem Alphabet $\Sigma$, kurz $MPCP_{\Sigma}$ ist die Menge aller lösbaren Instanzen des modifizierten Postschen Korrespondenzproblems über $\Sigma$.
  \subsubsection*{Plan: } Für Alphabet mit $|\Sigma| \geq 2$: \[H_{init} \stackrel{(3)}{\leq_m} MPCP_{\Gamma} \stackrel{(2)}{\leq_m} PCP_{\Gamma} \stackrel{(1)}{\leq_m} PCP_{\Sigma}\]
  \mysubsection{Lemma} Für ein Alphabet $\Sigma$ und $\Gamma$ mit $|\Sigma| \geq w$ gilt $PCP_{\Gamma} \leq_m PCP_{\Sigma}$
  \begin{proof}
    Wir suchen eine effektive Transformation, die jede Instanz $I$ des Postschen Korrespondenzproblems über $\Gamma$ in eine Instanz $I'$ des postschen Korrespondenzproblems über $\Sigma$ transformiert, so dass $I$ genau dann lösbar ist, wenn $I'$ lösbar ist. Seien $a_1, a_2 \in \Sigma$ verschieden und sein $b_1, \cdots, b_{|\Gamma|}$ die Elemente von $\Gamma$. Es bezeichne $\varphi : \Gamma^* \to \Gamma^*$ den eindeutigen Homomorphismus von Sprachen mit $\varphi (b_i) = a^i_1 a_2$ $\forall i \in [|\Gamma|]$. Gegeben eine solche Instanz $I$ wie oben sei $I' := \{(\varphi(u), \varphi(v)) : (u, v) \in I\}$. Die Funktion, die geeignete Codes von Instanzen $I$ auf geeignete Codes von Instanzen $I'$ abbildet ist berechenbar. Ist $(u_1, v_1), \cdots, (u_n, v_n)$ eine lösung $I$, so gilt \[\varphi(u_1) \cdots \varphi(v_1) = \varphi(u_1, \cdots, \varphi(v_n)) = \varphi(v_1, \cdots, v_n) = \varphi(v_1) \cdots \varphi(v_n)\] und somit ist $(\varphi(u_1), \varphi(v_1), \cdots, (\varphi(u_n)), \varphi(v_n))$ eine Lösung von $I'$. Die Instanz $I'$ ist also lösbar wenn $I$ lösbar ist. Ist $(u'_1, v'_1), \cdots, (u'_n, v'_n)$ eine Lösung von $I'$, so gibt es eine Folge $(u_1, v_1)\cdots (u_n, v_n)$ von Paaren in $I$ mit $\varphi(u'_i)$ und $\varphi(v_i) = v'_i$ $\forall i \in [n]$, also mit \[\varphi(u_1, \cdots, u_n) = u'_1, \cdots, u'_n = u'_1, \cdots, u'_n =\varphi(u_1, \cdots, u_n)\] Da $\varphi \vert_{\Sigma}$ injektiv und $\varphi (\Sigma)$ präfixfrei ist, ist $\varphi$ injektiv (siehe Übung), folglich gilt $u_1, \cdots, u_n = v_1, \cdots, v_n$ und somit ist $(u_1, v_1), \cdots, (u_n, v_n)$ eine Lösung von $I$. Die Instanz $I$ ist also lösbar wenn $I'$ Lösbar ist.
  \end{proof}


  \mysubsection{Lemma} Für Jedes alphabet $\Sigma$ mit $|\Sigma| \leq w$ gitl $MPCP_{\Sigma} \leq_m PCP_{\Sigma}$.
  \begin{proof}
    Sei $\Sigma$ ein Alphabet mit $|\Sigma| \geq 2$. Nach \hyperref[subsec:3.14]{Lemma 3.14}  genügt es ein Alphabet $\Gamma$ zu finden, so das $MPCP_{\Sigma} \leq_m PCP_{\Gamma}$ gilt. \\Wir suchen eine effektive Transformation , die jede instanz $(p, I)$ des modifizierten Postschen Korrespondenzproblems über $\Sigma$ in eine Instanz $I'$ des Postschen Korrespondenzproblems über einem geeignetem Alphabet$\Gamma$ transformiert, so dass $(p, I)$ genau dann lösbar ist, wenn $I'$ lösbar ist.
  \end{proof}
  \subsubsection*{Idee: }
  \begin{center}
    \begin{tikzpicture}
      \draw[blue] (0,0) rectangle (1,1) node[midway] {0};
      \draw[blue] (1,0) rectangle (2,1) node[midway] {1};
      \draw[blue] (2,0) rectangle (3,1) node[midway] {0};
      \draw[green] (3,0) rectangle (4,1) node[midway] {0};
      \draw[yellow] (4,0) rectangle (5,1) node[midway] {1};
      \draw[yellow] (5,0) rectangle (6,1) node[midway] {0};
      \draw[yellow] (6,0) rectangle (7,1) node[midway] {1};
      \draw[yellow] (7,0) rectangle (8,1) node[midway] {1};
      \draw[purple] (8,0) rectangle (9,1) node[midway] {1};
      \draw[purple] (9,0) rectangle (10,1) node[midway] {0};
      \draw[purple] (10,0) rectangle (11,1) node[midway] {1};
  
      \draw[blue] (0,1) rectangle (1,2) node[midway] {0};
      \draw[green] (1,1) rectangle (2,2) node[midway] {1};
      \draw[green] (2,1) rectangle (3,2) node[midway] {0};
      \draw[green] (3,1) rectangle (4,2) node[midway] {0};
      \draw[green] (4,1) rectangle (5,2) node[midway] {1};
      \draw[yellow] (5,1) rectangle (6,2) node[midway] {0};
      \draw[yellow] (6,1) rectangle (7,2) node[midway] {1};
      \draw[purple] (7,1) rectangle (8,2) node[midway] {1};
      \draw[purple] (8,1) rectangle (9,2) node[midway] {1};
      \draw[purple] (9,1) rectangle (10,2) node[midway] {0};
      \draw[purple] (10,1) rectangle (11,2) node[midway] {1};
    \end{tikzpicture}
  \end{center}
  ...
  Betrachte die Homomorphismus von Sprachen $\delta_{\rightarrow}$, $\delta_{\leftarrow} : \Sigma^* \rightarrow (\Sigma \cup {*})^*$ mit $\delta_{a} = a*$ und $\delta_{\leftarrow}(a) = *a$ $\forall a \in \Sigma$. Für jede Instanz $(p, I) = ((u_1, v_1), I)$ wie oben sei \[I' = \{(\delta_{\leftarrow}(u_1), *\delta_{\rightarrow}(v_1))\} \cup \{\delta_{leftarrow}(u), \delta_{rightarrow}(v):  (u, v) \in I\} \cup \{\delta_{\leftarrow}(u)*, \delta_{\rightarrow}(v): (u, v) \in I\}\] Die Funktion die geeignete Codes von Instanzen $(p, I)$ auf geeignete Codes der zugehörigen Instanzen $I'$ abbildet ist berechenbar. Gibt es eine Lösung $(u_1, v_1), \cdots, (u_n, u_n)$ von $(p,I)$ dann ist \[\delta_{\leftarrow}(u_1)\cdots \delta_{\leftarrow}(u_n)* = \delta_{\leftarrow} (u_1 \cdots u_n)*\]
  \[= \delta_{\leftarrow}(v_1 \cdots v_n)*\] \[= *\delta_{\rightarrow}(v_1 \cdots v_n)\] \[=*\delta_{\rightarrow}(v_1) \cdots \delta_{\rightarrow}(v_n)\] und folglich ist \[(\delta_{\leftarrow}(u_1), *\delta_{\rightarrow}(v_1)), (\delta_{\leftarrow}(u_2), \delta_{\rightarrow}(v_2)), \cdots, (\delta_{\leftarrow}(u_{n-1}), \delta_{\rightarrow}(v_{n-1})), (\delta_{\leftarrow}(u_{n}), \delta_{\rightarrow}(v_{n}))\] eine Lösung von $I'$. Es bleibt zu zeigen das $(p, I)$ lösbar ist, wenn $I'$ lösbar ist. Sei $\tau : (\Sigma \cup \{*\})^* \rightarrow \Sigma^*$ der Homomorphismus von Sprachen mit $\tau \vert_{\Sigma} = id_{\Sigma}$ und $\tau(*) = \lambda$. Für $(u', v') \in I'$ gilt $(\tau (u'), \tau(v')) \in I$. Sei $(u'_1, v'_1), \cdots, (u'_n, v'_n)$ eine Lösung von $I'$ und $(u'_i, v'_i) = (\tau(u'_i), \tau(v'_i))$ für $i \in [n]$. Es gilt \[\tau(u'_1) \cdots \tau(u'_n) = \tau(u'_1 \cdots u'_n) = \tau(v'_1 \cdots v'_n) = \tau(v'_1) \cdots \tau(v'_n)\] und somit ist $(u_1, v_1), \cdots, (u_n, u_v)$ eine Lösung von $I$ als Instanz des Postschen Korrespondenzproblems über $\Sigma$. Es genügt aber zu zeigen, dass $(u_1, v_1) = p $ gilt. Sei $p' = (\delta_{\leftarrow} (u_1), \not \tau (?wirklich nicht tau?) \delta_{\rightarrow}(v_1))$. Für $(u', v') \in I' / \{p'\}$ gilt $u'(1) \not = v'(1)$, da $(u'_1, v'_1), \cdots, (u'_n, v'_n)$ eine Lösung von $I'$ ist gilt also $(u'_1, v'_1) = p'$ und damit $(u_1, v_1) = (\tau(u'_1), \tau(v'_1)) = p$.

  \mysubsection{Lemma}
  Für jedes Alphabet $\Sigma$ mit $|\Sigma| \geq 2$ gilt $H_{init} \leq_m MPCP_{\Box, 0, 1, *, 6, +}$

  \begin{proof}
    Wir suchen eine effektive Transformation, die jede natürliche Zahl $e$ auf eine Instanz $(p_e,I_e)$ des modifizierten Portschen Korrespondenzproblems über $\{\Box, 0, 1, *, , +\}$ abbildet, so dass $\mathcal{M} _e(\lambda)\downarrow$ genau dann gilt, wenn $(p_e, I_e)$ lösbar ist. Sei $e \in \mathbb{N}_0$. Sei $Q$ Die Zustandsmenge und $\Delta$ die Übergangsrelation von $\mathcal{M}_e$. Es gelte also $\mathcal{M}_e = (Q, \Sigma, \Gamma, \Delta, s, F)$ für $\Sigma = \{0, 1\}$, $\Gamma = \{\Box, 0, 1\}$, $S = 0$, $F = \{0\}$ \\ Für eine Instanz $(p, I)$ des modifizierten Postschen Korrespondenzproblems über einem Alphabet bezeichnen wir eine Folge $p =  (u_1, v_1), \cdots, (u_n, v_n)$ für die $u_1 \cdots u_n \sqsubseteq v_1 \cdots v_n$ oder $v_1 \cdots v_n \sqsubset u_1 \cdots u_n$ gilt als \textbf{partielle Lösung} von $(p, I)$. Wir wollen $(p_e, I_e)$ so wählen, dass partielle Lösungen von $(p_e, I_e)$ partielle Rechungen von $\mathcal{M}_e$ entsprechen. Dabei codieren wie eine Konfiguration $(p, w, p) \in Q \times(\Gamma^*)*\mathbb{N}_0$ von $\mathcal{M}_e$ durch das Wort \[code (q, w, p) := \# w(1)\cdots w(p-q) * bin(q)* w(p) \cdots w(|w|)\#\] Im wesentlichen wollen wir erreichen, dass es genau dann für ein Wort w eine partielle lösung  $(u_1, v_1), \cdots, (u_n, v_n)$ von $(p_e, I_e)$ mit $w = v_1 \cdots v_n$ gibt, wenn $w$ Präfix der Konkation $code (C_1) \cdots code (C_n)$ der Code der Konfiguration einer partiellen Rechnung $C_1, \cdots, C_n$ von $\mathcal{M}_e$ bei Eingabe $\lambda$ ist. Eine solche partielle Lösung soll genau dann zu einer Lösung von $(p_e, I_e)$ vervollständigt werden können, wenn die durch $w$ beschriebene partielle Rechung mit einer Stoppkonfiguraion endet, alsp eine Rechung ist. Dann ist $(p_e, I_e)$ genau dann lösbar, wenn die Rechung von $\mathcal{M}_e$ zur Eingabe $\lambda$ endlich ist. \\\\ Für $q \in Q$ sei $\hat{q} : * bin(q)$\\ Als Startpaar sehen wir \[p_e = (0, 0 \# * \ *\Box\#)\] (die 0en sind nur dafür da da, damit "im?" komplment nicht leer ist.) Wir beschreiben nun die Konstruktion von $I_e$. Für jede Instruktion $(q, a, q', a', L) \in \Delta$ fügen wir folgende Paare ein \[(\# \hat{q}a, \# \hat{q}'\Box a'), (\Box \hat{q} a, \hat{q}'\Box a'), (0\hat{q}a, \hat{q}'0a')(1\hat{q}a, \hat{q}'1a')\] ein. Weiter, um unveränderte Infixe kopieren zu können fügen wir die Paare \[(\#, \#), (0, 0), (1,1), (\Box, \Box)\] ein. Nun brauchen wir noch Paare, die bei Terminierung der TM zu einer validen Instanz der $MPCP$ - Instanz führen.\\ $\leadsto \forall q \in Q \forall a \in \{\Box, 0, 1\}$ für die es keine Instruktion $(q, a, q', a', B)$ fürgen wir das Paar $(\hat{q}a, \dagger a)$ hinzu und auch \[(\dagger \Box, \dagger), (\dagger 0, \dagger), (\dagger 1, \dagger)\] \[(\Box \dagger, \dagger),(0 \dagger, \dagger),(1\dagger, \dagger)\] \[(\# \dagger \# 0, 0)\] Dies beschreibt die Konstruktion von $(p_e, I_e)$. Wir verzichten auf die einfache aber aufwändige Verfifikation, dass $\mathcal{M}_e$ genau dann bei Eingabe $\lambda$ terminiert, wenn $(p_e, I_e)$ lösbar ist.
  \end{proof}
  \mysubsection{Beispiel} Sei $e \in \mathbb{N}_0$ mit $\mathcal{M}_e = (\{0, 1\}, \{0, 1\}, \{\Box, 0, 1\}, \Delta, 0, \{0\})$, wobei $\Delta = \{(0, \Box, 1, 1, R), (1, \Box, 1, 1, L)\}$. Mitder Notation aus dem Beweis aus \hyperref[subsec:3.16]{Lemma 3.17} gilt dann [hier bild einfügen!]

  \mysubsection{Satz} Für jedes Alphabet $\Sigma$ mit $|\Sigma| \geq 2$ ist $PCP_{\Sigma}$ nicht entscheidbar. 
  \begin{proof}
    Mit \hyperref[subsec:3.16]{Lemma 3.16}, \hyperref[subsec:3.17]{Lemma 3.17} und \hyperref[subsec:3.18]{Lemma 3.18} folgt \[H_{init} \leq_m MPCP_{\Box, 0, 1, *, \#, \dagger} \leq_m PCP_{\Box, 0, 1, *, \#, \dagger} \leq_m PCP_{\Sigma}\] und damit $H_{init} \leq_m PCP_{\Sigma}$. Folglich ist $PCP_{\Sigma}$ nicht entscheidbar, da $H_{init}$ nicht entscheidbar ist.
  \end{proof}

  %alda
  \mysubsection{Fixpunktsatz, Rekusionstheorem und Satz von Rice} 
  
  Wir beschäftigen uns nun mit weiteren Konsequenzen der Standardaufzählung von TM. \[ \Phi_0, \Phi_1, \Phi_2, \cdots \] Standardaufzählung \[\Phi_{\Phi_e(0)}, \Phi_{\Phi_e(1)}, \Phi_{\Phi_e(2)}, \cdots\] andere Aufzählung $\Rightarrow$ \[\Phi_{f(0)}, \Phi_{f(1)}, \Phi_{f(2)}\]

  \mysubsubsection{Definition (Fixpunktsatz)} Ein \textbf{Fixpunkt} eine berechenbaren Funktion $f: \mathbb{N}_0 \to \mathbb{N}_0$ ist ein $e \in \mathbb{N}_0$ mit $\Phi_{f(e)} = \Phi_e$.

  \mysubsubsection{Satz (Fixpunktsatz, Hartley Rogers jr., 1967)} Alle berechenbaren Funktionen $f : \mathbb{N}_0 \to \mathbb{N}_0$ haben einen Fixpunkt.
  \begin{proof}
    $\forall e, x \in \mathbb{N}_0$ mit $\Phi_e(x) \uparrow$ sei $\Phi_{\Phi_e(x)} : \mathbb{N}_0 \leadsto \mathbb{N}_0$ die aprtiell berechenabre partielle Funktion mit $dom(\Phi_{\Phi_e(x)}) = \varnothing$. Sei $e_{\psi}$ ein Index von $\psi$. Gemäß $S_n^m$-Theorem (\hyperref[subsec:3.16]{Satz 3.7}) existiert eine berechenbare Funktion $s_1^1 : \mathbb{N}_0^2 \to \mathbb{N}_0$ mit $\Phi_{s_1^1(e_{\psi}, e)}(x) = \psi(e, x)$. $\forall e, x \in \mathbb{N}_0$.Sei $\eta : \mathbb{N}_0 \to \mathbb{N}_0$ die berechenbare Funktion mit $\eta (e) := s_1^1(e_{\psi}, e)$. Dann gilt \[\psi_{\eta(e)}(x) = \psi_{s_1^1(e_{\psi}, e)}(x) = \psi(e, x) = \Phi_{\Phi_e(e)}(x) \forall x \in \mathbb{N}_0\] also gilt \[\Phi_{\eta(e)} = \Phi_{\Phi_e (e)} (*)\] Sei $e_{f \circ h}$ ein Index der berechneten Funktion $f \circ h$ und $e_{fix} := \eta(e_{f\circ h})$. \[ \Phi_{f(e_{fix})} = \Phi_{f(\eta(e_{f \circ h}))} = \Phi_{\Phi{e_{f\circ h}} (e_{f\circ h})} \overset{(*)}{=} \Phi_{\eta(e_{f \circ h})} = \Phi_{e_\eta}\] Folglich ist $e_{fix}$ ein Fixpunkt von $f$. 
  \end{proof}
  Solche Fixpnkte wie oben sind "semandische" Fixpunkte und kein "syntaktischen" Fixpunkte. Aus dem Fixpunktsatz kann man leicht das Rekursionstheorem folgen, dsa es anschaulich erlaubt während der Konstruktion einer partiell berechebaren Funktion anzunehmen den Index der fertig definierten Funktion zu kennen. Auf Programmebene bedeutet das, das es möglich ist ein Programm so zu schreiben ,dass der fertige Quellcode im Programm zur Verfügung stelt (ohne diesen irgendwo, zum Beispiel vom Speicher des Quellcodes, einzulesen)

  \mysubsubsection{Satz (Rekursionstheorem, Stephen Cole Kleen, 1938)} 
  Für alle partielle Funktionen $\varphi : \mathbb{N}_0^2 \leadsto \mathbb{N}_0$ gibt es ein $e \in \mathbb{N}_0$ mit $\Phi_e(x) = \varphi(e, x) \quad \forall x \in \mathbb{N}_0$ 

  \begin{proof}
    Sei $e_{\varphi}$ ein index von $\varphi$. Gemäß $s_n^m$ - Theorem gibt es eine berechenbare Funktion $s_1^1 : \mathbb{N}_0^2 \to \mathbb{N}_0$ mit $\Phi_{s^1_1 (e_{\varphi}, e)} (x) = \varphi (e, x) \quad \forall x \in \mathbb{N}_0$
  \end{proof}

  Für Programme bedeutet dies die Existenz von sogenannten \textbf{Quines}. Dies sind Programme, die ihren eigenen Quellcode ausgeben (ohne diesen vom speicher zu lesen). Unsere Resultate zeigen, dass für hinreichend komplexe Programmiersprachen immer Quines existieren. Eine weitere Konsequent aus dem Fixpunktsatz ist die Einsicht, dass jede nicht triviale Programmiereigenschaft unentscheidbar ist.

  \mysubsubsection{Korollar}
  Es gibt ein $e \in \mathbb{N}_0$ mit $\Phi_e(x) = e \quad \forall x \in \mathbb{N}_0$.
  \begin{proof}
    Sei $\psi : \mathbb{N}_0^2 \leadsto \mathbb{N}_0$ die partiell berechebare Funktion mit $\psi(e, x) = e \quad \forall e, x \in \mathbb{N}_0$. Gemäß \hyperref[subsubsec:3.20.2]{Satz 3.20.2} gibt es nun ein $e \in \mathbb{N}_0$ mit $\Phi_e (x) = \psi (e,x) = e \quad \forall x \in \mathbb{N}_0$
  \end{proof}

  \mysubsubsection{Definition (Indexmenge)} 
  Eine Teilmenge $I \subseteq \mathbb{N}_0$ heißt Indexmenge, wenn $e \in I \Leftrightarrow e' \in I \quad \forall e, e' \in \mathbb{N}_0$ mit $\Phi_e = \Phi_e'$ gilt.

  \mysubsubsection{Satz (Satz von Rice, Henry Horden Rice, 1951)}
  Ist $I$ ein Indexmenge $\varnothing \not = I \not = \mathbb{N}_0$, so ist $I$ nicht entscheidbar.
  \begin{proof}
    Sei $e_0 \not \in I, e_1 \in I$ und sei $f: \mathbb{N}_0 \to \mathbb{N}_0$ die Funktion mit $f(e) = e_0 \quad \forall e \in I$ und $f(e) = e_1 \quad \forall e \in \mathbb{N}_0 / I$. (Ist $I$ entscheidbar dann ist $f$ offensichtlich berechenbar.) $\forall e \in \mathbb{N}_0$ gilt $f(e) \in I \Leftrightarrow e \not \in I$ und da $I$ eine Indexmenge ist ist somit $\Phi_{f(e)} \not = \Phi_e$. Die Funktion $f$ hat also keinen Fixpunkt. Wäre $I$ entscheidbar, so hätte $f$ aber einen Fixpunkt nach dem \hyperref[subsubsec:3.20.1]{Fixpunktsatz}.
  \end{proof}
\coversection{Automaten/finit1.png}{Automaten}{Imagine you're in a city with a limited number of locations (like a park, library, cafe, etc.). You can move from one place to another following specific paths (like roads). The paths you take depend on some rules, like the time of the day, or the type of ticket you have. The places you can reach with these rules represent different states in a finite automaton, and the rules themselves act like the transition function.\\ \hspace*{\fill} - ChatGPT}
Wir wollen Turingmahinen un stark einschränken. Wir betrahten ein Modell, das im wesentlichen ohne speicher zurechtkommt (=Tm ohne band \(\longrightarrow\)  brauchen es nur für die Eingabe). Der Ausgabemechanismus kennt nur Akzeptanz und Nichtakzeptanz. Als TM kann der wie folgt realisiert werden:
\begin{itemize}
    \item Es ist nur ein Band erlaubt.
    \item Bei jedem Rechenschritt bewegt sich der Kopf nach rechts. Ob und wie die Felder des Bandes dabei überschreiben werden spielt dann keine Rolle, denn der Kopf kann nie zurück bewegt werden; wir lehen aber fest, dass Symbole nicht überschrieben werden. Die Symbole die des Bandalphabet \(\Gamma\)  neben denen des Eingabealphabets \(\Sigma\)  und des \(\Box\) Symbols hat spielen keine Rolle. Wir legen hier \(\varGamma  = \Sigma \cup  \{\square \}\) fest.
    \item Beim Einlesen des ersten \(\square\)  Symbols muss die Rechnung der Machine enden. Wir soll die Rechnung nicht vor dem Einlesen des ersten \(\square\) Symbols enden. 
\end{itemize}
Dies bedeutet, dass wir TM \(M = (Q, \Sigma, \Sigma \cup \{\square \}, \Delta, s, F)\) die nur Instruktionen der Form (q, a, q', a, R) mit \(q \in Q\) und \(a \in \sigma\) hat. Dies sind nun stark eingeschränkte TM. Wir wählen eine äquivalente Form, die als endliche Automaten bezeichnet werden. 

\mysubsection{Definition (Endliche Automaten)}
    Ein endicher Automat, kurz EA, ist ein Tupel \(A = (Q, \Sigma, \Delta, s, F)\). Dabei ist 
    \begin{itemize}
        \item Q eine endliche Menge, der Zustandsmenge;
        \item \(\Sigma\) das Eingabealphabet;
        \item \(\Delta \subseteq Q \times \sigma \times Q\) die Übergangsrelation, eine relation, so dass es für alle \(q \in Q\) und \( a\in \sigma\) ein \(q' \in Q\) mit (q, a, q');
        \item \(s \in Q\) der Startzustand;
        \item \(F \subseteq Q\) die Menge der akzeptierten Zustände.
    \end{itemize}
    Der endliche Automat A ist ein deterministischer endlicher Automa,
    kurz DEA, wenn es \(\quad \forall (q,a) \in Q x \sigma\) genau ein q' gibt mit \((q,a,q') \in \Delta\). Im Sinne der obigen Betrachtung entspricht ein EA \(A = (Q, \Sigma, \Delta, s, F)\) der 1-TM \(M_{a} = (Q, \sigma, \sigma \cup \{\Box\}, \{(q, a, q', a, R) : (q, a, q') \in \Delta\}, s, F)\).
        \paragraph*{\(\leadsto\)}
            Band spielt keine wesentliche Rolle, Zustände mir gerade gelesenen Symbol bilden die Konfiurationen.

\mysubsection{Definition (Übergangsfunktion eines EA)}
    Sei \(A = (Q, \Sigma, \Delta, s, F)\) ein EA. Die \textbf{Übergangsfunktion} von A ist die Funktion \(\delta_{A} : Q \times \Sigma \rightarrow 2^{Q}\) \footnote{(=Potenzmenge von Q)} mit 
    \[
        \delta_{A}(q,a) = \{q'\in Q : (q, a, q')\in \Delta\} \quad \forall q \in Q, a \in \Sigma
    \] 
    \textbf{erweiterter Übergangsfunktion} von A ist die Funktion 
    \[
        \delta_{A}^{*} : Q \times \Sigma^{*} \rightarrow 2^{Q} \delta_{A}^{*}(q, \lambda) = \{q\}
    \] 
    und 
    \[
        \delta_{A}^{*}(q, aw) = \bigcup \limits_{q'\in \delta_{A}(q,a)} \delta_{A}^{*}(q', w) \quad \forall q\in Q a\in \Sigma
    \]
    und 
    \(w\in \Sigma^{*}\). Für \(Q_{0} \subseteq Q\) und \(w \in \Sigma^{*}\) schreiben wir \(\delta_{A}^{*} (Q_{0}, w)\) statt \(\bigcup \limits_{q \in Q_{0}} \delta_{A}^{*}(q,w)\).
    \\ Für einen EA \(A = (Q, \Sigma, \Delta, s, F)\), mit entsprechnder TM \(M_{A} = (Q, \Sigma, \Gamma, \Delta', s, F), q \in Q\) und \(w \in \Sigma^{*}\) ist \(\delta_{A}^{*}(s,w)\) die Menge der zustände, die sich als erst Komp.?? der letzten Konfig einer Rechnung von \(M_{A}\) zur Eingabe zu ergeben.

\mysubsection{Bemerkung (Eigenschaften von endlichen Automaten)*}
Sei \(A = (Q, \Sigma, \Delta, s, F)\) ein EA
\begin{itemize}
    \item [(i)] \( \forall q \in Q\) und \(a\in \Sigma\) gilt \(\delta_{A}^{*}(q,a) = \delta_{A}(q, a)\).
    \item [(ii)] Ist A ein DEA, \(q \in Q\), \(a \in \Sigma\) und \(w \in \Sigma^{*}\), und \(\lvert \delta_{A}^{*}(q,w) \rvert\) = 1.
    \item[(iii)] Seien \(u,v \in \Sigma^{*} \quad \forall q \in Q\) gilt \(\delta_{A}^{*}(q, uv) = \delta_{A}^{*}(\delta_{A}^{*}(Q_{0}, u), v)\).
\end{itemize}

\mysubsection{Definition (Übergangsfunktion eines DEA)}
Sei \(A = (Q, \Sigma,  \Delta, s, F)\) eine DEA. Auch die Funktion \(\delta_{det, A}: Q \times \Sigma \rightarrow Q\) mit \(\delta_{A}(q,a) = \{\delta_{det, A}(q, a)\} \quad \forall q \in Q\) und \(a \in \Sigma\) wird auch \textbf{Übergangsfunktion} von A gennant. Analoges gilt für \(\delta_{det, A}^{*}(Q_{0}, w)\) statt \(\bigcup \limits_{q \in Q_{0}}\{\delta_{det, A}^{*}(q, w)\}\).

\mysubsection{Bemerkung (Folgerungen für DEA)*}
    Ist \(A = (Q, \Sigma, \Delta, s, F)\) ein DEA, so gelten \hyperref[subsec:4.3]{Berkung 4.3 (i)} und \hyperref[subsec:4.3]{(iii)} auch wenn \(\delta_{A}\) durch \(\delta_{det, A}\) und \(\delta_{A}^{*}\) durch \(\delta_{det, A}^{*}\) ersetzt wird.

\mysubsection{Bemerkung(Eindeutigkeit endlicher Automaten)*}
    Sei Q eine endliche Menge, \(\Sigma\) ein Alphabet, \(s\in Q\), und \(F\subseteq Q\). 
    \begin{itemize}
        \item [(i)] \(\forall\) Funktionen \(\delta : Q \times \Sigma \rightarrow 2^{Q}\) gibt es genau einen EA \(A = (Q, \Sigma, \Delta, s, F)\) mit \(\delta_{A} = \delta\).
        \item [(ii)] \(\forall\) Funktionen \(\delta : Q \times \Sigma \rightarrow Q\) gibt es genau einen \(\delta_{det, A} = \delta\). 
    \end{itemize}

\mysubsection{Definition (akzeptierte Sprache Automat)}
    Sei \(A = (Q, \Sigma, \Delta, s, F)\) ein EA. Die Sprache \(L(A) := \{w \in \Sigma^{*} : \delta_{A}^{*}(s, w)\cap F \neq \varnothing \}\) ist die \textbf{akzeptierte Sprache} von A.

\mysubsection{Definition (regulär)} 
    Eine Sprache L heißt \textbf{regulär} wenn es einen EA A mit L(A) = L gibt. Wir schreiben \textbf{REG} für die Klasse der regulären Sprachen. Zu jedem Zeitpunkt während der Verbindung der Eingabe durch einen endlichen Automaten höngt der restliche Bearbeitung immer nur vom gegewärtigen Zustand und dem noch einzulesenden Teil der Eingabe ab, nicht aber wie bei TM im allgemeinen von vergangenen Bandmanipulation. Interpretiert man die Eingabe als von einer äußeren Quelle kommend, so ist der  Zustand des Automaten also allein durch seinen Zustand gegeben und der nächste Zustand hängt nur vom Zugeführten Symbol ab. Daher bietet sich eine Darstellung eines EA durch ein Übergangsdiagramm oder eine sogenannte Übergangstabelle an.

\newpage
\mysubsection{Beispiel (Endlicher Automat A)*}
    Sei \(A := (\{q_{0}, q_{1}\}, \{0, 1\}, \Delta, q_{0}, \{q_{1}\})\) mit \(\Delta = \{(q_{0})\}\). Das Übergangsdiagramm und die Übergangstabelle sehen wie folgt aus:

    \begin{center}
        \begin{tabular}{|c|c|c|}
            \hline
            \textbf{Zustand/Symbol} & \textbf{0} \footnotemark[1] & \textbf{1} \footnotemark[2] \\
            \hline
            \( \rightarrow\) \footnotemark[3] \(q_{0}\) & \(q_{0}\) & \(q_{1}\) \\
            \hline
            \(q_{1}\), * \footnotemark[4] & \(q_{1}\) \footnotemark[5] & \(q_{0}\) \\
            \hline
        \end{tabular}
    \end{center}
    \begin{enumerate}
        \item  die Elemente von \(\Sigma\)
        \item  die Elemente von \(\Sigma\)
        \item  Startzustand
        \item  Zustand \(\in F\)
        \item  \((q_1, 0, a_1) \in \Delta\) (wenn a in \(q_1\) ist und 0 einliest, geht A in \(q_1\) über)
    \end{enumerate}
    \createDiagram{Übergangstabelle}
    {

    }

    \begin{center}
        \begin{tikzpicture}[->,>=stealth,shorten >=1pt,auto,node distance=3cm,semithick]
          \tikzstyle{every state}=[fill=white,draw=black,text=black,minimum size=25pt]
        
          \node[state] (q1) {\(q_1\)};
          \node[state] (q2) [right of=q1] {\(q_2\)};
        
          \path (q1) edge [loop above] node {0} (q1)
                (q1) edge [bend left] node {1} (q2)
                (q2) edge [bend left] node {0} (q1)
                (q2) edge [loop above] node {1} (q2);
        \end{tikzpicture}
    \end{center}

    \createDiagram{Übergangsdiagramm}{}

    \paragraph*{Übergangsdiagramm:}
        Für jeden Zustand gibt es einen Kreis. Zustände in F bekommen einen Doppelkreis. Für (q, a, q') \(\in \Delta\) für einen Pfeil von dem Kreis von q zu dem Kreis von q' mit der Beschreibung a. Zusätzlich gibt es einen Pfeil (ohne Beschriftung) aus dem "Nichts" zus deom Kreis des Starzustandes. Ähnlich wie bei allgemeinen und normierten TM bleibt die Klasse der akzeptierten Sprachen glich wenn man nur deterministisch endliche Automaten zulässt. Um dies zu beweisen führen wir den Potentautomaten ein.
\newpage
\mysubsection{Definition (Potenzautomaten)}
    Sei \(A = (Q, \Sigma, \Delta, s, F)\) ein EA. der \textbf{Potenzautomat} von A ist der DEA \(P_{A} = (2^{Q}, \Sigma, \Delta', \{s\}, \{P \subseteq Q : P \cup F \neq \varnothing  \})\) mit 
    \[
        \delta_{det, P_{A}}(Q_{0}, a) = \bigcup\limits_{q \in Q_0} \delta_A (q, a) \quad \forall Q_0 \subseteq Q \quad \forall a \in \Sigma 
    \] 

    \textit
    {
        Anmerkung: Es gibt eine einfache möglichkeit einen nicht Deterministischen Automaten in einen Deterministischen umzuwanden. Das wird hier in Zukunft beschrieben. Siehe Tutoriumaufschrieb. (das wird hier in zukunft angefügt)
    }

\mysubsection{Satz(Charakterisierung regulärer Sprachen)*}
    Eine Sprache L ist genau dann regulär, wenn es eine DEA A mit L(A) = L gibt. 

    \begin{proof}
        Sei \(A = (Q, \Sigma, \Delta, s, F)\) ein EA mit Potenzautomat \(P_{A}\). Es genügt zu zeigen, dass \(L(A) = L(P_{A})\). Hierfür genügt es zu zeigen, dass:
        \[
            \delta_{det,P}^{*} = \delta_{A}^{*}(s, w) \quad \forall w \in \Sigma^{*} \circledast 
        \]
        Denn damit folgt
        \[
            w \in  L (P_{A}) \Leftrightarrow \delta_{P_{A}}^{*}(\{s\}, w) \cap \{P\subseteq Q : P\cap F \neq \varnothing \} \neq \varnothing 
        \] 
        \[
            \Leftrightarrow \delta_{det, P_{A}}^{*}(\{s\}, w) \cap F \neq \varnothing 
        \]
        \[
            \underset{\circledast}{\Leftrightarrow } \delta_{A}^{*}(\{s\}, w) \cap F \neq \varnothing 
        \]
        \[
            \Leftrightarrow w \in L(A)
        \]
        Wir zeigen \(\circledast \) mittels vollständiger Induktion über \(\lvert w \rvert\). Es gilt \(\delta_{det, P_{A}}^{*}(\{s\}, \lambda) = \delta_{A}^{*}(s, \lambda)\). Sei \(w \in \Sigma^{+}\) mit \(\delta_{det, A}^{*}(\{s\}, v) = \delta_{A}^{*}(s, v) \quad \forall v \in \Sigma^{\leq \lvert w \rvert - 1}\). Nun zeigen wir \(\circledast \) Sei va := w mit \(a \in \Sigma und \lvert v \rvert = \lvert w \rvert - 1\).
        \[
            \delta_{det, P_{A}}^{*} (\{s\}, w) \underset{\hyperref[subsec:4.5]{Bem 4.5}}{=} \delta_{det, P_{A}}^{*} (\delta_{det, P_{A}}^{*}(\{s\}, v), a)
        \]
        \[
            \underset{\text{Ind. hyp}}{=} \delta_{det, P_{A}}^{*}(\delta_{det, P_{A}}^{*}(\{s\}, v), a)
        \]
        \[
            = \bigcup \limits_{q \in \delta_{det, A}^{*}}\delta_{A}(q, a)
        \]
        \[
        = \delta_{A}^{*}(\delta_{A}^{*}(s, v), a)
        \]
        \[ 
            = \delta_{A}^{*}(s, va)
        \]
        \[ 
            = \delta_{A}^{*}(s, w)
        \]
    \end{proof}
\coversection{ReguläreSprachen/regul.png}{Reguläre Sprachen}
{
  A regular language can be thought of as a collection of sentences in a secret code. This secret code has a set of rules that determine which sentences are valid. You can think of it like a secret handshake, where only certain movements are allowed to be performed in a particular order.\\ \hspace*{\fill} - ChatGPT
}
\mysubsection{Definition (Äquivalenzrelation)} 
  Sei A eine Menge. Eine Äquivalenzrelation auf A ist eine Relation \(\sim \leq A^{2}\), so dass die folgende Eigenschaft erfüllt sind. (wie bei Relationen üblich verwenden wir Infixnotation)
  \begin{itemize}
    \item [(i)] \(a \sim a \quad \forall a \in A\)  (Reflexivität)
    \item [(ii)] \(a \sim b \Rightarrow  b \sim a \quad \forall a, b, c \in A\) (Symetrie)
    \item [(iii)] \(a \sim b, b \sim c \rightarrow a \sim c \quad \forall a, b, c \in A\)(Transitivität)
  \end{itemize}
  Die \textbf{Äquivalenzklasse} eines Elements \(a \in A\) bezüglich \(\sim\) ist die Menge \([a]_{~} := {a' \in A : a' ~a}\). Der \textbf{Index} von \(\sim\) ist die Kardinalität der Menge \(A_{/\sim} := {[a]_{\sim} : a \in A}\) falls diese endlich ist und \(\infty\) andernfalls.

\mysubsection{Definition (A-Äquivalenz)} 
  Sei \(A = (Q, \Sigma, \Delta, s, F)\) ein DEA mit erweiterter Übergangsfunktion \(\delta^{*}: Q \times \Sigma \rightarrow Q\). Die A-Äquivalenz ist die Relation \(\sim_A\) auf \(\Sigma^{*}\) mit 
\[
  u \sim_A v \Leftrightarrow \delta^*(s, u) = \delta^*(s,v) \quad \forall u, v \in \Sigma
\]

\mysubsection{Bemerkung} 
Sei A = \((Q, \Sigma, \Delta, s, F)\) eine DEA.
\begin{itemize}
  \item [(i)] Die A-Äquivalenz ist eine Äquivalenzrelation.
  \item [(ii)] Der Index von \(\sim_{A}\) ist höchstens \(|Q|\).
  \item [(iii)] Es gilt \(L(A) = \bigcup \limits_{w \in L(A)} [w]_{\sim A}\).
\end{itemize}

\mysubsection{Definition (Rechtskongruenz)} 
  Sei \(\Sigma\) ein Alpha. Eine Rechtskongruenz auf \(\Sigma^{*}\) ist eine Äquivalenzrelation \(\sim \subseteq (\Sigma^{*})^{2}\) mit \(u \sim v \Rightarrow uw \sim vw \quad \forall u, v, w \in \Sigma^{*}\).

\mysubsection{Proposition} 
  Sei \(A = (Q, \Sigma, \Delta, s, F)\) ein DEA. Die A-Äquivalenz \(\sim_{A}\) ist eine Rechtskonqruenz auf \(\Sigma^{*}\).
  \begin{proof}
    Seien \(u, v, w \in \Sigma^{*}\) mit \(u \sim_{A} v\). Dann gilt 
    \[
      \delta_{det, A}^{*}(s, uw) = \delta_{det,A}^{*}(\delta_{det,A}^{*}(s, u), w) = \delta_{det,A}^{*}(\delta_{det,A}^{*}(s,v), w)\] \[= \delta_{det,A}^{*}(s, vw).
    \] 
    (hier benutzen wir \hyperref[subsec:4.3]{Bemerkung 4.3} und \hyperref[subsec:4.5]{Bemerkung 4.5})    
  \end{proof} 
  Dann gilt \(uw\sim_{A}vw\). Zu jedem DEA A gibt es also eine dazugehärige Rechtskonguenz \(\sim\) auf \(\Sigma^{*}\) mit endlichem Index so dass L(A) die Vereinigung von Äquivalenzklasse von \(\sim_{A}\) ist. Tatsächlich gilt auch die Umkehrung: Ist L die Vereinigung von Äquivalenzklasse einer Rechtskongruenz \(\sim\) mit endlichem Index, so gibt es einen DEA A mit L(A) = L

\mysubsection{Definition(DEA-Konstruktion für Äquivalenzklassen)} 
  Sei \(\Sigma\) eine Alphabet und L Vereinigung von Äquivalenzklasse einer Rechtskongruenz \(\sim\) auf \(\Sigma^{*}\) mit endlichem Index. Es bezeichne
  \[
    A_{\sim , L} := (\Sigma^{*}_{/\sim}, \Sigma, \Delta, [\lambda]_{\sim}, {[w]_{\sim} : w \in L}
  \]
  den DEA mit \(\delta_{det, A_{\sim}, L}([w]_{\sim}, a) = [wa]_{\sim} \forall w \in \Sigma^{*}\) und \(a \in \Sigma\). Die Wohldefiniertheit von \(\delta_{det, A_{\sim}, L}\) ergibt sich daraus, dass \(\sim\) eine Rechtskongruenz ist. Um uns davon zu überzeugen, dass \(L(A_{\sim, L}) = L\) gilt betrachten wr zunächst die Arbeitsweise von \(A_{\sim, L}\).

\mysubsection{Lemma} 
  Sei \(\Sigma\) ein Alphabet, L Vereinigung von Äquivalenzklassem einer Rechtskongruenz \(\sim\) auf \(\Sigma^{*}\) mit endlichem Index und sei \(\delta^{*} : \Sigma^{*}_{/\sim} \times \Sigma^{*} \rightarrow \Sigma^{*}_{/\sim}\) die erweiterte Übergangsfunktion von \(A_{\sim, L}\). Dann gilt \(\delta^{*}([\lambda]_{\sim}, w) = [w]_{\sim} \forall w \in \Sigma^{*}\). 
  \begin{proof}
    Wir verwenden vollständige Induktion über |w|. Es gilt \(\delta^{*}([\lambda]_{\sim}, \lambda) = [\lambda_{\sim}]\). Sei nun w \(\in \Sigma^{+} \cdots\)  
  \end{proof}

\mysubsection{Satz} 
  Sei L die vereinigung von Äquivalenzklasse einer Rechtskongruenz \(\sim\) mit endlichem Index Es gibt \(L(A_{\sim, L}) = L\) 
  \begin{proof}
    Sei \(\Sigma\) das Alphabet, so dass \(\sim\) eine Rechtskongruenz auf \(\Sigma^{*}\) ist. Sei \(\delta^{*} : \Sigma^{*}_{/\sim} \times \Sigma^{*} \rightarrow \Sigma^{*}_{/\sim}\) die erweiterte Übergangsfunktion von \(A_{\sim, L}\) und sei w \(\in \Sigma^{*}\). Aus \hyperref[subsec:5.5]{Lemma 5.5} folgt. 
    \[w \in L(A_{\sim, L}) \Leftrightarrow \delta^{*}([\lambda]_{\sim}, w) \in {[v]_{\sim} : v \in L}\]
    \[\Leftrightarrow [w]_{\sim} \in {[v]_{\sim} : v\in L}\]
    \[\Leftrightarrow \exists v \in L : [w]_{\sim} = [v]_{\sim}\]
    \[\Leftrightarrow \exists v \in L : w \sim v\]
    \[\Leftrightarrow w \in L\]
  \end{proof}

\mysubsection{Korollar} 
  Eine Sprache L ist genau dann regulär, wenn sie die Verienigung von Äquivalenzklasse einer Rechtskongruenz mit endlichem Index ist. 
  \begin{proof}
    Folgt aus \hyperref[subsec:5.3]{Bemerkung 5.3}, \hyperref[subsec:5.5]{Proposition 5.5} und \hyperref[subsec:5.8]{Satz 5.8}
  \end{proof}
  Betrachten man nur deterministische endliche Automaten ohne unerreichbare Zustände, so entsprechen diese bis auf Unbenutzung von Zuständen sogar den Rechtskongruenz mit endlichem Index zusammen mit Vereinigung von Äquivalenzklassn dieser.

\mysubsection{Definition(erreichbar)} 
  Sei \(\Sigma\) ein Alphabet. Sei \(A = (Q, \Sigma, \Delta, s, F)\) ein EA mit erweiterter Übergangsfunktion \(\delta^{*}\). Ein zustand \(q\in Q\) heißt erreichbar in A wenn es ein Wort \(w \in \Sigma ^{*}\) mit \(q\in \delta^{*}(s, w)\) gilt.

\mysubsection{Definition(isomorph)} 
  Sei \(A_{i} = (Q_{i}, \Sigma, \Delta_{i}, s_{i}, F_{i})\) für \(i \in {1,2}\) ein EA mit Übergangsfunktion \(\delta_{i}\). Die endliche Automaten \(A_{1}\) und \(A_{2}\) sind \textbf{isomorph}, kurz \(A_{1}? \cong A_{2}\), wenn es eine Projektion \(f:Q_{1}\rightarrow Q_{2}\) gibt, sodass folgendes gilt:
  \begin{itemize}
    \item [(i)] \(f(s_{1}) = s_{2}\)
    \item [(ii)] \(\delta_{2}(f(q_{1}), a) = f(\delta_{1}(q_{1}), a)\)
    \item [(iii)] \(f(F_1) = F_2\)
  \end{itemize}

\mysubsection{Satz}
  \begin{itemize}
    \item [(i)] Ist A eine DEA ohne unereichbare Zustände, so gilt \(A \cong A_{\sim A, L(A)}\)
    \item [(ii)] Ist L die Vereinuíngung von Äquivalenzklasse einer Rechtskongruenz \(\sim\) mit endlichem Index, so gilt \((\sim, L) = (\sim_{A_{\sim, L}, L(A_{\sim, L})})\).
  \end{itemize}

  \begin{proof}
    \begin{itemize}
      \item [(i)] Sei \(A = (Q, \Sigma, \Delta, s, F)\) eine DEA mit erweiterte Übergangsfunktion \(\delta^* : Q \times \Sigma^* \to Q\) ohne unereichbare Zustände , \(\sim := \sim_A, A' := A_{\sim, L(A)}\) und sei \(\delta' : \Sigma^* / \sim \times \Sigma^* \to \Sigma/\sim\) die erweiterte Übergangsfunktion von A'. Sei \(f : Q \to \Sigma^* / \sim\) die Bijektive mit \(f(q) := \{ w \in \Sigma^* : \delta^*(s,w) = q\}\). Es gelte \(f(s) = [\lambda]_{\sim}\) und \(f(F) = \{[w]_{\sim} : w \in L(A)\}\). Es genügt somit zu zeigen , dass \(\delta'(f(q), a) = f(\delta(q,a)) \forall q \in Q, a \in \Sigma\). Sei \(q \in Q, a \in \Sigma^*\). Es genügt \(w \in \delta' (f(q), a) \Leftrightarrow \delta^*(s,w) = \delta^*(q, a)\) zu zeigen. Sei \(v \in \Sigma^*\) mit \(\delta^*(s,v) = q\). Nun gilt \(w \in \delta'(f(q), a) \leftrightarrow w \in \delta'([v]_{\sim}, a) \leftrightarrow w \sim va \leftrightarrow \delta^*(s,w) = \delta^*(s, va) \leftrightarrow \delta^*(s,w) = \delta^*(q,a)\)  
      \[
        füge hier bei dem letzetn pfeil noch "Bem 4.3 und 4.5" unter den pfeil hinzu"
      \]
      \item [(ii)] Sei \(\Sigma\) ein Alphabet, \(\sim\) eine Rechtskongruenz auf \(\Sigma^*\), L Vereinigung von Äquivalenzklassen von \(\sim\), \(A' := A_{\sim, L} = (\Sigma^*/\sim, \Sigma, A'\), \([\lambda]_{\sim}, \uparrow), \delta'^* : \Sigma^*/\sim \times \Sigma^* \to \Sigma^*/\sim\) die erweiterterte Übergangsfunktion von. Nach \hyperref[subsec:5.8]{Satz 5.8} gilt \(L = L(A')\), es genügt also \(\sim = \sim'\) zu zeigen. Sei \(u, v \in \Sigma^*\). Aus \hyperref[subsec:5.7]{Lemma 5.7}  folgt \(u \sim v \leftrightarrow 
      [u]_{\sim} = [v]_{\sim} \leftrightarrow \delta'(...)\dots\)
    \end{itemize}
  \end{proof}
  \hyperref[subsec:5.12]{Satz 5.12} Bedeutet insbesondere folgendes: Ist \(A_i, i \in \{1, 2\}\) ein DEA ohne unereichbare Zustände, so gilt \(A_1 \cong  A_2 \leftrightarrow (\sim_{A_1}, L(A_1)) = (\sim_{A_2}, L(A_2))\) und ist \(L_i\) für \(i \in \{1, 2\}\). Vereinigung von Äquivalenzklassen einer Rechtskongruenz \(\sim_i\) mit endlichem Index, so gilt \((\sim_1, L_1) = (\sim_2, L_2) \leftrightarrow A_{\sim_1,L_1} \cong A_{\sim_2,L_2}\). \\\\ Ist L eine reguläre Sprache, so gibt es verschiedene endliche Automaten (ohne unereichbare Zustände) mit L(A) = L. Äquivalenzklassen verschiedener Rechtskongruenz mit endlichem Index. Für alle solche Rechtskongruenz \(\sim\) und \(\forall u, v, w \Sigma^*\) mit \(u \sim v\) gilt aber 
  \[
    uw \in L \leftrightarrow \delta^*_{det, A} (s, uw) \in F \leftrightarrow \delta^*_{det, A} (s, vw) \in F \leftrightarrow vw \in L
  \] 
  Dies führt zum Begriff der L-Äquivalenz und zeigt, dass die Parition in die Äquivalenzklassen von \(\sim\) Vereinfacht der Parition in die Äquivalenzklasse der L-Äquivalenz ist.

\mysubsection{Definition (L-Äquivalenz)} 
  Sei L eine Sprache über einem Alphabet \(\Sigma\). Die \textbf{L-Äquivalenz} von L als Sprache ist die Relation \(\sim_L\) auf \(\Sigma^*\) mit 
  \[
    u \sim_L v \leftrightarrow (uw \in L \leftrightarrow vw \in L \quad \forall w \in \Sigma^*)
  \] 

\mysubsection{Bemerkung} 
  Sei L eine Sprache über \(\Sigma\). 
  \begin{itemize}
    \item [(i)] Die L-Äquivalenz ist eine Rechtskongruent.
    \item [(ii)] Es gilt \(L = \bigcup \limits_{w \in L}[w]_{\sim L}\).
  \end{itemize}

\mysubsection{Definition (Parition)} 
  Sei A eine Menge. Eine Parition von A ist eine Menge \(\mathcal{A} = {A_1, \cdots, A_n}\) paarweise disjunkt nichtleere Teilmengen von A mit \(\bigcup \limits_{i \in [n]} A_i = A\).

\mysubsection{Definition (Verefeinerung)} 
  Seien \(\mathcal{A}_1\) und \(\mathcal{A}_2\) Paritionen einer Menge A. Die Parition \(\mathcal{A}_2\) \textbf{Verefeinert} \(\mathcal{A}_1\) (heißt Verefeinerung von \(\mathcal{A}_1\)), wenn es \(\forall A_2 \in \mathcal{A}_2\) ein \(A_1 \in \mathcal{A}_1\), mit \(A_2 \subseteq A_1\) gibt.

\mysubsection{Bemerkung} 
  Seien \(\mathcal{A}_1\) und \(\mathcal{A}_2\) Paritionen einer Menge \(A_1\), so dass \(A_2\) die Parition \(\mathcal{A}_1\) verefeinert.
  \begin{itemize}
    \item [(i)] \(\forall A' \in \mathcal{A}_1\), gibt es eine Teilmengen \(\mathcal{A}'_2 \subseteq \mathcal{A}_2\), die eine Parition von A' ist.
    \item [(ii)] Es gilt \(|\mathcal{A}_1| \leq |\mathcal{A}_2|\)
    \item [(iii)] Gilt \(|\mathcal{A}_1| = |\mathcal{A}_2|\) dann ist \(\mathcal{A}_1 = \mathcal{A}_2\)
  \end{itemize}

\mysubsection{Proposition} 
  Sei \(\Sigma\) eine Alphabet und L eine Sprache über \(\Sigma\) und \(\sim\) eine Rechtskongruenz auf \(\Sigma^*\) mit \(L = \bigcup \limits_{w \in L} [w]_{\sim}\). Die Parition \(\Sigma^*/\sim\) ist eine Verefeinerung der partition \(\Sigma^*/ \sim L\).
  \begin{proof}
    Seien \(u, v \in \Sigma^*\) mit \(u \sim v\). Es genügt zu zeigen, dass \(u \sim_{L} v\). Sei \(w \in \Sigma^*\). Es genügt \(uw \in L \leftrightarrow vw \in L\) zu zeigen. Da \(\sim\) eine Rechtskongruenz ist dolgt \(uw \sim vw\). Ist \(u, w \in L\), so folgt aus \(L = \bigcup \limits_{w' \in L} [w']_\sim\) auch \(vw \in L\)(analog folgt auch auch \(vw \in L \Rightarrow uw \in L\)).\\ \(\Rightarrow u \sim_L v\).
  \end{proof}
  Das heißt \(\sim_L\) ist die größte Parition, die L darstellen kann.

\mysubsection{Definition (Minimalautomat)} 
  Sei L eine reguläre Sprache über \(\Sigma\). Der Minimalautomat von L als Sprache über \(\Sigma\) ist der DEA \(A_{\sim L, L}\).

\mysubsection{Satz}
  Sei L eine reguläre Sprache über \(\Sigma\) und sei \(M_(Q, \Sigma, \Delta, s ,F)\) der Minimalautomat von L. Dann gilt:
  \begin{itemize}
    \item [(i)] L(M) = L
    \item [(ii)] Ist A ein DEA mit Zustandsmenge \(Q_A\) und L(A) = L, so gilt \(|Q_A| \geq |Q|\).
    \item [(iii)] Ist A ein DEA mit |Q| Zuständen und L(A) = L, so gilt \(A \cong M\).
  \end{itemize}
  \begin{proof}
    \begin{itemize}
      \item [(i)] Folgt direkt aus \hyperref[subsec:5.8]{Satz 5.8} 
      \item [(ii)] Aus \hyperref[subsec:5.3]{Bemerkung 5.3 (ii)} folgt \(|Q_A| \geq |\Sigma^* / \sim_A|\). Nach \hyperref[subsec:5.18]{Proposition 5.18} ist \(\Sigma^* / \sim_A\) eine Verefeinerung von \(\Sigma^* / \sim_L\), nach \hyperref[subsec:5.17]{Bemerkung 5.17 (ii)} gilt also \(|\Sigma^* / \sim_A| \geq |\Sigma^* / \sim_L|\). Wegen \(|\Sigma^* / \sim_L| = |Q|\) folgt somit 
      \[
        |Q_A| \geq |\Sigma^* / \sim_A| \geq |\Sigma^* / \sim_L| = |Q|
      \]
      \item [(iii)] Sei A ein DEA mit |Q| Zuständen und L(A) = L. Hätte A unereichbare Zustände, so folgt \(|\Sigma^* / \sim_A| < |Q| = |\Sigma^* /\sim_L|\) im wiederspruch zu \hyperref[subsec:5.17]{Bemerkung 5.17 (ii)} und \hyperref[subsec:5.18]{Proposition 5.18} 
    \end{itemize}
    Nach \hyperref[subsec:5.12]{Satz 5.12} genügt es zu zeigen, dass \(\sim_A = \sim_M\) zu zeigen. Die Relationen \(\sim_M\) und \(\sim_A\) sind nach \hyperref[subsec:5.3]{Bemerkung 5.3} und \hyperref[subsec:5.5]{Proposition 5.5} Rechtskongruent mit endlichem Index und L ist Verfeinerung von Äquivalenzklassen davon. Damit sind \(\Sigma^* / \sim_A\) und \(\Sigma^* / \sim_M\) nach \hyperref[subsec:5.18]{Proposition 5.18} Verfeinerungen von \(\Sigma^* /\sim_L\). Somit sind die Indices von \(\sim_A\) und \(\sim_M\) mindestens so groß wie der index von \(\sim_L\). Weiter sind die Indices von \(\sim_A\) und \(\sim_M\) nach \hyperref[subsec:5.3]{Bemerkung 5.3 (ii)} aber auch höchstens so groß wie \(|Q| = |\Sigma^* / \sim_L|\). Die Indices von \(\sim_A\), \(\sim_M\) und \(\sim_L\) sind alle gleich groß. Da \(\Sigma^* / \sim_A\) und \(\Sigma^* / \sim_M\) Verefeinerungen von \(\Sigma^* /\sim_L\) sind, folgt mit \hyperref[subsec:5.17]{Bemerkung 5.17 (ii)} somit \(\Sigma^* /\sim_A = \Sigma^* /\sim_M = \Sigma^* /\sim_L und \sim_A = \sim_L = \sim_M\). 
  \end{proof}
  Unsere bisherigen Betrachtungen erlauben verschiedene Äquivalenten Charkterisierungen der Klasse der regulären Sprachen.

\mysubsection{Satz (Satz von Myhill und Nerode)} 
  Für eine Sprache L über einem Alphabet \(\Sigma\) sind die folgenden Aussagen äquivalent:
  \begin{itemize}
    \item [(i)] L ist regulär.
    \item [(ii)] Der Index von \(\sim_{L}\) ist endlich.
    \item [(iii)] L ist die Vereinigung von Äquivalenzklasse einer Rechtskongruenz mit endlichem Index.
  \end{itemize}

  \begin{proof}
    (i) \(\Leftrightarrow\) (iii) ist die Aussage von \hyperref[subsec:5.9]{Korollar 5.9}. Die Relation \(\sim_L\) ist nach \hyperref[subsec:5.14]{Bemerkung 5.14} eine Rechtskongruenzund es gilt \(L = \bigcup \limits_{w \in L} [w]_{\sim L}\). Somit folgt folgt (i) \(\Rightarrow\) (iii). Die Implikation (iii) \(\Rightarrow\) (ii) folgt aus \hyperref[subsec:5.17]{Bemerkung 5.17(ii)} und \hyperref[subsec:5.19]{Proposition 5.19}.
  \end{proof}

  Wie wollen nun ein Kriterium beschreiben das hilft nicht reguläre Sprachen zu erkennen.

\mysubsection{Satz (Pumping-Lemma)} 
  Sei \(\Sigma\) ein Alphabet. Für jede reguläre Sprache \(L \subseteq \Sigma^*\) gibt es eine Konstante \(k \in \mathbb{N}\), so dass folgendes gilt:\\ Ist \(z \in L\) mit \(|z| \geq k\), so gilt es Wörter um \(v \in \Sigma^*\) mit z = uvw, so dass folgendes gilt:
  \begin{itemize}
    \item [(i)]\(v \not = \lambda\)
    \item [(ii)]\(|uv| \leq k\)
    \item [(iii)]\(uv^iw \in L \forall i \in \mathbb{N}_0\) (?ist das w noch in dem wort oder ist es ausserhalt aber in l?)
  \end{itemize} 
  \begin{proof}
  %Sei L \subseteq \Sigma^* eine reguläre Sprache und ... ein DEA mit L(A) = L und erweiterter Übergangsfunktion \delta^*: Q \times \Sigma^* \to Q. Sei k := |Q|. Sei z \in L  mit |z| \geq k. (Falls kein solches z existiert ist nichts zu zeigen). Die Funktion f: \{0,\cdots, k\} \to Q, i \mapsto \delta^*(s, z(1) \cdots z(i)) ise keine Injektion, denn es gibt |\{0, \cdots, k\}| = k + 1 > |Q|. Sieen j_1, j_2 \in \{0, \cdots, k\} mit j_1, j_2 und f(j_1) = f(j_2). Sei u := z(1)\cdots z(j_1), v = z(j_1+1)\cdots z(j_2), w = z(j_2 + 1) \cdots z(|w|). Dann gilt z = uvw. Aus j_1 < j_2 folgt v \not = \lambda. Aus j_2 \leq k folgt |uv| \leq k. Es bleibt zu zeigen, dass uv^iw\in L \forall i \in \mathbb{N}_0 gilt. Dafür genügt es zu zeigen \delta^*(s, uv^i) = \delta^*(s, u) (*). Denn dann gilt mit Bemerkung 4.3(iii) und Bemerkung 4.5 \[\delta^*(s, uv^iw) = \delta^*(\delta^*(s, uv^i), w) = %\delta^*(\delat^*(s, u), w) = \delta^*(\delta^*(s, uv), w) = \delta^*(s, uvw)\] und damit uv^iw \in L. Wir zeigen (*) mittels vollständiger Induktion über i.\\
  %i = 0\\
  %Gelte nun \delta^*(s, uv^{i-1}) = \delta^*(s, u) für ein i \in \mathbb{N}. %Wieder folgt \[\delta^*(s, uw^i) = \delta^*(\delta^*(s, uv^{i-^})) ...\]
  \end{proof}

  \mysubsection{Beispiel} 
    Die Sprache \(L = \{0^n i^n : u \in \mathbb{N}_0\}\) ist nicht regulär. Dies lässt sich mit den Pumping-Lemma wie folgt zeigen.

    \begin{proof}
      Angenommen L wäre regulär.\\
      \(\Rightarrow \exists k \in \mathbb{N}_0 : \forall z \in L\) mit \(|z| \geq k gilt, \exists u, v, w \in \Sigma^*\) mit z = uvw und (i) \(v \not = \lambda\) (ii) \(|uv| \leq k\) (iii) \(uv^i w \in L \forall i \in \mathbb{N}_0\). \\ \(Sei z:= 0^k 1^k\). \\Aus (i) und (ii) folgt, dass \(v = 0^l\) für \(l > 0\) und damit folgt \(uw = 0^{k\cdot l} 1^k\)
    \end{proof}
  

\usetikzlibrary{trees}
\coversection{Automaten/finit1.png}{Formale Grammatiken}{\\ \hspace*{\fill} - ChatGPT}
\paragraph*{Idee:} Konstruktion aller Wörter einer Sprache.
\paragraph*{Beispiel:} 
\begin{itemize}
    \renewcommand{\labelitemi}{} % Remove bullet point

    \item \(L=\{0^n\}_{n\in\mathbb{N}_0}\)
    \item S Startsymbol
    \item \(S\to \lambda, \ S\to 0S\) Regel \( \quad (S, \lambda), (S, 0S)\)
    \item \(S\to 0S\to 00S\to 000S\to 000\)


\end{itemize}
\mysubsection{Definition (Grammatiken)}
    Eine Grammatik ist ien Tupel \(G=(N,T,P,S)\). Dabei ist 
    \begin{itemize}
        \item N das Alphabet der \textbf{Nichtterminalsymbole/Variablen} 
        \item T das Alphabet der \textbf{Terminalsymbole} mit \(N\cup T=\varnothing \)
        \item \(P\subseteq((N\cup T)^*\backslash T^*)\times (N\cup T)^*\) eine endliche Menge von \textbf{Regeln/Produktionen}, wobei wir für ein Paar \((u, v) \in P\) auch \( u \to v\) schreiben.
        \item \(S\in\mathbb{N}\) das \textbf{Startsymbol}
    \end{itemize}
    Eine \textbf{Satzform} von G ist ein Wort \(s\in(N\cup T)^*\) und eine \textbf{Terminalwort} von G ist ein Wort \(t\in T^*\).
\mysubsection{Definition (Ableitung)}
    Sei \(G=(N,T,P,S)\) eine Grammatik. Eine Satzform \(w'\) von G ist in einem Schritt aus einer Satzform w von G \textbf{ableitbar}, wenn es Satzformen u,v,x,y von G gibt, so dass \(w=xuy,u\to v\in P\) und \(w'=xvy\) gelten. Es bezeichne \(\to_G\) die Relation auf der Menge der Satzformen von G, sodass \(w\to_Gw'\) genau dann für Satzformen von G gilt, wenn w' aus w in einem Schritt ableitbar ist.\\
    Für Satzformen u,v von G ist eine \textbf{Ableitung} von v aus u eine Folge \(u=w_1,\cdots,w_n=v\) mit \(w_i\to_Gw_{i+1}\) \(\forall i\in[n-1]\) und eine Ableitung von v in G ist eine \textbf{Ableitung} von v aus S in G. Für \(n\in\mathbb{N}\) schreiben wir \(u\to^n_Gv\) wenn es eine Ableitung von $v$ aus $u$ der Länge $n$ gibt und wir schreiben \(u\to^*_Gv\) wenn eine Ableitung von v aus u in G existiert.
    \paragraph{Erklärung:}
        Stell dir vor, du möchtest ein Rezept zum Backen von Kuchen haben. Das Rezept besteht aus verschiedenen Schritten, wie zum Beispiel das Mischen der Zutaten und das Backen im Ofen. In ähnlicher Weise kann man sich eine Grammatik vorstellen, die Regeln für den Aufbau von Sätzen in einer Sprache festlegt. Nehmen wir an, du hast eine Grammatik namens G, die aus Buchstaben (N) und Wörtern (T) besteht. Diese Grammatik hat auch Regeln (P) und einen Startpunkt (S). Eine Satzform ist ein Satz, der in der Grammatik G gebildet werden kann. Jetzt stellen wir uns vor, du hast einen Satz, den wir als "w" bezeichnen. Du möchtest einen anderen Satz, "w'", in einem Schritt aus dem Satz "w" ableiten. Das bedeutet, dass es bestimmte Regeln gibt, die angewendet werden können, um von "w" zu "w'" zu gelangen. Man kann sich das wie einen Schritt in einem Rezept vorstellen, bei dem man eine Zutat durch eine andere ersetzt oder sie anders kombiniert. Eine Ableitung ist eine Folge von Schritten, bei der man von einem Satz "u" zu einem anderen Satz "v" gelangt. Jeder Schritt in der Ableitung wird durch eine Regel aus der Grammatik G dargestellt. Eine Ableitung von "v" in G ist eine Ableitung von "v" ausgehend vom Startpunkt "S" in G. %Um die Länge einer Ableitung anzugeben, verwenden wir die Zahl "n". Wenn wir schreiben "u ->^n_G v", bedeutet das, dass es eine Ableitung von "v" aus "u" gibt, die aus genau "n" Schritten besteht. Wenn wir schreiben "u ->^*_G v", bedeutet das, dass es eine Ableitung von "v" aus "u" in G gibt.
    
    
\mysubsection{Definition (Erzeugte Sprache)}
    Sei \(G=(N,T,P,S)\) eine Grammatik. Die von G erzeugte Sprache \(L(G)\) ist die Menge aller Wörter \(w\in T^*\) für die es eine Ableitung von w in G gibt.
\mysubsection{Lemma} 
    Sei \(G=(N,T,P,S)\) eine Grammatik und seien u,v,x,y Satzformen von G und seien \(n,m\in\mathbb{N}\) mit \(u\to_G^nv\) und \(w\to_u^nxuy\).\\
    Dann gilt \(w\to_u^{m+n-1}xvy\).
\begin{proof}
    Sei \(\alpha _1,\cdots,\alpha_n\) ein Ableitung von xuy aus w und \(\beta_1,\cdots,\beta_m\) eine Ableitung von v aus u. Dann ist\\ 
    \(\alpha_1,\cdots,\alpha_{n-1},x\beta_1y,\cdots,x\beta_my=xvy\)\\
    eine Ableitung von xvy aus w in G der Länge n+m-1.\par\bigskip
    Im folgenden beschäftigen wir uns mit dem Thema welche Sprache Grammatiken verschiedener Komplexitätsstufen erzeugen können.
\end{proof}

\mysubsection{Satz}
    Eine Sprache ist genau dann rekuriv aufzählbar, wenn sie von einer Grammatik erzeugt wird.
\paragraph*{Beweisidee} 
    Wird eine Sprache $L$ von einer Grammatik erzeugt, so ist $L$ die erkannte Sprache einer TM, die in geeigneter Weise Ableitungen von $G$ erzeugt, prüft ob diese Ableitung dem Wort der Eingabe entspricht und gegebenfalss akzeptiert. Wenn eine Ableitung der Eingabe gefunden ist.\par\bigskip
    Gegeben eine rekursiv aufzählbare Sprace $L$ und eine TM, die $L$ erkennt. So konstruieren wir ähnlich dem Postschen Korrespondenzproblems Regeln und Symobole, sodass wir die Arbeitsweise der TM modellieren können und entsprechend mit einem Terminalwort enden wenn dies von der TM erkannt wird.
\mysubsection{Definition (Rechtslinear)}
    Eine Grammatik \(G=(N,T,P,S)\) ist rechtslinear, wenn alle Regeln von der Form 
    \[X\in uy \text{ oder } X\to u\]
    mit \(X,y\in\mathbb{N}\) und \(u\in T^*\) sind.\par\bigskip 
    Hier ist es sinnvoll endliche Automaten zu betrachten bei denen es nicht \(\forall\) Zustände q und Eingabesymbole a ein Tripel \((q,a,q')\) in der Übergangsrelation geben muss. Solche Automaten sind zwangsläufig nicht deterministisch.
\mysubsection{Satz}
    Eine Sprache ist genu dann regulär, wenn sie von einer rechtslinearen Grammatik erzeugt wird.
\paragraph*{Beweisidee}
    Zunächst überzeugt man sich davon, dass eine Sprache L genau dann von einer rechtlinearen Grammatik erzeugt wird, wenn sie von einer Grammatik \(G=(N,T,P,S)\) erzeugt wird bei der alle Regeln von der Form \(X\to ay\) oder \(X\to \lambda\)\\
    mit \(x,y\in\mathbb{N}\) und \(a\in T\) sind. Eine Solche Grammatik wird als Grammatik in Simulationsform bezeichnet.\par\bigskip
    Die Sprache L die von einer rechtslinearen Grammatik (in Simulationsform) gebildet wird von dem EA 
    \[
        A=(N,T,\Delta,S,\{X\in N: X\to \lambda\in P\})
    \]
    mit 
    \[
        \Delta=\{(X,a,y)\in N\times T\times N:X\to ay\in P\}
    \]
    erkannt.\\
    Umgekehrt ist es einfach zu sehen, dass jede Reguläre Sprace von einer rechtslinearen Grammatik erzeugt wird. 

\mysubsection{Beispiel}
    Die Sprache \(\{0\}^*\) wird von der rechtlinearen Grammatik \[ G = (\{S\},\ \{0,\ 1\}^*,\ \{S \to 0S,\ S \to \lambda\},\ S)\] erzeugt von EA \[A = (\{S\},\ \{0,\ 1\}^*,\ \{(S,\ 0,\ S)\},\ S, \{S\})\] erkannt. Sowohl auf der Seite der Machinenmodelle als auch auf der Seite der Grammatiken gibt es weitere wichtige Sprachklassen, die sich ergeben wenn der Maschinenarbeitsweise oder Menge der zulässigen Regeln weniger stark eingeschränkt wird bei EA und rechtslinear Grammatiken.

\mysubsection{Definition (kontextfrei)} 
    Eine Grammatik \(G = (N,\ T,\ P,\ S)\) ist \textbf{kontextfrei}, wenn alle Regeln von G von der From \[X \to w\] mit \(X \in N\) und \(w \in (N \cup T)^*\) sind. Eine Sprache ist \textbf{kontextfrei}, wenn sie von einer Kontextfreien Grammatik erzeugt wird. Die Menge aller kontextfreien Sprachen bezeichnen wir mit \textbf{CF}.

\mysubsection{Definition (lexikographische Ordnung)}
    Sei \(\Sigma \) ein Alphabet. Die \textbf{lexikographische Ordnung} aud \(\Sigma^*\) ist die lineare Ordnung \(leq\) aud \(\Sigma^*\) für doe \(u \leq v\) für \(u, v \in \Sigma^*\) genau dann gilt wenn eine der folgenden Bedingungen erfüllt ist.
    \begin{itemize}
        \item [(L1)] \( u \sqsubseteq v\).
        \item [(L2)] Es gibt ein \(i \in [min \{|u|, |v|\}] \) mit \( u(j) = v (j) \quad \forall j \in [i - 1]\) und \(u (i) \not = v (i)\) und \(u (i) \leq v(i)\) für eine gebildete, lineare Ordung \(\leq\) auf \(\Sigma\)
    \end{itemize}
    Schreiben wir \(\leq\) für eine lineare Ordnung, so bedeutet \(u < v\) für Elemente \(u, v\), dass \(u \leq v\) und \(u \not = v\) gelten.
    \newpage
    Betrachte \(G = (\{S, X\}, \{0, 1\}, P, S)\) mit \[P = \{S \to XX,\ X \to 0X1,\ X \to \lambda\}\] 
    Betrachte Ableitung \(S,\ XX, X0X1,\ 0x10x1,\ 0X101,\ 00X1101,\ 001101\)
    \vspace{1cm}
    \createDiagram{}
    {
        \begin{tikzpicture}[level distance=2cm,
            level 1/.style={sibling distance=5cm},
            level 2/.style={sibling distance=2cm}]
            
            \node {S}
            child 
            {
                node {X}
                child{node {0}}
                child 
                {
                    node {X}
                    child {node {0}}
                    child 
                    {
                        node{X}
                        child{node {$\lambda$}}
                    }
                    child{node {1}}
                }
                child {node {1}}
              }
              child 
              {
                node {X}
                child 
                {
                    node {0}
                }
                child 
                {
                    node {X}
                    child 
                    {
                        node {$\lambda$}
                    }
                }
                child 
                {
                    node {1}
                }
            };
              
            \node[right=1cm] at (current bounding box.east) 
            {
                \begin{minipage}{6cm}
                    \begin{itemize}
                        \item oben Wurzel
                        \item die Ordnung der Knoten/ Ecken ist wichtig
                        \item Knoten/ Ecken haben Beschriftungen
                    \end{itemize}
                \end{minipage}
            };
        \end{tikzpicture}
    }
    \begin{center}
        
    \end{center}
    \createDiagram{}
    {
        \begin{tikzpicture}[level distance=2cm,
        level 1/.style={sibling distance=5cm},
        level 2/.style={sibling distance=2cm}]
        
            \node {$\lambda$}
            child 
            {
                node {1}
                child{node {11}
                    child{node {111}}
                    child{node {112}}
                }
                child{node {12}}
            }
            child 
            {
                node {2}
                child{node {21}}
            }
            child 
            {
                node {3}
                child{node {31}}
                child{node {32}}
                child{node {33}}
            };
        \end{tikzpicture}
    }
    \begin{itemize}
        \item Abbildung 12 repräsentiert die Sprache $T = \{\lambda, 1, 2, 3, 11, 12, 21, 31, 32, 33, 111, 112\}$.
        \item Der Wurzelknoten ist $\lambda$.
        \item Die Reihenfolge ist lexikographisch.
    \end{itemize}
    
\mysubsection{Definition (Baum)} 
    Eine endliche nicht leere Sprache T heißt \textbf{Baum}, wenn sie unter Präfixbildung abgeschlossen ist, also wenn für alle \(w \in T\) und \(p \sqsubseteq w\) auch \(p \in T\) gilt. Ein Wort \(w \in T\) heißt \textbf{Blatt} von T, wenn w bezüglich \(\sqsubseteq \) maximal in T istm also wenn \(w = w' \quad \forall w' \in T\) mit \(w \sqsubseteq w'\) gilt. Wörter \(w \in T\), die keine Blätter von T sind heißen \textbf{innere Ecken} von T.

\mysubsection{Lemma}
    Sei \(\Sigma\) eine Alphabet und \(\leq\) die lexikographische Ordnung auf \(\Sigma^*\). Seien \(u, w \in \Sigma^*\) mit \(u \not \sqsubseteq w\) und \(u \leq w\). Seien \(v_1<\cdots<v_n \in \Sigma^*\) mit \(v_1 \not = \lambda\). Dann gilt \(u < u v_1< \cdots < uv_n < w\)

    \createDiagram{Baum und Präfixbildung}
    {
        \begin{tikzpicture}[level distance=2cm,
            level 1/.style={sibling distance=5cm},
            level 2/.style={sibling distance=1cm},
            circleNode/.style={circle, draw, minimum size=1.5em}]

            \node [circle,draw]{\phantom{X}}
            child 
            {
                node [circle,draw](node1){\phantom{X}}
                child{node [circle,draw, fill=red](node2){\phantom{X}}}
                child{node [circle,draw, fill=red](node3){\phantom{X}}}
                child{node [circle,draw, fill=red](node4){\phantom{X}}}
                child{node [circle,draw, fill=red](node5){\phantom{X}}}
            }
            child 
            {
                node [circle,draw](node6){\phantom{X}}
            };

            \draw[->, draw=green, line width=2pt] (node1) -- (node2) -- (node3) -- (node4) -- (node5) -- (node6);
        \end{tikzpicture}
    }

    \begin{proof}
        Für \(v, v' \in \Sigma^*\) mit \(v \leq v'\) gilt offenba \(uv \leq uv'\)m es genügt also zu zeigen, dass \(uv \leq w \quad \forall v \in \Sigma^*\) gilt. Wegen \(u \not \sqsubseteq w\) existiert \(i \in [min \{|u|, |w|\}]\) mit \(u(j) = w(j) \quad \forall j \in [i-1]\) und \(u(i) < w(i)\). Für \(v \in \Sigma^*\) gilt dann \((uv)(j)=w(j) \quad \forall j \in [i - 1]\) und \((uv)(j)<w(i)\), also \(uv < w\).
    \end{proof}

\newpage

\mysubsection{Lemma}
    Sei T ein Baum, \(\leq\) die lexikographische Ordnung auf T, \(p \in T\) und \(Q:=\{w\in T: p\sqsubseteq w\}\). Es gelten \(p = min Q\) und \(Q = \{w \in T : min Q \leq w \leq max \leq max Q\}\). 

    \createDiagram{}
    {
        \begin{tikzpicture}[level distance=2cm,
            level 1/.style={sibling distance=2cm},
            level 2/.style={sibling distance=2cm}]
            
            \node [circle,draw]{\phantom{X}}
            child 
            {
                node [circle,draw]{\phantom{X}}
              }
              child 
              {
                node [circle,draw = green, line width=2pt]{\phantom{X}}
                child 
                {
                    node [circle,draw, fill = red]{\phantom{X}}
                    child 
                    {
                        node [circle,draw, fill = red]{\phantom{X}}
                    }
                    child 
                    {
                        node [circle,draw, fill = red]{\phantom{X}}
                    }
                }
                child 
                {
                    node [circle,draw, fill = red]{\phantom{X}}
                    child 
                    {
                        node [circle,draw, fill = red]{\phantom{X}}
                    }
                    child 
                    {
                        node [circle,draw = green, fill = red, line width=2pt]{\phantom{X}}
                    }
                }
            }
            child
            {
                node[circle,draw][circle,draw]{\phantom{X}}
            };
        \end{tikzpicture}
    }
    \begin{proof}
        Offensichtlich gilt \(p = min Q\). Für \(w \in Q\) gilt \(min Q \leq w \leq max Q\) nach Definition von min Q und max Q. Für \(w \not \in Q\) mit \(w \sqsubseteq w' \quad \forall w' \in Q\), also \(w < min Q\). Für \(w \not \in Q\) mit \(w \not \sqsubseteq p \quad \exists i \in [min {|p|, |w|}]\) mit \(w(i) \not = p(i)\), insbesondere existiert also ein minimales solches i und nach Definition von \(\leq\) folgt damit \(w \leq w' \quad \forall w'\in Q\) oder \(w' \leq w \quad w' \in Q\), insbesondere also wegen \( w \not \in Q\) somit \( w < min \ Q\) oder \(max \ Q < w\).
    \end{proof}

\mysubsection{Definition (Beschriftung)} 
    Eine \textbf{Beschrftung} eines Baumes T mit Elementen einer Menge X ist eine Funktion \(b: T \to X\).

\newpage

\mysubsection{Definition (Blattwort)}
    Sei \(\Sigma\) eine ALphabet, sei (T,b) ein Paar aus einem Baum T und einer Beschrifung \(b: T \to \Sigma \cup \{\lambda\}\) und sei \(t_1, \cdots, t_n\) dieFolge der Blätter von T in lexikographischer Reihenfolge. Das \textbf{Blattwort} von (T, b) ist \(b(t_1) \cdot b(t_n)\). Blattwort: 001101
    \createDiagram{Blattwort}
    {
        \begin{tikzpicture}[level distance=2cm,
            level 1/.style={sibling distance=5cm},
            level 2/.style={sibling distance=2cm}]
            
            \node (node1) {S}
            child 
            {
                node (node2) {X}
                child{node (node3) {0}}
                child 
                {
                    node (node4){X}
                    child {node (node5){0}}
                    child 
                    {
                        node (node6){X}
                        child{node (node7) {$\lambda$}}
                    }
                    child{node (node8) {1}}
                }
                child {node (node9) {1}}
              }
              child 
              {
                node (node10) {X}
                child 
                {
                    node (node11) {0}
                }
                child 
                {
                    node (node12){X}
                    child 
                    {
                        node (node13){$\lambda$}
                    }
                }
                child 
                {
                    node (node14){1}
                }
            };
            \draw[->, draw=blue, line width=2pt] (node5) -- (node7) -- (node8) -- (node11) -- (node13) -- (node14);

        \end{tikzpicture}
    }
\newpage
\mysubsection{Definition (Ableitungsbaum)} 
    In einer kontextfreien Grammatik \(G = (N,\ T,\ P,\ S)\) ist ein \textbf{Ableitungsbaum} für eine Satzform \(w \in (N \cup T)^*\) aus \(A \in N\) ein paar (T, b), bestehend aus einem Baum T und einer Beschriftung \(b: T \to N \cup T \cup \{\lambda\}\). Der baum T ist eine Teilmenge von \([d]^*\), wobei \(d:= max (\{|w|: x \to w \in P\} \cup \{1\})\). Die Wurzel des baumes ist mit dem Symbol A beschriftet und das Blattwort entspricht der Satzform w. Für jede innere Ecke t im Baum T existiert eine Regel \(X \to v \in P\), sodass Folgendes gilt.\footnote{Tex. Ed. sug.}
    \begin{itemize}
        \item [(i)] \(b(t) = X\)
        \item [(ii)] \(\{t' \in T : t \sqsubseteq t' \text{ und } |t'| = |t| + 1\} = \{ta : a \in max \{|v|, 1\}\}\)
        \item [(iii)] \(b(ta) = v(a) \quad \forall a \in [|v|]\)
        \item [(iv)] \(b(t_1) = \lambda \text{ falls } v = \lambda\)
    \end{itemize}
    Die \textbf{Blätter} von (T, b) sind die Blätter von T und die \textbf{inneren Ecken} von (T, b) sind die inneren Ecken von T.

\mysubsection{Beispiel}     
    Sei \(G = (\{S\},\ \{0,\ 1\},\ \{S \to 0S1,\ S \to \lambda\},\ S),\ T = \{\lambda,\ 1,\ 2,\ 3,\ 21,\ 22,\ 23,\ 221\}\) und sei \(b:T\to\{S,\ 0,\ 1,\ \lambda\}\) die Beschriftung mit \(b(\lambda) = b(2) = b(22) = S,\ b(1) = b(21) = 0\) und \(b(221) = \lambda\). Dann ist (T, b) ein Ableitungsbaum von 0011 aus S un G. Die \textbf{Darstellung} von T sieht wi folgt aus: \[HIERABBEINFÜGEN\]
    \createDiagram{Darstellung von T}{}

    \[Visualisierung 1 + 2 \]
    \createDiagram{Colorization}{}

    \mysubsection{Lemma}
    Sei \(G = (N,\ T,\ P,\ S)\) eine kontextfreie Grammatik und \(w_1, \cdots, w_l\) eine Ableitung von w aus S in G. Dann gibt es einen Ableitungsbaum (T, b) von w aus S in G mit \(l-1\) innere Ecken. 
    \begin{proof}
        Induktion über l.\\
        \textbf{L = 1} S = (T, b)\\
        Sei nun \(l \geq 2\) und die Aussage wahr \(\forall l' < l\). Sei (T', b') ein Ableitungsbaum von \(w_{l-1}\) aus S in G mit \(l-2\) innere Ecken (existiert per Induktions Hypotese). Sei \(w' := w_{l-1}\). Sei \(i \in [|w'|]\) und \(v \in (N \cup T)^*\) mit \(w'(i) \to v \in P\) und \(w = w'(1) \cdot w'(i-1) v w'(i+1)\cdots w'(|w'|)\). Sei t' das in lexikographischer Reihenfolge i-te Blatt von T' und sei (T, b) der Ableitungsbaum mit \(T = T' \cup \{t'a : a \in [|v|]\},\ b(\tilde{w}) = b'(\tilde{w}) \quad \forall \tilde{w} \in T'\) und \(b(t'a) = v(a) \quad \forall a \in [|v|]\). Aus Lemma 6.12 folgt dann, dass w das Blattwort von (T, b) ist. Außerdem ist die Anzahl der inneren Ecken von T durch \(l - 2 + 1 = l - 1\) gegeben.
    \end{proof}

\mysubsection{Lemma} 
    Sei \(G = (N,\ T,\ P,\ S)\) eine kontextfreie Grammatik und (T, b) ein Ableitungsbaum von w aus S in G mit \(l - 1\) inneren Ecken. Dann gibt es eine Ableitung \(leq = w_1, \cdots\) 
    \[ohne beweis\]


\newpage
\mysubsection{Satz (Pumping-Lemma für kontextfreie Sprachen)}
    Sei \(\Sigma\) ein Alphabet. Für jede kontextfreie Sprache \(L \subseteq \Sigma^*\) gibt es eine Konstante \(\in \mathbb{N}\), so dass folgendes gilt:\\ Ist \(z \in L\) mit \(|z| \geq k\), so gibt es Wörter \(u, v, w, x, y \in \Sigma^*\) mit \(z = uvwxy\), so dass folgendes gilt.
    \begin{itemize}
        \item [(i)] \(vx \not = \lambda\)
        \item [(ii)]\(|vwx| \leq k\)
        \item [(iii)] \(uv'wx'y\in L \quad \forall i \in \mathbb{N}_0\)
    \end{itemize}
    \begin{proof}
            Sei \(G = (N,\ T,\ P,\ S)\) eine kontextfreie Grammatik mit \(L(G) = L\), sei \(d:=max\ (\{|v|: X \to v \in P\} \cup \{1\})\), sei \(k := |[d]|^{\geq |N| + 2}\). Sei \(z \in L\) mit \(|z|\leq k\). Sei (T, b) eine Ableitungsbaum von z aus S in G mit minimaler Anzahl innerer Ecken. Sei \(r \in T \) ein Wort maximaler Länge. Da z Blattwort von T ist folgt insbesondere \(|T| \leq |z| \leq k > |1[d]^{|N|+1}|\), also \(T \not \subseteq [d]^{\geq |\mathbb{N}|+1}\) und somit gilt \(|r|\geq |N|+2\). Seien \(s, t\in[d^*]\) mit \(r = st\) und \(|t| = |N|+2\). Nach schubfachprinzip existieren Präfixe \(t_1, t_2\) von t mit \(b(st_1) = b(st_2)\in N\) und \(|t_1|<|t_2|\). Dann ist auch \(t_1\) Präfix 
    \end{proof}
    \createDiagram{Ableitungsbäume}
    {
        \begin{tikzpicture}

            \coordinate (G) at (-3,0);
            \coordinate (H) at (5,0);
            \coordinate (I) at (1,{4*sqrt(3)});
            \draw[fill=white] (G) -- (H) -- (I) -- cycle;
            \node at (G) [below] {};

            \coordinate (D) at (-1,0);
            \coordinate (E) at (3,0);
            \coordinate (F) at (1,{2*sqrt(3)});
            \draw[fill=blue] (D) -- (E) -- (F) -- cycle;
            \node at (D) [below] {};

            \coordinate (A) at (0,0);
            \coordinate (B) at (2,0);
            \coordinate (C) at (1,{sqrt(3)});
            \draw[fill=green] (A) -- (B) -- (C) -- cycle;
            \node at (A) [below] {};

            \coordinate (M) at (1,{3*sqrt(3)});
            \draw[red, line width=2pt] (I) -- (M);

            \draw[blue, line width=2pt] (M) -- (F);
            \draw[green, line width=2pt] (F) -- (C);


            \coordinate (J) at (1,{4*sqrt(3)});
            \draw[fill=white] (J) circle [radius=0.3];    
            \node at (J) {\Large S};

            \coordinate (K) at (1,{2*sqrt(3)});
            \draw[fill=white] (K) circle [radius=0.3];
            \node at (K) [text=blue]{\Large A};

            \coordinate (L) at (1,{sqrt(3)});
            \draw[fill=white] (L) circle [radius=0.3];
            \node at (L) [text=blue]{\Large A};


            \coordinate (N) at (0.8,{3.5*sqrt(3)});
            \node at (N) [text=red]{s};

            \coordinate (O) at (0.8,{2.5*sqrt(3)});
            \node at (O) [text=blue]{$t_1$};

            \coordinate (P) at (0.8,{1.5*sqrt(3)});
            \node at (P) [text=green]{$t_+$};


            \coordinate (Q) at (-2,-1);
            \node at (Q) {u};
            \coordinate (R) at (-0.5,-1);
            \node at (R) {v};
            \coordinate (S) at (1,-1);
            \node at (S) [text=green]{w};
            \coordinate (T) at (2.5,-1);
            \node at (T) {x};
            \coordinate (U) at (4,-1);
            \node at (U) {y};

            \coordinate (V) at (-1, -1.5);
            \coordinate (W) at (3, -1.5);
            \draw[blue, line width=2pt] (V) -- (W);

            \coordinate (Y) at (0, -0.5);
            \coordinate (Z) at (2, -0.5);
            \draw[green, line width=2pt] (Y) -- (Z);


            \coordinate (X) at (1, -2);
            \node at (X) [text=blue] {w'};

            

        \end{tikzpicture} 
    }
    
    Wir betrachten vier Ableitungsbäume.
    \begin{itemize}
        \item [(1)] Sei \((T_1, b_1)\) des Ableitungsbaum mit \(T_1 = \{q \in [d]^*:s,t,q \in T\}\). Sei w' das Blattwort von \((t_1, b_1)\).\((T_1, b_1)\) ist ein Ableitungsbaum von w' aus A in G.
        \item [(2)] Sei \(T_2, b_2\) der Ableitungsbaum mit \(T_2 = \{q \in [d]^* : st_2q\in T\}\) und \(b_2(q) = b(st_2q \quad \forall q\in T_2)\). Sei \textbf{w} das Blattwort von \(T_2, b_2\). Der Ableitungsbaum \(T_2, b_2\) ist ein Ableitungsbaum von w aus A in G.
        \item [(3)] Sei \((U_0, c_0)\) der Ableitungsbaum 
        
        mit \(U_0 = T/ ( \{st_1\} (T\{ \lambda
        \}))\) und \(c_0(q) = b(q) \quad \forall q \in U_0\) und sei \(z\_\) das Blattwort von \(U_0, c_0\). Aus Lemma 6.13 folgt, dass es \(u, y \in (N \cup T)^*\) gibt, sodass \(z = uw'y\) und \(z\_ = uAy\).
        \item[(4)] Sei \((u_1, c_1)\) der Ableitungsbaum mit \(U_1 = T_1/ (\{t_1\}(T_2/\{\lambda\}))\) und \(c_1(q) = b_1(q) \quad \forall q \in U_1\) und sei \(w'\_\) aus A in G. Aus Lemma 6.13 folgt, dass es \(v,x \in (N\cup T)^*\) gibt, so dass \(w' = vwx und w'\_ = vAx\) gelte.
    \end{itemize}
    Insgeasmmt existiert somt Wörter u, v, w, x, y mit \(z = uvwxy = uw'y\) und 
    \begin{center}
        \begin{tabular}{lll}
            & & bezeugt durch \\
            & \(S \to_G^* uAy\) & \((U_0, c_0)\) \\
            \(\circledast\) & \(A \to_G^* vAx\) & \((U_1, c_1)\) \\
            & \(A \to_G^* w\) & \((T_2, b_2)\) \\
       \end{tabular}       
    \end{center}
    Da T minimal ist gilt \(vx \not = \lambda\). Da \(|t| = |N| + 2\), ist die Anzahl der Ecken in \(T_2, b_2\) höchstens \(k \Rightarrow |vwx| \leq k\). Aus \(\circledast\) folgt \(uv^iwx^iy \in L \quad \forall i \in \mathbb{N}_0\)

\mysubsection{Beispiel}
    \(L = \{0^n 1^n 0^n : n\in \mathbb{N}_0\} \not \in \textbf{CF}\) lässt sich mit dem Pumping-Lemma für kontextfreie Sprachen wie folgt zeigen:
    \begin{proof}
        Angenommen \(L \in \)\textbf{CF}. Sei \(k \in \mathbb{N}\) für L wie im Satz 6.20 gewählt. Sei \(z:=0^k1^k0^k\). Seien u, v, w, x, y \(\in \{0, 1\}^*\) gemäß der Wahl von k Wörter mit uvwxy = z, \(vx \not = \lambda\), \(|vwx| \leq k\) und \(uv^iwx^iy \in L \quad \forall i \in \mathbb{N}_0\). Aus \(uvwxy = z\) und \(|vwx| \leq k\) folgt, dass \(r, s \leq k\) mit \(u = 0^r \wedge y = 1^s0^k\) oder \(u = 0^k 1^r \wedge y = 0^s\) existieren. Wegen \(vx \not = \Lambda\) existieren also \(r, s \leq k\) mit \(r \leq k - 1\) oder \(s \leq k -1\) so dass \(uwy = 0^r1^s0^k\) oder \(uwy = 0^k1^r0^s\) gilt. In jedem Fall gilt also \(uwy \not \in L\) im Wiederspruch dazu, dass \(uv^iwx^iy \in L \quad \forall i \in \mathbb{N}_0\) gilt.
    \end{proof}

\mysubsection{Beschreibung (Kellerautomat)}
Die Konstruktion von \textbf{Kellerautomaten} folgt den folgenden Ideen:
\begin{itemize}
    \item Es handelt sich im wesentlichen um endliche Automaten, die nicht wie TM zusätzlich ein Band zur freien Verfügung haben, sonder stattdeseen einer Stack/ Keller.
    \item Zustandsübergänge hängen nicht nur vom Zusatand und eingelesebeb Symbol ab, sondern auch vom obersten kellersymbol.
    \item Bei jedem Zustandsübergang wird das oberste Kellersymbol entfehrnt und es werden nacheinander Symbole eines Wortes über dem kelleralphabet oben auf den Keller gelegt.
\end{itemize}
Bei geeigneter Formalisierung von Kellerautomaten lösst sich folgendes zeigen.

\mysubsection{Satz}
    Eine Sprache wird genau dann von Kellerautomaten erkannt, wenn sie kontextfrei ist.

\mysubsection{Beispiel}
    Die Sprache \(\{0^n1^n : n\in \mathbb{N}_0\}\) wird von der kontextfreien Grammatik 
    \[
        G = (\{S\}, \{0, 1\}, \{S \to 0S1, S \to \lambda\}, S)
    \] 
    erzeugt. Eine Kellerautomat könnte prüfen ob die Eingabe von der Form \(0^m 1^n\) ist und den Keller benutzen um zu erkennen ob \(\#_0(w) = \#_1(w)\) für eine Eingabe w ist.
    
\mysubsection{Definition (Kontextsensitiv)}
    Eine Grammatik \(G = (N, T, P, S)\) heißt \textbf{kontextsensitiv}, wenn alle Regeln von G von der Form 
    \[
        uXv \to uwv    
    \]
    mit \(x \in N, u,v \in (N \cup T)^+\). Eine Sprache L heißt \textbf{kontextsensitiv}, wenn \(L/{\lambda}\) von einer kontextfreien Grammatik erzeugt wird. Die Menge aller kontextsensitiven Sprachen bezeichnen wir mit \textbf{CS}.
    
\mysubsection{Definition (nicht verkürzend)}
    Eine Grammatik \(G = (N, T, P, S)\) ist \textbf{nichtverkürzend}, wenn alle Regeln von G von der Form \(u \to v\) mit \(u,v \in (N \cup T)^* \) mit \(|u| \leq |v|\) sind.

\mysubsection{Satz}
    Eine Sprache L ist genau dann kontextsensitiv, wenn \(L / \{\lambda\}\) von einer nichtverkürzten grammatik erzeugt wird.
    \begin{proof}
        \textbf{Beweisidee: } Kontextsensitive grammatiken sind immer nichtverkürzend. Es genügt also nichtverkürzende Grammatiken in kontextsensitive Grammatiken umzuwanden. Jede nichtverkürzende Grammatik kan in eine nichtverkürzende Grammatik umgewandeln, die dieselbe Sprache erzeugt, sodass alle Regeln von der From 
        \[
            X_1 \cdots X_m \to Y_1 \cdots Y_n \text{oder} X \to a   
        \]
        mit \(m\leq, X, X,\cdots,X_m, Y_1,\cdots, Y_n \in N\) und \(a \in T\) sind. (Das man Nichtterminalsymbol A einführt für jedes \(a \in T\) und in allen Regeln a durch A ersetzt und die Regeln \(A \to a \quad \forall a \in T hinzufügt\).) Eine solche grammatik heißt \textbf{separiert} und kann in eine kontextsensitive Grammatik umgewandet werden. Dazu betrachten wir jede regel \(p = X_1 \cdots X_m \to Y_1\cdots Y_n\) mit \(X_1,\cdots, X_m, Y_1, \cdots, Y_n \in N\). Wir ersetzen p durch die Regeln 
        \[
            X_1 \to Z_1^P
        \]
        \[
            Z_1^p X_{i+1} \to Z_i^p Z_{i+1}^p \text{ mit } i \in [m-2]    
        \]
        \[
              Z_{m-1}^p X_{i+1} \to Z_i^p \tilde{Z}^p_{m}>_{m+1}\cdots Y_n
        \]
        \[
            Z_i^p \tilde{Z}_{i+1}^p \to \tilde{Z}_i^p \tilde{Z}_i+1^p \text{ mit } i \in [m-1]   
        \]
        \[
            \tilde{Z}_i^p \tilde{Z}_{i+1}^p \to \tilde{Z}_i^p Y_{i+1} \textbf{ mit } i \in [m-i]
        \]
        \[
            \tilde{Z}_1^p \to Y_1    
        \]
        Dabei sind \(Z_1^p,\cdots, Z_{m-1}^p, \tilde{Z}^p_1 \cdots \tilde{Z}_m^p\) neue Nichtterminalsymbole.
    \end{proof} 

\mysubsection{Beschreibung (linear beschränkter Automat)}
    Ein linear beschränkter Automat, kurz LBA, ist ein l-TM M, die \(\forall\) Wörter w über dem Eingabealphabet in allen Rechnungen von M zur Eingabe w nur die durch den Eingabemechanismus mit den Symbolen von w beschränkten Feldern und die an diese angrenzenden Felder besucht.

\mysubsection{Satz}
    Eine Sprache ist genau dann kontextsensitiv, wenn sie von einem LBA erkannt wird. \[ohne beweis\]

\mysubsection{Beispiel}
    Die Sprache \(\{0^n1^n0^n : n \in \mathbb{N}\}\) wird von einer nicht verkürzenden Grammatik \(G = (N, \{0,\ 1\}, P,\ S)\) erzeugt mit \(N =\{S,\ A,\ B,\ C,\ \tilde{A},\ \tilde{B},\ \tilde{C}\}\) und 
    \[
        P = S \to \tilde{A} B \tilde{C}, S \to 010,\ \tilde{C} \to CAB \tilde{C},\ BA \to AB,\ CA \to AC,\ CB\to BC,\ \tilde{A}A \to 0\tilde{A},\ \tilde{A}B \to 0\tilde{B},\ \tilde{B}B
    \]

    Diese Sprachklassen bilden die \textbf{Chomsky-Hirarchie}

\mysubsection{Definition}
    Die Klassen der Chomsky-Hirarchie sind \textbf{CH}(0) := \textbf{RE}, \textbf{CH}(1) := \textbf{CS} \textbf{CH}(2) := \textbf{CF}, \textbf{CH}(3) := \textbf{REG}
    
\mysubsection{Satz}
    Es gilt \[\textbf{REG} \not \subseteq \textbf{CF} \not \subseteq \textbf{CS} \not \subseteq\textbf{REC}\not \subseteq \textbf{RE}\]
    \begin{proof}
        \textbf{Beweisidee: } Für die Echtheit der Inklusionen siehe Beispiel 6.33. Für die Gültigkeit der Inklusionen bleiben \textbf{CF} \(\leq\) \textbf{CS} und \textbf{CD} \(\leq\) \textbf{REC} zu zeigen. Ist \(G = (N,\ T,\ P,\ S)\) eine kontextfreie Grammatik, so erhält man die Menge P' der Regeln einer kontextsensitiven Grammatik \(G' = (N,\ T,\ P',\ S)\) mit \(L(G) = L(G)/ \{\lambda\}\) wenn ausgehend von P in zewi Schnitten wie folgt verfahren wird. Zunächst werden alle Regeln \(X \to u \in P\) die Regeln \(X \to u'\) hinzugefügt, bei denen u' aus u entsteht indem ein oder mehrere Symbole aus u entfernt werden für die es eine Ableitung von \(\lambda aus Y\) gibt. Dann werden alle Regeln der Form \(X \to \lambda\) entfehrnt. \(\Rightarrow\) \textbf{CF} \(\subseteq\) \textbf{CS}\\ Jede kontextfreie Sprache ist entscheidbar, denn zu einem geeignet gesetzten LBA M kann wegen der endlichen Anzahl von Konfigurationen von M effektiv von einerTM entschieden ob dieser eine gegebene Eingabe akzeptiert oder nicht.
    \end{proof}

\mysubsection{Beispiel}
\begin{itemize}
    \item [(i)] Die Sprache \(L_1 = \{1^n : n\in \mathbb{N}\}\) ist offensichtlich regulär.
    \item [(ii)] Die Sprache \(L_2 = \{0^n1^n: n \in \mathbb{N}\}\) ist kontextfrei (Beispiel 6.24), aber nicht regulär -> Pumping Lemma.
    \item [(iii)] Die Sprache \(L_3 = \{0^n 1^n 0^n: n \in \mathbb
    N\}\) ist kontextsensitiv (Beispiel 6.30), aber nicht kontextfrei (Beispiel 6.21, Pubping Lemma für kontextfreie Sprachen)
    \item [(iv)] Sei \(M_0,\ M_1,\ \cdots\) eine geeignete Aufzählung aller LBAs mit Zustandsmenge \(\{0,\ \cdots,\ n\}\) für ein \(n \in \mathbb{N}\), Eingabealphabet \(\{1\}^*\) und Bandalphabet \( \{0, \cdots, m, \Box \} \) für ein \(m \in \mathbb{N}\). \\Die sprache \(L_4 = \{1^e \not \in L(M_e)\}\) ist entscheidbar aber nicht kontextsensitiv.
    \item [(v)] Die Sprache \(\{1^e:e\in H_{diag}\}\) ist rekusiv aufzählabr, aber nicht entscheidbar.
    \item [(vi)] Die Sprache \(\{1^e : e \not \in H_{diag}\}\) ist nicht rekursiv aufzählbar.
\end{itemize}
\coversection{Automaten/finit1.png}{Zeitkomplexität}{\\ \hspace*{\fill} - ChatGPT}
\paragraph*{Komplexität} 
    Einschrkung von TM bezüglich Zeit (Anzahl Rechenschritte einer TM) und Platz (Anzahl besuchte Felder einer TM)
    \begin{itemize}
        \item Worst case
        \item Average case
        \item Best case
    \end{itemize}

\mysubsection{Definition (Rechenzeit)}
    Sei \(M = (Q, \Sigma, \Gamma \Delta, s, F)\) eine DTM. Es bezeichne \(time_M:\Sigma^* \leadsto \mathbb{N}_0\) die partielle Funktion mit \(dom(time_M) = dom (\varphi_M)\), so dass \(time_M(w)\) für alle \(w \in dom (\varphi_M)\) die länge der Rechnung von M zur Eingabe w ist. Die \textbf{rechenzeit} von M zur Eingabe \(w \in dom (\varphi_M)\) ist \(time_M(w)\). \\
    Da im Folgenden nahezu alle Aussagen nur im Hinblick auf die Betrachtung Eingabelänge \(\to \infty\) sinnvoll sind, lassen wir in der nächsten Definition \textbf{endlich} viele Ausnahmen zu.

\mysubsection{Definition (zeitbeschränkt)}
    Eine \textbf{Zeitschranke} ist eine berechenbare Funktion \(t: \mathbb{N}_0 \to \mathbb{N}_0\) mit \(t(n)\leq n \quad \forall n \in \mathbb{N}_0\).\\ Sei \(t: \mathbb{N}\to \mathbb{R}_{\geq 0}\) eine Funktion. Eine TM M ist \textbf{t-zeitbeschränkt}, wenn M total ist und es ein \(n_0 \in \mathbb{N}_0\) gibt, so dass \(\forall w \in \Sigma^{\leq n_0}\) alle Rechnungen von M zur Eingabe w höchstens Länge \(t(|w|)\) haben.
    \vspace{0.5cm}
    \\
    Ist n ein Variablensymbol und t ein Term, der eine Funktion \(f: n \mapsto t\) festlegt, so verwendet man auch die Zeichnung t-zeitbeschränkt an Stelle von f-zeitbeschränkt. Man spricht also zum Beispiel von \(n^2\)-zeitbeschränkt der \(3x^3\)-zeitbeschränkt oder auch von \(f(n)\)-zeitbeschränkt für eine Funktion \(f: \mathbb{N}_0 \to \mathbb{N}_0\). Betrachten wir eine Funktion \(t: \mathbb{N}_0 \to \mathbb{R}_{\geq 0}\) als Zeitschranke, so ist damit die Fukntion \(t': \mathbb{N}_0 \to \mathbb{N}_0\), \(n \mapsto \lfloor t(n)\rfloor\) gemeint.

\mysubsection{Satz (lineare Beschleunigung)}
    Sei \(\tilde{c} > 1\), \(\epsilon > 0\) und \(t: \mathbb{N}_0 \to \mathbb{R}_{\geq 0}\) mit \(t(n) \leq (1+ \epsilon)n \quad \forall n \in \mathbb{N}_0\). Ist M eine \(\tilde{c}\cdot t(n)\)-zeitbeschränkt k-DTM, \(k \leq 2\), so gibt es auch eine \(t(n)\)-zeitbeschränkte k-TM M' mit L(M) = L(M').
    \paragraph*{Beweisskizze:}    
        Die Idee ist es sei \(c \leq 2\) wir wollen c Felder in einen Vektor der Länge c zusammenfassen und damit das Bandalphabet vergrößern und andererseits ungefähr "c Schritte auf einmal machen".
        Mit einem neuen Bandalphabet 
        \[
            ((\Gamma \cup \{\underline{a}: a \in \Gamma\})^c/\{\Box\}^c) \cup \{\Box\} \cup \Sigma
        \]
        dabei übernimmt \(\Box\) die Rolle von \(\{\Box\}^c\) und die Markierung ist für den Kopf der urspünglichen TM. 
        Konstruktion einer neuen TM M', die schneller ist als eine gegebene TM M.
        \begin{itemize}
            \item [1.Schritt:] Transformation der Eingabe in die "komprimierte"\ Form auf dem 2.Band (|w|+1 Schritte)
            \item [2.Schritt:] Simulation von M bandweise (1. und 2. Band vertauscht). Wir benutzen Vektoren der Länge c um c Felder von M darzusetellen. Zeichen mit einer Markierung \_ , markieren den Köpfen den simulierten TM M.
            \item [3.Schritt:] Die TM M' speichert in ihren Zuständen die Beiden Felder rechts und links des Feldes wo die simulierten Köpfe von M sind. 
            \begin{itemize}
                \item Simulation von c Schritten von M.
                \item Update der Felder, die von M verändert wurden 6 Schritte notwendig
            \end{itemize}
        \end{itemize}
        Insgesamt macht M' also \(|w|+ 1 + 6\cdot \frac{\tilde{c}t(|w|)}{c} + 10 \approx \frac{6\tilde{c}}{c} t(|w|) = t(|w|)\) für \(c := 6\tilde{c}\). In gewissem Sinne ist die obige Konstruktion invers zum \hyperref[subsec:2.18]{Alphabetwechel des bandalphabets bei der Konstruktion normierter TM}. Entsprechend steigt die Laufzeit auch nur um einen konstanten Faktor wenn wir die TM normieren 

\mysubsection{Satz (Alphabetwechel)}
    Ist M eine \(t(n)\)- zeitbeschränkte k-TM M mit Eingabealphabet \{0, 1\}, so gilt es eine Konstante \(c \geq 1\) und eine \(ct(n)\)-zeitbeschränkte k-TM M' mit Eingabealphabet \{0, 1\}, Bandalphabet \(\{\Box, 0, 1\}\) und L(M') = L(M).
    \medskip
    \(\leadsto\) Konstanten spielen nur eine untergeordnete Rolle für uns.

\mysubsection{Definition(Landau-Symbole)*}
    Für eine Funktion \(f: \mathbb{N}_0 \to \mathbb{R}_{\geq 0}\) sei
    \begin{itemize}
        \item [] \(o(f) = \{g: \mathbb{N}_0 \to \mathbb{R}_{\geq 0} : \forall \epsilon > 0 : \exists n_0 \in \mathbb{N}_0 : \forall n \geq n_0 : g(n) \leq \epsilon f(n)\}\)
        \item [] \(O(f) = \{g: \mathbb{N}_0 \to \mathbb{R}_{\geq 0} : \exists c > 0 : \exists n_0 \in \mathbb{N}_0 : \forall n \geq n_0 : g(n) \leq c f(n)\}\)
        \item [] \(\theta(f) = \{g: \mathbb{N}_0 \to \mathbb{R}_{\geq 0} : \exists c, \epsilon > 0 : \exists n_0 \in \mathbb{N}_0 : \forall n > n_0 : \epsilon f(n) \leq g(n) \leq cf(n)\}\)
        \item [] \(\Omega (f) = \{g: \mathbb{N}_0 \to \mathbb{R}_{\geq 0} : \exists \epsilon > 0 : \exists n_0 \in \mathbb{N}_0 : \forall n \geq n_0 : g(n) \geq \epsilon f(n)\}\)
        \item [] \(w(f) = \{g: \mathbb{N}_0 \to \mathbb{R}_{\geq 0} : \forall c > 0 : \exists n_0 \in \mathbb{N}_0 : \forall n \geq n_0 : g(n) \leq c f(n)\}\)
    \end{itemize}

\mysubsection{Bemerkung(Eigenschaften Landau-Symbole)*}
    Seien \(f, g : \mathbb{N}_0 \to \mathbb{R}_{\geq 0}\) Funktion dann gilt:
    \begin{itemize}
        \item [(i)] \(g \in o(f) \Leftrightarrow f \in w(g)\)
        \item [(ii)] \(g \in O(f) \Leftrightarrow f \in \Omega(g)\)
        \item [(iii)] \(g \in \theta(f) \Leftrightarrow ( g \in O(f) \text{ und }g \in \Omega(f))\)
    \end{itemize}

\mysubsection{Bemerkung(Eigenschaften Landau-Symbole)*}
    Seien \(f, g : \mathbb{N}_0 \to \mathbb{R}_+\) Funktion dann gilt: 
    \begin{itemize}
        \item [(i)] \(g \in o(f) \Leftrightarrow \underset{n \to \infty}{\lim} \frac{g(n)}{f(n)} = 0\)
        \item [(ii)] \(g \in O(f) \Leftrightarrow \underset{n \to \infty}{\lim sub} \frac{g(n)}{f(n)} < \infty\)
        \item [(iii)] \(g \in \Omega(f) \Leftrightarrow \underset{n \to \infty}{\lim inf} \frac{g(n)}{f(n)} > 0\)
        \item [(iv)]\(g \in w(f) \Leftrightarrow \underset{n \to \infty}{\lim} \frac{g(n)}{f(n)} = \infty\)
    \end{itemize}
    Bei diesen Klassen ist es üblich \("\ = "\) statt \(" \in "\) zu verwenden, also beispielsweise \(g = O(f)\) statt \(g \in O(f)\) zu schreiben.
    \medskip
    Unsere Konstruktion für lineare Beschleunigungbedurfte mehrere Bänder. Tatsächlich kann es einen Unterschied machen ob ein oder mehrere Bänder zur Verfügung stehen. Als Beispiel betrachten wir Binärpalindrome:\\ Für \(w \in \{0, 1\}^*\). Sei \(w^R = w(|w| \cdots w(1))\). Die Sprache der Binärpalindrome ist folglich \(\{w \in \{0, 1\}^* : w = w^R \}\).

\mysubsection{Proposition(Binärpalindrome erkennen)*}
    Es gibt eine \(t(n)\)- zeitbeschränkte DTM M mit \(L(M) = \{w \in \{0, 1\}^* : w = w^R\}\) und \(t(n) = O(n)\).
    \begin{proof}
        Eine 2-DTM M die die Eingabe von links nach rechts durchläuft und dabei die Eingabe auf das 2.Band kopiert, dann auf dem ersten Band den Kopf zurück auf die Ausgangsposition bringt, dann gleichzeitig die Eingabe vpn links nach rehcts und die Kopie symbolweise vergeicht und genau dann akzeptiert, wenn alle Vergleiche positiv waren, dann erkennt M die Binärpalindrome.
    \end{proof}

\mysubsection{Satz (Zeitbeschränkung einer Turingmaschine)*}
    Ist M eine \(t(n)\)-zeitbeschränkte 1-DTM mit \(L(M) = \{w \in \{0, 1\}^* : w = w^R\}\), so ist \(t(n) = \Omega (n^2)\)
    \begin{proof}
        Sei M eine \(t(n)\)-zeitbeschränkter 1-DTM mit Eingabealphabet \{0, 1\}, \(L := \{ww^R : w \in \{0, 1\}^*\} \subseteq L(M)\) und \(t(n) \not = \Omega (n^2)\). Wir zeigen \(L \not = L(M)\). Aus der Existenz von M folgt die Existenz einer \((2t(n) + 1)\)-zeitbeschränkte TM M' mit Zustandsmenge Q wobei \(|Q \leq 2|\), Eingabealphabet \{0, 1\} und L(M) = L(M'), so dass der Kopd von M' am Ende jeder Rechnung auf dem Feld steht auf dem er am Anfang stand. Wir betrachten die Felder auf dem Band von M' als beginnend mit dem Feld auf dem der Kopf zu Rechungsbeginn steht mit dem natürlichen Zahlen 1, 2, 3, ...\\ Für jede Rechung R von M' sie \(C_i (R)\) die Folge \(q_1, \cdots, q_l\) von Zuständen von M', so dass M' während der Rechung R insgesamt l und die Grenze zwischen i und i+1 überschritt und so dass \(q_j\) für \(j \in [l]\) der Zustand ist, in dem sich M' unmittelbar vor dem Auführen der Anweisungen befindet, die das j-te übertreten auslöst. Die Folgen \(C_i(R)\) werden Crossing-Sequenz von M' gennant.\\ Die wesentliche Eigenschaft dieser Crossing-Sequenzen, die wir benutzen, ist wie folgt:\\ Ist R eine akzeptierte Rechung von M' zur Eingabe u und S eine Rechung zur Eingabe v und sind \(i \in [|u|]\) und \(j \in [|v|]\) mit \(C_i(R) = C_j(S)\), so gilt 
        \[
            u(1) \cdots u(i) v(j+1) \cdots v(|v|) \in L(M')   
        \]
        Wir nutzen die kurze laufzeit von M' nun mittels Schubfachprinzip gleiche Crossing-Sequenzen zu Rechnungen von unterschiedlichen Eingaben zu finden, um zu folgern, dass M' ein Wort akzeptiert, das nicht in L liegt. \\ Sei \(\epsilon := \frac{1}{300 \log_2 |Q|}\). Aus \(t(n) \not = \Omega(n^2)\) folgt die Existenz eines und mit \(\epsilon n \leq 1\). Sei \(n' := \frac{n}{4}\).\\ Für jedes Wort \(L' := \{w' 0^{\frac{n}{2} w'^R : w' \in \{0, 1\}^{=\frac{n}{4}}}\} \subseteq L\) betrachten wir die kürzeste Crossing-Sequenz \(C_w \in \{C_{n'}(R), \cdots, C_{n-n'}(R)\}\) einer akzeptierten Rechung R von M' zur Engabe w. Für jede Rechung von M' zu einer Eingabe \(w \in \{0, 1\}^{= n}\) gilt 
        \[
            \sum\limits_{i=n'}^{n-n'}|C_i(R)| \leq 2t(n)+1
        \]
        wobei |C| die länge von C bezeichnet. Für jedes Wort \(w \in \{0, 1\}^{=n}\) folgt damit 
        \[
            |C_w| \leq \frac{2t(n)+1}{n-2n' + 1} \leq \frac{4cn^2}{\frac{n}{2} + 1} \leq 16\epsilon u    
        \]
        Die Anzahl der Crossing-Sequenzen von M' der länge \(16\epsilon n -1\) ist höchstens 
        \[
            (|Q| + 1)^{16\epsilon n} \leq \frac{1}{2} (2|Q|)^{16 \epsilon n} \leq \frac{1}{2} |Q|^{32 \epsilon n} \leq 2^{n'-1} 
        \]
        \(mit \Rightarrow |Q| = 2^{\log_2 |Q|}\) \\ Die Anzahl der Wörter in L' ist aber \(2^{n'}\). Es gibt also voneinander verschiedene Wörter in L' ist und \(i \in [|u|]\) und \(j \in [|v|]\) mit \(C_i(R) = C_j(S)\) wobei R die Rechung von u ist und S die Rechung von v in M'. Es gibt also einen Präfix u' von u mit \(|u'| \leq n'\) und einen Suffix v' von v mit \(|v'| \leq n'\), sodass \(u' v' \in L(M') = L(M)\) gilt. Da u und v verschieden sind gilt \(u' v' \not \in L\). 
        \medskip
        Im Sinne von \hyperref[subsec:7.9]{Satz 7.9} macht es zwar einen Unterschied ob TM mit einem oder mit mehreren Bändern betrachtet werden, der Anstieg der Laufzeit ist bei Simulation mehrerer Bänder auf einem band mittels Spurentechnik aber nur quadratisch.
    \end{proof}

\mysubsection{Satz(Reduktion auf 1-Band-TM)*}
    Ist M eine \(t(n)\)- zeitbeschränkte TM, so gibt es eine \(t'(n)\)- zeitbeschränkte 1-TM mit \(t'(n) = O((t(n))^2)\), die dieselbe Sprache erkennt. Während der laufzeitanstieg bei der Reduktion auf ein Band quadratisch sein kann, ist die Reduktion von k auf zwei Bänder effizienter möglich 

\mysubsection{Satz(Codekonstruktion und Simulation)*}
    Ist für alle deterministischen TM \(M = (\{0, \cdots, n\}, \{0, 1\}, \{0, 1, \Box\}, \Delta, 0, \{0\})\) und für alle Wörter \(w \in \{0, 1\}^*\) das Wort code (M, w) ein geeigneter Codefür (M, w), so gibt es eine 2-DTM U, so dass folgendes gilt:

    \begin{itemize}
        \item [(i)] \(\forall\) DTM M wie oben und \(\forall w \in \{0, 1\}^*\) akzeptiert U den Code code(M, w) genau dann, wenn M das Binärwort w akzeptiert.
        \item [(ii)] Für alle zeitschranken t, alle t-zeitbeschränkten DTM M wie oben, gibt es ein \(n_0\), sodass \(\forall w \in \{0, 1\}^{\leq n_0}\) eine Konstante \(c \in \mathbb{N}\) gibt, sodass die Rechnung von U zur Eingabe code(M, w) Länge höchstens \(ct(|w|)\log(t(|w|))\) hat.
    \end{itemize}

    \begin{proof}
        \textbf{Idee: }
            \begin{enumerate}
                \item Verwende \hyperref[subsec:2.18]{Spurentechnick} um die womöglich vielen Bänder von M auf einem Band zu simulieren.
                \item Inteligentes Speichermanagement: Simulierte Köpfe bleiben in der Mitte simuliertes Band "bewegt" sich + geschicktes Lücken lassen im Speicher.
            \end{enumerate}
            Bild für den Fall, dass M eine Band hat
            \[ABBHIEREINFÜGEN\]
            Genauer verfügt U bei Eingabe \(code(M,w)\) wie folgt:
            \begin{itemize}
                \item Zu Beinn werden einige Initisierungsschritte ausgeführt, die es erlauben die k Bänder von M in jeweils einer Spur auf den ersten Band von U geeignet zu simulieren, wobei auf dem zweiten Band eine Repräsentition von M erstellt wird. Es wird aber sichergestellt, beispielsweise mittels mehrerer Spuren auf dem zweiten Band, dass das zweite Band noch verwendet werden kann um mit un der Abschnittslänge linearem Aufwand Abschnitte auf dem ersten NBand verschienen zu können.
                \item Während der Simulation von M wird jede Spur auf dem ersten Band von U in Abschnitte unterteilt 
                \begin{itemize}
                    \item Die Felder auf dem ersten Band von U auf dem der Kopf zu Beginn steht bildet für alle Spren \(i \in [k]\) den aus einem Feld bestehenden Abschnitt \(H^{(i)}\). Die Simulation von M wird so mittels angepasster Spurentechnik durchgeführt, dass die Simulierte Kopfposition auf jeder Spur \(z \in [k]\) das Feld im Abschnitt \(H^{(i)}\) ist und auf keiner Spur wird dieses Feld am Ende der Simulation eines Schrittes von M mit \# Beschrieben sein.
                    \item Auf jeder Spur \(i \in [k]\) ist unmittelbar links von \(H^{(i)}\) der Abschnitt \(L_1^{(i)}\) der Länge 2 und für jedes \(j \leq 1\) unmittelbar links von \(L_j^{(i)}\) der Abschnitt \(L_{j+1}^{(i)}\) der länge \(2^{j+1}\). 
                    \item Analog ist auf jeder Spur \(i \in [k]\) unmittenbar rechts von \(H^{(i)}\) der Abschnitt \(R_1^{(i)}\) der Abschnitt \(R_{j+1}^{(i)}\) der länge \(2^{j+1}\).
                    \item Jder Abschnitt auf einer Spur \(i \in [k]\), außer \(H^{(i)}\), wird beim ersten Verwenden in dem Sinne als halb leer initialisiert, dass er zur Hälfte mit den die Blank-Symbole von M repräsentierenden Symbolen beschrieben.
                    \item Zu jeden Zeitpunkt stehen die Lückensymbole \# in einem Abschnitt am weitesten links im Abschnitt.
                    \item Ein Abschnitt \(L_j^{(i)}\) oder \(R_j^{(i)}\) heißt \textbf{leer}, wenn alle Symbole in diesem Abschnitt \# sind, \textbf{halb leer} wenn die Hälfte der Symbole in diesem Abschnitt \# oder er nicht initial ist und \textbf{voll} wenn kein Symbol in diesem Abschnitt \# ist.
                    \item Die Darstellung der durch die Spuren repräsentierten bändervon M ist dabei so zu verstehen, dass die sich wie bei derSpurentechnick üblichen Darstellung der Bänder ergeben wenn die Felder mit den Lückensymbolen \# ignoriert werden.
                    \item Vor Simulationsbegin wird w als Eingabe an die simulierte Maschine M übergeben ohne dabei die Lückensymbole \# zu überschreiben. 
                \end{itemize}
                \item Werden nun die Bandbewegungen geeignet durch die Spurenmanipulationen realisiert, so kann U die Turingmaschine M simulieren, da den Abschnitt \(H^{(i)}\) die zur Simulation nötigen Informationen über die gegenwärtige Konfiguration von M entnommen werden können.
                \item Bandbewegungen nach links (die Kopfbewegung nach rechts sinnvoll) werden durch manipulationen der entsprechenden Spur wie folgt simuliert:
                \begin{itemize}
                    \item Für das minimale \(j_0 \leq 1\), so dass \(R_{j_0}^{(i)}\) nicht leer ist, ersetzt das erste Symbol der rechten Hälfte, falls \(R_{j_0}^{(i)}\) halb voll ist, und der linken Hälfte, falls \(R_{j_0}^{(i)}\) voll ist, das Symbol im Abschnitt \(H^{(i)}\) und die \(2^{j_0 - 1}-1\) weitere Symbole Symbole der Hälfte von \(R_{j_0}^{(i)}\) werden im unveränderter Reihenfolge auf die rechten Hälften von \(R_1^{(i)}, \cdots, R_{j_0-1}^{(i)}\) geschrieben.
                    \item Die von \# verschiedenen Symbole der Abschnitte \(L_1^{(i)}, \cdots, L_{j_0^{(i)}}\) werden im unveränderter Reihenfolge auf den Abschnitt \(L_{j_0}^{(i)}\) und die rechten Hälften der Abschnitte \(L_1^{(i)}, \cdots, L_{j_0 -1}^{(i)}\) geschrieben und die Symbole in den linken Hälften der Abschnitte \(L_1^{(i)}, \cdots, L_{j_0 -1}^{(i)}\) werden durch \# überschrieben.
                    \item Schließlich wird das nun ersetzte Symbol, das zuvor im Abschnitt \(H^{(i)}\) stand auf die rechte Hälfte des Abschnitts \(L_1^{(i)}\) geschrieben.
                \end{itemize}
                \item Bandbewegungen nach rehcts werden analog simuliert.
                \item Die Bandbewegungen lassen sich eindeutig in dieser Weise simulieren da für alle \( j \leq 1\) zu jede, Zeitpunkt für die Abschnitte \(L_j^{(i)}\) und \(R_j^{(i)}\) folgendes gilt: Entweder sind beide Abschnitte halb voll oder einer ist leer und der andere voll.
                \item Beisliel hier!
                \item Die Laufzeit von U ergibt sich im Wesentlchen aus der Simulation der Bandbewegungen. Bei einer Bandbewegung ist die Anzahl der bewtrachteten Felder \(O(\sum\limits_{j = 1}^{j_0} 2^j) = O(2^{j_0})\) für ein \(j_0 \leq 1\). Dank des zweiten Bandes lassen solche Bandbewegungen im Zeit \(O(2^{j_0})\) durchführen. Ist t die Laufzeit von M zur Eingabe w, so ist eine solche Bandbewegung höchstens \(\frac{t}{2^{j_0}}\) und durchzuführen. Als Laufzeit von U ergibt sich so bis auf konstante Faktoren \(\sum\limits_{j_0 = 1}^{\log_2 t} \frac{t}{2^{j_0}} = t\log_2 t\).
                \begin{itemize}
                    \item \(\frac{t}{2^{j_0}}\) \(\to\) häufigkeit
                    \item \(2^{j_0}\) \(\to \) Kosten
                \end{itemize}
            \end{itemize}
    \end{proof}
\newpage
\mysubsection{Satz (Universelle TM-Simulation mit Zeitbeschränkung)*}
    Ist für alle TM \(M = (\{0, \cdots, n\}, \{0, 1\}, \{\Box, 0, 1\}, \Delta, 0, \{0\})\) und alle Wörter \(w \in \{0, 1\}^*\) das wort \(code(M, w)\) ein geeigneter Code für (M, w), so gibt es eine 2-TM U, so dass folgendes gilt.
    \begin{itemize}
        \item [(i)] \(\forall\) TM M wie oben und \(\forall w \in \{0, 1\}^*\) akzeptiert U den Code \(code(M, w)\) genau dann, wenn M das Wort w akzeptiert.
        \item [(ii)] Für alle Zeitschranken t, \(\forall\) t - zeitbeschränkten TM M wie oben gibt es ein \(u_0\) und c, so dass \(\forall w \in \{0, 1\}^{\geq n_0}\) jede Rechnung von U zur Eingabe \(code(M, w)\) Länge höchstens \(ct(|w|)\) hat, falls U eine Rechnung von M zur Eingabe w simuliert, die von M akzeptiert wird.
    \end{itemize}
    \begin{proof}
        Die Idee ist die Bänder ist die Bänder einer TM zu Simulieren. Genauer verfährt U bei Eingabe \(code (M, w)\) wie folgt:
        \begin{itemize}
            \item Zu Beginn werden einige Initialierungsschritte ausgeführt, die es erlauben die k Bänder von M nacheinander auf dem ersten Band von U zu simulieren, wobei auf einer Spur auf dem ersten Band eine repräsentation von M erzeugt wird, die während der Simulation so verschoben wird, das sie vor jedem zu simulierenden Schritt unmittelbbar rechts von Kopf auf dem ersten Band steht. 
            \item Die k-TM M wird mit w als Eingabe simuliert, indem nacheinander für jedes Band \(i \in [k]\) alle Kopfbewegungen auf Band i bis zum Rechnungsende simuliert werden. Es werden also zunächst alle Kopfbewegungen auf dem ersten, dann auf dem zweiten Band usw. simuliert.
            \begin{itemize}
                \item Dies erfordert im Sinne der Arbeitsweise von M Kenntnis der in jedem Schritt gelesenen Bandsymbol auf allen anderen Bändern und der gewählten Instruktion, weshalb diese zu Beginn der Simulation des ersten Bandes für die nicht simulierten Bänder nicht deterministisch geraten werden und auf dem zweiten Band gespeichert werden.
                \item Bei der Simulation der weiteren Bänder wird dann geprüft, ob die tatsächlich dort auftretenden Bandsymbole zu den geratenen Bandsymbole und Instruktion passen. Ist dies nicht der Fall \\ \(\rightarrow\) Terminierung mit Nichtakzeptanz.\\ Andernfalls war die Simulation erfolgreich und die Eingabe wird genau dann akzeptiert, wenn die Simulation die Simulierte Eingabe akzeptiert hat.
            \end{itemize}
        \end{itemize} 
        Nach Konstruktion sieht man leicht, dass es ein \(c, n_0 \in \mathbb{N}\) gibt, so dass U höchstens \(ct(|w|)\) Schritte benötigt falls \(|w| \geq n_0\).
    \end{proof}
\mysubsection{Definition (Zeitbeschränkte Funktionen und Sprachen)*}
    Für eine Menge T von Funktionen von \(\mathbb{N}_0 \to \mathbb{R}_{\geq 0}\) definieren wir
    \begin{itemize}
        \item \textbf{DTIME}(T) := \(\{L(M) \subseteq \{0, 1\}^* : \exists t \in T\) ist M eine t-zeitbeschränkte DTM\}
        \item \textbf{FTIME}(T) := \(\{\varphi(M) : \mathbb{N}_0 \to \mathbb{N}_0 : \exists t \in T\) und M ist eine t-zeitbeschränkte DTM\}
        \item \textbf{NTIME}(T) := \(\{L(M) \subseteq \{0, 1\}^* : \exists t \in T\) ist M eine t-zeitbeschränkte TM\}
    \end{itemize}

\mysubsection{Definition (Zeitkonsturierbar)}
    Eine Zeitschranke t ist genau dann \textbf{Zeitkonsturierbar}, wenn es eine DTM M mit Eingabealphabet \{1\} gibt, so dass \(time_M(1^n) = t(n)\) gibt.

\mysubsection{Bemerkung (Polynomielle Zeitkonstruierbarkeit)*}
    \begin{itemize}
        \item [(i)] Ist p ein Polynom in einer variable über \(\mathbb{Z}\) mit \(p(n) \geq n+1 \quad \forall n \in \mathbb{N}_0\), so ist \(t: \mathbb{N}_0 \to \mathbb{N}_0\), \(n \mapsto p(n)\) Zeitkonstruierbar.
        \item [(ii)] Ist t Zeitkonsturierbar, so ist auch \(T: \mathbb{N}_0 \to \mathbb{N}_0\), \(n \mapsto 2^{t(n)}\) Zeitkonsturierbar.
    \end{itemize}

\mysubsection{Satz (Zeithirarchiesatz dür deterministische TM)}
    Sei t eine Zeitschranke und T eine Zeitkonsturierbare Zeitschranke mit \(t(n) \log t(n) = o(T(n))\). Dann gilt \textbf{DTIME}(t(n))\(\not \subseteq\) \textbf{DTIME}(T(n)).
    \begin{proof}
        Sei \(M_0, M_1, \cdots\) eine geeignete effektive Aufzählung deterministische TM der Form \((\{0, \cdots, n\}, \{0,1\}, \{\Box, 0, 1\}, \Delta, 0, \{0\})\), so dass \(\forall\) DTM M dieser Form es unendlich viele \(e \in \mathbb{N}_0\) mit \(M = M_e\) gibt.\\Wir betrachen eine DTM U, die bei Eingabe \(1^e\) die DTM \(M_e\) bei eingabe \(1^e\) simuliert und genau dann akzeptiert, wenn die Simulation terminiert und nicht akzeptiert. \\ Nach \hyperref[subsec:7.11]{Satz 7.11} können wir U so wählen, dass es \(\forall c \geq 1\) und alle \(ct(n) \log t(n)\) - zeitbeschränkter DTM M der obigen Form ein \(e_{c_1 M}\) mit \(M = M_{e_{c_1}M}\), so dass die Rechnung von U zur Eingabe \(1^e_{c_1 M}\) höchstens Lämge \(T(e_M, c)\) hat.\\Da T zeitbeschränktist, ist es möglich U so zu modifizieren, dass U die Simulation immer nach \(T(e)\) Schritten abbricht, wobei wir bei Abbruch nie akzeptieren.\\Dann ist U offenbar \(T(n)\) - zeitbeschränkt, es gilt also \(L(U) \in \textbf{DTIME}(T(n))\). \\Wir zeigen nun, dass keine \(t(n)\) - zeitbeschränkte DTM M existiert mit \(L(M) = L(U)\)\\Ist M eine \(t(n)\) - zeitbeschränkte DTM, so existiert nach \hyperref[subsec:7.4]{Satz 7.4} ein \(c \geq 1\) und eine \(ct(n)\) - zeitbeschränkte DTM M' der obigen Form mit \(L(M) = L(M')\).\\Die TM U bricht dann die Simulation der Eingabe \(1^{e_{c_1 M'}}\) nicht ab, es gilt also \(L(M_{e_{c_1 M'}}) \not = L(U)\) und damit wegen \(L(M_{e_{c_1 M'}}) = L(M') = L(M)\) also \(L(U) \not = L(M)\).
    \end{proof}

\mysubsection{Proposition(Zeitbeschränkte TM-Äquivalenz)*}
    Für alle \(t:\mathbb{N}_0 \to \mathbb{R}_{\geq 0}\) und t - zeitbeschränkte TM M ibt es eine DTM M' mit \(L(M) = L(M')\), sodass M' \(2^{O(t)}\) - zeitbeschränkt ist.
    \begin{proof}
        \(\forall\) M wie oben gibt es \(c \in \mathbb{N}_0\) sodass es bei Eingabe w höchstens \(c^{t(|w|)}\) viele rechnungen zur Eingabe w gibt. \\
        \(\leadsto\) Brute-Force
    \end{proof}

\mysubsection{Satz (Zeithirarchiesatz für nichtdeterministische TM)}
    Sei t eine Zeitschranke und T eine zeitkonsturierbare Zeitschranke mit \(t(n+1) = o(T(n))\). Dann gilt \(\textbf{NTIME}(t(n)) \not \subseteq \textbf{NTIME}(T(n))\). 
    \begin{proof}
        Sei \(M_0, M_1, \cdots\) eine geeignete Aufzählung von TM der Form \(M = (\{0, \cdots, m\}, \{0, 1\}, \{\Box, 0, 1\}, \Delta, 0, \{1\})\), so dass \(\forall\) TM M dieser Form unendlich viele \(e \in \mathbb{N}\) mit \(M = M_e\) gibt.\\
        Sei \(l_0 := 0\). Sei \(l_e\) für \(e \geq 1\) deterministisch durch \(l_e := 2^{T(l_{e-1})}\).\\
        Für \(e \in \mathbb{N}_0\), sei \(I_e := (l_e, l_{e+1}) \cap \mathbb{N}_0\).\\
        Wir betrachten eine TM U, die bei Eingabe \(1^x\) wie folgt verfährt.
        \begin{itemize}
            \item Es wird \(e\in\mathbb{N}_0\) mit \(x \in I_e\) bestimmt.
            \item Gilt \(x < l_{e+1}\), wird \(M_e\) bei Eingabe \(1^{x+1}\) simuliert und dann genau akzeptiert, wenn die Simulation terminiert und akzeptiert.
            \item Gilt \(x = l_{e+1}\), so werden alle Rechnungen von \(M_e\) bei Eingabe \(1^{l_e +1}\) simuliert und es wird genau dann akzeptiert, wenn \(M_e\) nicht akzeptiert. 
        \end{itemize}
        Da T Zeitkonsturierbar ist, lässt sich gegeben x, \(I_e\) (bzw. e) bestimmen. Nach \hyperref[subsec:7.12]{Satz 7.12} können wir daher U so wählen, dass es \(\forall x \leq 1\) und \(\forall\) \(ct(n)\) - zeitbeschränkter TM M der obigen Form ein \(e_{c,M}\) mit \(M_{e_{c,M}} = M\) gibt, so dass alle rechnungen von U zur Eingabe \(l^x\) mit \(x \in I_{e_{c, M}} / \{l_{e_{c,M + 1}}\}\) höchstens Länge T(x) haben (falls sie eine akzeptierte Rechnung simmulieren) und so dass die Rechnung von U zur Eingabe \(1^{l_{e_{c,M + 1}}}\) hächstens \(2^{T(l_{e_{c,M}})} = l_{e_{c,M + 1}} \leq T(l_{e_{c, M +1}})\) hat.\\
        Da T Zeitkonsturierbar ist, ist es möglich U so zu modulieren, dass U die Simulation immer nach T(x) Schritten abbricht, wobei U dann bei abgebrochener Simulation wie akzeptiert.\\
        Dann ist U offenbar T(n) - Zeitbeschränkt, es gilt also \(L(U) \in \textbf{NTIME}(T)\).\\
        Wir zeigen nun, dass keine t(n) - zeitbeschränkte TM M mit L(U) = L(M) existiert.\\
        Ist M eine t(n) - zeitbeschränkte TM mit L(M) = L(U), so existiert nach \hyperref[subsec:7.4]{Satz 7.4} ein \(x \leq 1\) und eine ct(n) - zeitbeschränkte TM M' der obigen Form mit 
        \[
            L(M') = L(M) = L(U)
        \]
        Die TM U bricht dann die simulation bei allen Eingaben \(1^x\) mit \(x \in I_{e_{c, M'}}\) nicht ab, \(\forall x \in I_{e_{c, M'}} / \{l_{e_{c, M' +1}}\}\) gilt also 
        \[
            1^x \in L(M') \Leftrightarrow 1^x /in L(U) \Leftrightarrow 1^{x+1} \in L(M')
        \]
        Folgendes gilt 
        \[
            1^{l_{e_{c,M' + 1}}} \in L_{M'} \Leftrightarrow 1^{l_{e_{x, M' + 1}}} \in L(M') \Leftrightarrow ...
        \]
        Dies ist im Wiederspruch zu L(M') = L(M).
    \end{proof}
    Im Folgenden fixieren wir eine Aufzählung \(M_0, M_1, \cdots\) aller normierten DTM.

\mysubsection{Definition(abstraktes Komplexitätsmaß)}
    Eine partielle Funktion \(\phi : \mathbb{N}_0 \times \{0, 1\}^* \leadsto \mathbb{N}_0\) heißt \textbf{abstraktes Komplexitätsmaß} falls folgendes gilt:
    \begin{itemize}
        \item [(i)] \(\phi(e,x) \downarrow \Leftrightarrow M_e(x) \downarrow \quad \forall e \in \mathbb{N}_0, x \in \{0, 1\}^*\)
        \item [(ii)] Die Relation \(\{(e, x, t) \in \mathbb{N}_0 \times \{0, 1\}^* \times \mathbb{N}_0 : \phi(e,x) = t\}\) ist im Sinne entscheidbar, dass die Funktion \(\psi : \mathbb{N}_0 \times \{0, 1\}^* \times \mathbb{N}_0 \to \{0, 1\}\) mit \(\psi(e, x, t) = 1 \Leftrightarrow \phi (e,x) = t \quad \forall e, x, t\) berechenbar ist.
    \end{itemize}

\mysubsection{Definition(T-Beschränkte TM und DCOM)*}
    Sei \(t: \mathbb{N}_0 \to \mathbb{N}_0\) eine Funktion für \(e \in \mathbb{N}_0\) wird die TM \(M_e\) als \textbf{t - \(\phi\) - beschränkt} bezeichnet, wenn \(M_e\) total ist und es ein \(n_0 \in \mathbb{N}_0\) gibt, sodass \(\phi(e, x) \leq t(|x|) \quad \forall x \in \{0, 1\}^{\geq n_0}\) gilt. Wir setzen:
    \[
        \textbf{DCOM}^{\phi}(t) := \{L(M) \subseteq \{0, 1\}^* : M ist t - \phi - beschränkt\}  
    \] 
    Beispielsweise ist Zuordnung \(e, x \mapsto time_{M_e} (x)\) ein abstraktes Komplexitätsmaß.

\mysubsection{Satz (Lückensatz)}
    Sei \(\phi\) ein abstracktes Komplexitätsmaß und sei \(f, g : \mathbb{N}_0 \to \mathbb{N}_0\) streng monoton wachende Funktion. Dann gibt es eine streng monoton wachende berechenbare Funktion \(t : \mathbb{N}_0 \to \mathbb{N}_0\) mit \(f(n) \leq t(n) \quad \forall n \in \mathbb{N}_0\) und \(\textbf{DCOMP}^{\phi}(t) = \textbf{DCOMP}(g \circ t)\).
    \begin{proof}
        Wir konstruieren t so, dass es \(\forall e \in \mathbb{N}_0\) ein \(n_0 \in \mathbb{N}_0\) gibt, so dass \(\phi(e, x) \not \in [t(|x|) + 1, g(t(|x|))] \quad \forall x \in \{0, 1\}^{\geq n_0}\) gilt.
        \[SKIZZE HIER \]
        Dafür genügt es t(n) für \(n \in \mathbb{N}_0\) so zu wählen, dass \(\phi (e,x) \not \in [t(n) + 1, g(t(n))] \quad \forall e < n\) und \(x \in \{0, 1\}^{= n}\) gilt.\\
        \(\forall n \in \mathbb{N}_0\) ist \(\{\phi(e,x) : e < n, x \in \{0, 1\}^{=n}\}\) eine endliche Menge.\\
        \(\Rightarrow\)(hier nocht limits eintragen!!!) effektive Bestimmung einer abstrakten Schranke möglich.\\
        \(\Rightarrow\) t(n) groß genug wählen genügt
        \paragraph{Genauer:}
            \(t(n) := min \{m \in \mathbb{N}_0 : t(n-1)<m \wedge f(n) < m \wedge \forall e < n : \forall x \in \{0, 1\}^{=n} : \phi(e,x) \in [m+1, g(m)]\}\).
            \(\Rightarrow\) t ist berechenbar, streng monoton wachend und \(f(n) \subseteq t(n)\)\\
            Sei \(L \in \textbf{DCOM}^{\phi}(g /circ t)\). Sei \(e \in \mathbb{N}_0\) so dass \(M_e\) eine \((g \circ t)\) - \(\phi\) - zeitbeschränkte TM ist mit \(L(M_e) = L\). Wir zeigen nun, dass \(M_e\) t - \(\phi\) - beschränkt ist. Sei \(x \in \{0, 1\}^{\geq e + 1}\). Es genügt zu zeigen , dass \(\phi (e, x) \leq t (|x|)\). Nach Konstruktion von t gilt \(\phi (e, x) \not \in [t(|x|) + 1, g(t(|x|))]\) und nach Wahl von e gilt \(phi (e, x) \leq g ((t(|x|)))\). Es folgt also \(\phi(e,x) \leq t(|x|)\).
    \end{proof}
\coversection{ReguläreSprachen/regul.png}{Komplexitätsklassen P und NP}
{}

\mysubsection{Definition(Komplexitätsklassen P, FP und NP)*}
    Wir setzen 
    \begin{itemize}
        \item \(\textbf{P} := \textbf{DTIME} (poly),\)
        \item \(\textbf{FP} := \textbf{FTIME}(poly),\)
        \item \(\textbf{NP} := \textbf{NTIME}(poly)\)
    \end{itemize}
    wobei poly := \{\(n^c + c, c \in \mathbb{N}_0\)\}
    Offenbar gilt \textbf{P} \(\subseteq\) \textbf{NP}, allerding ist unklar ob \textbf{P} \(\not \subseteq\) \textbf{NP}.

\mysubsection{Definition (p-m-Reduktion)}
    Eine Sprache A über \{0, 1\} ist in \textbf{polynomieller Zeit manny-to-one- reduzierbar}, auch \textbf{p-m-reduzierbar}, kurz \(A \leq^P_m B\), auf eine Sprache B über \{0, 1\}, wenn es eine Funktion \(f \in\textbf{FP}\) gibt, so dass 
    \[
        w\in A \Leftrightarrow f (w) \in B, \quad \forall w \in \{0, 1\}^*  
    \] 
    gilt. Gelte \(A \leq^P_m\) und \(B \leq^P_m\), so sind A und B \textbf{p-m-äquivalent}, kurz \(A=^P_m B\)

\mysubsection{Bemerkung(Eigenschaften der p-m-Reduktion)*}
    \begin{itemize}
        \item [(i)] \(\leq^P_m\) transitiv
        \item [(ii)] Seien A und B Sprachen mit \(A \leq ^P_m B\). Aus \(B \in \textbf{P}\), folgt \(A \in \textbf{P}\) und aus \(B \in \textbf{NP}\) folgt \(A \in \textbf{NP}\).
        \item [(iii)] Alle Sprachen \(L \in \textbf{P}\) mit \(\varnothing \not = L \not = \{0,1\}^*\) sind p-m-äquivalent.
    \end{itemize}

\mysubsection{Definition (\textbf{NP}-schwer, \textbf{NP}-vollständig)}
    Eine Sprache S wird als \textbf{NP}- Schwer bezeichnetm, wenn \(L \leq ^P_m S \quad \forall L \in \textbf{NP}\) gilt. Gilt zusätzlich \(S \in \textbf{NP}\), so wird S als \textbf{NP}-vollständig bezeichnet.

    \paragraph*{Nächstes ziel:} 
        Finden eines NP-vollständigen Problems.

\mysubsection{Definition (aussagenlogische Formel)}
    Sei Var eine abzählbare Menge mit \(\lnot, \wedge \not \in Var\). Die Menge der \textbf{aussagenlogischen Formeln} A ist die ??? Inklusion kleinster Menge von Wörtern über \(Var \cup \{\lnot, \wedge, (,)\}\) mit \(a \in A \quad \forall a \in Var\) und \(\lnot \varphi, (\varphi \wedge \psi) \in A \quad \varphi, \psi \in A\). Für \(\varphi, \psi, \in A\) schreiben wir statt \(\lnot(\lnot \varphi \wedge \lnot \psi)\) auch \((\varphi \vee \psi)\) und statt (\(\lnot \varphi \vee \psi\)) auch (\(\varphi \to \psi\)). Wie üblich vereibaren wir Klammerregeln zur besseren lesbarkeit: \\ \(\lnot\) bidet stärker als \(\wedge\) und \(\vee\). 
    \paragraph*{Beispielsweise}
        \(\lnot a \wedge b \wedge c = ((\lnot a \wedge b)\wedge c)\)
    
\mysubsection{Definition(Grundbegriffe der aussagenlogischen Formeln)*}
    \begin{itemize}
        \item [(i)]DieElemente von Var heißen \textbf{Variablen}.
        \item [(ii)] Für \(\varphi, \varphi_1, \cdots, \varphi_n \in A\) mit \(n \in \mathbb{N}\) heißt die aussagen ??? Formel \( \lnot \varphi\) \textbf{Negation} von \(\varphi\), die aussagen ??? Formel \(\varphi_1 \wedge \cdots \wedge \varphi_n\) heißt \textbf{Konjunktion} von \(\varphi_1, \cdots, \varphi_n\) und \(\varphi_1 \vee \cdots \vee \varphi_n\) heißt \textbf{Disjunktion} von \(\varphi_1, \cdots, \varphi_n\).
    \end{itemize}

\mysubsection{Definition (konjunktive Normalform)}
\begin{itemize}
    \item [(i)] Ein \textbf{Literal} ist eine variable oder eine negaltion einer Variablen.
    \item [(ii)] Eine \textbf{disjunktive Klausel} ist eine Disjunktion von Literalen.
    \item [(iii)] Eine aussagen??? Formel in \textbf{konjunktiver Normalform}, kurz KNF ist eine Konjunktion disjunkiver klauseln
\end{itemize}

\mysubsection{Definition (Wahrheitswert)}
    Eine \textbf{Belegung} ist eine Funktion \(b : Var \to \{0,1\}\). Der \textbf{Wahrheitswert} \(val_b(\varphi)\) einer Formel \(\varphi\) für eine Belegung b ist induktiv wie folg definiert. 
    \vspace*{0.5cm}
    \\
    \(
        \forall a \in Var \text{ sei } val_b(a) = b(a) \text{ und für } \varphi, \psi \in A \text{ sei} 
    \)
    \begin{equation}
        val_b(\lnot \varphi) := 
        \begin{cases}
            1 & val_b(\varphi) = 0\\
            0 & val_b(\varphi) = 1\\
        \end{cases}    
    \end{equation}
    und 
    \begin{equation}
        val_b(\varphi \wedge \psi) := 
        \begin{cases}
            1 & val_b(\varphi) = val_b(\psi) = 1\\
            0 & \text{sonst}\\
        \end{cases}    
    \end{equation}
    Eine aussagenlogische Formel \(\varphi\) heißt \textbf{erfüllbar}, wenn es eine Belegung b gibt, sodass \(val_b = 1\) gilt.

\mysubsection{Definition (logisch äquivalent)}
    Zwei aussagenlogische Formel \(\psi\) und \(\varphi\) sind \textbf{logisch äquivalent} falls \(val_b(\varphi) = val_b(\psi) \quad \forall\) Belegungen b gilt.

\mysubsection{Definition (SAT)}
    Wir setzen 
    \[
        SAT := \{\varphi \in A : \varphi \text{ist in KNF und erfüllbar}\}  
    \]

\mysubsection{Bemerkung (Das Erfüllbarkeitsproblem SAT in der Komplexitätsklasse NP)*}
    Es gibt SAT \(\in\) \textbf{NP}.

\newpage

\mysubsection{Satz (Satz von Cock)}
    Die Sprache SAT ist \textbf{NP}-vollständig. 
    \begin{proof}
        Nach \hyperref[subsec:8.11]{Bemerkung 8.11} gilt SAT \(\in\) \textbf{NP}. Es genügt also zu zeigen, dass es \(\forall p \in poly\) und p-zeitbeschränkten TM M eine Funktion g \(\in\) \textbf{FP} gibt, sodass \(\forall w \in \{0, 1\}^*\) die aussagenlogische From g(w) ind KNF genau dann erfüllbar ist, wenn \(w \in L(M)\). Dafür wählen wir g(w) so, dass g(w) in wesentlich der Aussage u entspricht. Sei \(c \in \mathbb{N}_0\) und \(p: \mathbb{N} \to \mathbb{N}\), \(n \to n^c + c\) und sei \(M = (Q, \{0, 1\}, \Gamma, \Delta, s, F)\) eine p-zeitbeschränkte TM. \hyperref[subsec:7.10]{Satz 7.10} und \hyperref[subsec:7.4]{Satz 7.4} erlauben uns anzunehmen, dass M eine 1-TM mit Bandalphabet \(\Gamma = \{\Box, 0, 1\}\) ist. Sei \(w \in \{0, 1\}^*\) und \(n :=|w|\).
        \vspace*{0.5cm}
        \\
        Wir setzen \(I:= [p(n)]\) und \(J: \{-p(n)+1, \cdots, p(n)+1\}\). Wir definieren nun eine ganze Ansammlung an variablen, die die Arbeitsweise von M bei eingabe w codieren.
        \begin{center}
            \setlength{\extrarowheight}{3pt} % Adjust the length as needed
            \begin{tabular}{l|p{10cm}}
                Variable & Aussage \\
                \hline
                \hline
                \(T_{i, j, a}\) mit \(i \in I, j \in J, a \in \Gamma\) & Zum Zeitpunkt i ist a das Symbol auf Feld j \\
                \hline
                \(p_{i, j}\) mit \(i \in I, j \in J\) & Zum Zeitpunkt i steht der Kopf auf dem Feld j \\
                \hline
                \(S_{i, q}\) mit \(i \in I, q \in Q\) & Zum Zeitpunkt ist der Zustand q \\
                \hline
                \(D_{i, 0}\) & Der i-te 'Listeneintrag' ist eine Wiederholung einer vorangehenden Stoppkonfiguration. \\
                \hline
                \(D_{i, d}\) mit \(i \in I\) und \(d \in \Delta\) & Die Instruktion d bezeugt im Sinne \hyperref[subsec:2.3]{Definition 2.3(Nachfolgekonfiguration)}, dass der (i+1)-te 'Listeneintrag' Nachfolgekonfiguration des i-ten 'Listeneintrags ist' \\
            \end{tabular}    
        \end{center}
        \paragraph*{\(\leadsto\)}
            Wir haben eine Liste/ Logbuch mit den variablen \(D_{i, 0}\), \(D_{i, d}\), die im i-ten Feld beschreiben was M im i-ten Schritt der Rechnung macht.
        
        Es bleibt g(w) zu konstruieren, so dass \(val_b(g(w)) = 1\) für eine Belegung b gilt, wenn b eine Liste darstellt, die einer rechnung von M zur Eingabe w (gegebenenfalls mit Wiederholungen der Stoppkonfiguration) entspricht.
        \vspace*{0.5cm}
        \\
        Wir geben dazu g(w). Teilformeln die wir nicht als KNF angeben, lassen sich leicht in solche umwandeln. 
        \vspace*{0.5cm}
        \\
        Die Startkonfiguration wird korrekt realisiert:
        \[
            S_{1,S}, \quad P_{1, 1}, \quad \bigwedge \limits_{j\in [n]} T_{1,j, w(j)},\quad \bigwedge \limits_{J/[n]} T_{1, j, \Box}   
        \]
        Zu jedem zeitpunkt sind die relevanten Felder mit höchstens einem Symbol beschriftet:
        \[
            \bigwedge \limits_{(i, j, a, a') \in I \times J \times \Gamma \times \Gamma : a \not = a'} (T_{i, j, a} \to \lnot T_{i, j, a'})
        \]
        Zu jedem Zeitpunkt gibt es höchstens eine aktuelle Kopfposition.
        \[
            \bigwedge \limits_{(i, j, j') \in I \times J \times J : j \not = j'} (P_{i, j} \to \lnot P_{i, j'})      
        \]
        Zu jedem Zeitpunkt gibt es hächstens eine aktuellen Zustand
        \[
            \bigwedge \limits_{(i, q, q') \in I \times Q \times Q : q \not = q'} (S_{i, q} \to \lnot S_{i, q'})      
        \]
        Zu jedem Zeitunkt gibt es höchstens eine auszuführende Instruktion oder es wird die vorherige wieerholt:
        \[
            \bigwedge \limits_{(i, d, d') \in I \times \Delta^+ \times \Delta^+ : d \not = d'} (D_{i, d} \to \lnot D_{i, d'})      
        \]
        Zu jedem Zeitpunkt kann sich nur das Feld ändern auf dem der Kopf steht: 
        \[
            \bigwedge \limits_{(i, j, a) \in I^- \times J \times \Gamma} ((T_{i, j, a \wedge \lnot P_{i, j}}) \to T_{i+1, j, a})      
        \]
        Wird eine Stoppkonfiguration wiederholt, ändert sich das Band, Kopfposition und Zustand nicht:
        \[
            \bigwedge \limits_{(i, j, a) \in I^- \times J \times \Gamma} ((D_{i,0} \wedge T_{i, j, a}) \to T_{i+ 1, j, a} )
        \]
        \[
            \bigwedge \limits_{(i, j) \in I^- \times J } ((D_{i,0} \wedge P_{i, j}) \to P_{i+ 1, j} )
        \]
        \[
            \bigwedge \limits_{(i,q) \in I^- \times Q } ((D_{i,0} \wedge S_{i, q}) \to S_{i+ 1, q})
        \]
        Zu denem Zeitpunkt gibt es eine auszuführende Instruktion oder es wird die vorherige Konfiguration wiederholt:
        \[
            \bigwedge \limits_{i \in I} \bigvee \limits_{d \in \Delta^+} D_{i, d}
        \]
        Konfigurationen, die den Bedingungsteil einer Instruktion erfüllen werden nicht wiederholt:
        \[
            \bigwedge \limits_{(i, j(q, a, a', a', B)) \in I^- \times J \times \Delta} ((S_{i, q} \wedge P_{i, j} \wedge T_{i, j, a}) \to \lnot D_{i, 0})  
        \]
        Gibt es zu einer Konfiguration eine auszuführende Instruktion, so muss deren Bedingungsteil durch die Konfiguration erfüllt werden
        \[
            \bigwedge \limits_{(i, (q, a, q', a', B)) \in I^- \times \Delta} D_{i, (q, a, q',a',B)} \to S_{i,q}
        \]  
        \[
            \bigwedge \limits_{(i, j,(q, a, q', a', B)) \in I^- \times J \times  \Delta} (D_{i, (q, a, q',a',B) \wedge P_{i, j}} \to T_{i,j, a})
        \]  
        Wird eine Instruktion angewendet, so muss die Nachfolgekonfiguration aus dieser Instruktion aus der Vorgängerkonfiguration hervorgehen:
        \[
            \bigwedge \limits_{(i, (q, a, q', a', B)) \in I^- \times \Delta} (D_{i, (q, a, q',a',B)} \to S_{i + 1, q'})
        \]  
        \[
            \bigwedge \limits_{(i, j, (q, a, q', a', B)) \in I^- \times J \times \Delta} (D_{i, (q, a, q',a',B)} \wedge P_{i, j} \to T_{i + 1, j, a})
        \] 
        \[
            \bigwedge \limits_{(i, j, (q, a, q', a', B)) \in I^- \times J \times \Delta} (D_{i, (q, a, q',a',B)}\wedge P_{i, j} \to P_{i + 1, j, + \delta_B})
        \]  
        wobei \(\delta_L := -1\), \(\delta_S := 0\), \(\delta_R:= 1\).
        \vspace*{0.5cm}
        \\
        Es wird eine Stoppkonfiguration erreicht und zum Zeitpunkt p(n) ist der Zustand akzeptierend:
        \[
            D_{p(n), 0}, \quad \bigvee \limits_{q \in F} S_{p(n), q}.    
        \]
        Nach Konstrukton ist damit g(w) genau dann erfüllbar, wenn es eine endliche Rechnung von M zur Eingabe w gibt, die höchstens Länge p(n) hat, also genau, wenn \(w \in L(M)\) gilt.
    \end{proof}

\newpage

\mysubsection{Definition (k-konjunktive Normalform)}
    Für \(k \in \mathbb{N}\) ist eine aussagenlogische Formel in k-konjunktive Normalform, kurz k-KNF, falls sie eine Konjunktion disjunkiver Klauseln, die jeweils die Länge hächstens k haben, ist.

\mysubsection{Definition (k-SAT)}
    Für \(k \in \mathbb{N}\)
    \[
        k-SAT := \{ \varphi \in A : \varphi \text{ist im k-KNF und erfüllbar}\}     
    \]

\mysubsection{Bemerkung(k-SAT als Entscheidungsproblem in der Komplexitätsklasse NP)*}
    Für \(k \in \mathbb{N}\) gilt k-SAT \(\in\) \textbf{NP}.

\mysubsection{Satz (NP-Vollständigkeit von k-SAT für k \(\leq\) 3)*}
    Für \(k \leq 3\) ist k-SAT \textbf{NP}- vollständig.

    \[SKIZZE HIER\]

    \[
        (x \vee y )
    \]
    \[
        \downarrow
    \]

    \[
        (\frac{\quad x \quad }{} \vee a) \wedge (\lnot a \vee \frac{}{\quad y \quad})
    \]
    Nach \hyperref[subsec:8.15]{Bemerkung 8.15} genügt es SAT \( \leq^p_m\) 3-SAT. Wir müssen also eine beliebige aussagenlogische Formel \(\varphi\) in KNF in eine Formel in 3-KNF überführen, sodass die \(\varphi \in\) SAT \(\Leftrightarrow \varphi \in\) 3-SAT****. Sei \(\varphi \in\) SAT und wir konstruieren \(\varphi\)' in 3-SAT wie folgt:
    \vspace*{0.5cm}
    \\
    Wir geben für jede Klausel von \(\varphi\) eine Konjunktion von Klauseln für \(\varphi\) an, die logisch äquivalent ist. Sei \((l,v \cdots vl_r)\) eine Klausel von \(\varphi\). Sei \(r \leq 4\), sonst ist nichts zu tun.
    
    \[
        (l, va_2) \wedge (\bigwedge \limits_{j = 2}^{r = 1} (\lnot a_j \wedge l_j \wedge a_{j+1})) \wedge (\lnot a_r \wedge l_r)  \quad \circledast 
    \]
    Hier sind \(a_2, \cdots, a_r\) neue Variablen. 
    \vspace*{0.5cm}
    \\
    Wir fügen \(\circledast\) für \((l, v \cdots vl_r)\) zu \(\varphi'\) hinzu und verfahren für jede andere Klausel analog. Nun kann man überprüfem, dass \(\varphi\) und \(\varphi '\)

    \[SKIZZE HIER\]

\mysubsection{Definition (Graphen (V, E))*}
    Ein \textbf{Graph} ist ein Paar (V,E), wobei V eine endliche Menge ist, die \textbf{Echenmenge}, und \(E \subseteq 2^V\) eine Menge zweielementiger Mengen, die \textbf{Kantenmenge}, ist.

\mysubsection{Definition (k-Färbung)}
    Für \(k \in \mathbb{N}\) ist eine k-Färbung eines graphen (V,E) eine Funktion \(\phi : V \to [k]\), sodass \(\phi(u) \not = \phi(v) \quad \forall \{u, v\} \in E\) gilt.

\mysubsection{Definition (k-COLORING)}
    Es sei 
    \[
        \text{COLORING} := \{(G,k) : \text{es gibt eine k-färbung des Graphen G}\}
    \]
    und für \(k \in \mathbb{N}\) sei
    \[
        \text{k-COLORING} := \{G : \text{es gibt eine k-Färbung für den Graphen G}\}  
    \]

\mysubsection{Bemerkung(NP-Eigenschaft von COLORING und k-COLORING)*}
    Es gibt COLORING \(\in\) \textbf{NP} und für \(k \in \mathbb{N}\) gilt k-COLORING \(\in\) \textbf{NP}.

\mysubsection{Satz(NP-Vollständigkeit von 3-Coloring)*}
    Das Problem 3-Coloring ist \textbf{NP}-vollständig.
    \begin{proof}
        Nach \hyperref[subsec:8.20]{Bemerkung 8.20} und \hyperref[subsec:8.16]{Satz 8.16} genügt es 3-SAT \(\leq^P_m\) 3-COLORING zu zeigen. Wir müssen also eine aussagenlogische Formel \(\varphi\) in einen Graphen G transformieren \(\cdots\)

        \(\varphi\) in 3-KNF
    \end{proof}

\mysubsection{Korrolar (k-COLORING NP-vollständig für k \(\leq\) 3)*}
    Für \(k \leq 3\) ist k-COLORING \textbf{NP}-vollständig.
    \begin{proof}
        Übung
    \end{proof}

\mysubsection{Korrolar(COLORING NP-vollständig)*}
    COLORING ist \textbf{NP}-vollständig.

\mysubsection{Definition (EXACTCOVER)}
    Für eine Menge S von Mengen ist eine exakte Überdeckung von S eine Teilmenge T \(T \subseteq S\) paarweise disjunkive Menge mit \(\bigcup \limits_{A\in T} A = \bigcup \limits_{A \in S} A\). 
    \[
        \text{EXACTCVER} := \{S : \text{S ist eine endliche Menge endlicher Mengen mit einer exakten Überdeckung}\}  
    \]
\mysubsection{Bemerkung (EXACTCOVER NP Eigenschaft)*}
    EXACTCOVER \(\in\) \textbf{NP}

\mysubsection{Satz (EXACTCOVER NP-vollständig)*}
    EXACTCOVER ist \textbf{NP}-vollständig.
    \[\text{ohne beweis.}\]

\mysubsection{Definition (SUBSETSUM)}
    Sei
    \[
        \text{SUBSETSUM} := \{(a_1, \cdots, a_n, b) \in \mathbb{N}_0^{n+1} : \exists x_1, \cdots x_n \in \{0,1\} : \sum_{i}^{} x_i a_i = b\}    
    \]

\mysubsection{Bemerkung(SUBSETSUM NP Eigenschaft)*}
    SUBSETSUM \(\in\) \textbf{NP}

\mysubsection{Satz(SUBSETSUM NP-vollständig)*}
    SUBSETSUM ist \textbf{NP}-vollständig.
    \[\text{ohne beweis.}\]
%\section{Bilder}

\includegraphics[width=0.5\textwidth]{Turingmachine/turing.png}
\includegraphics[width=0.5\textwidth]{Berechenbarkeit/predic.png}
\includegraphics[width=0.5\textwidth]{Automaten/finit1.png}
\includegraphics[width=0.5\textwidth]{ReguläreSprachen/regul.png}
%\includegraphics{}



\end{document}