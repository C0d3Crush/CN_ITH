\documentclass[a4paper,11pt]{article}

%\setlength{\headheight}{22.62503pt}

% Use packages to set margins, fonts, and spacing
\usepackage[margin=2.5cm,headheight=22.28003pt,top=2.5cm]{geometry}
\usepackage{amssymb}
\usepackage{mathptmx}
\usepackage{setspace}
\usepackage{amsmath}
\usepackage{mathptmx}
\usepackage{graphicx} % for graphics
\usepackage{lipsum} % for sample text
\usepackage{amssymb}
\usepackage{xcolor}
\usepackage{tikz}
\usepackage{dsfont}
\usepackage{amssymb}
\usepackage{tikz}
\usepackage{caption}
\usetikzlibrary{tikzmark}



%\usepackage{fontspec}
%\setmainfont{TeX-Gyre-Schola/texgyreschola-regular.otf}
\newcommand{\coversection}[3]{
  \newpage
  \thispagestyle{empty}
  \begin{center}
    \vspace*{\fill}
    \includegraphics[width=0.5\linewidth]{#1}
    \vspace*{0.5cm} % Anpassung des Abstands
    \par
    \Large\textbf{#2}
    \par\vspace{0.5cm} % Anpassung des Abstands
    \begin{quote}
      \itshape\small\raggedleft #3
    \end{quote}
    \par\vspace{\fill}
  \end{center}
  \newpage
}

\begin{document}
\coversection{predic.png}{Berechenbarkeit}{Predictability is like knowing the path a river takes. The river starts at its source and flows down to the sea. Along the way, it may turn, twist, and divide, but it always follows the path of least resistance due to gravity. Knowing the terrain allows us to predict where the river will go.\\ \hspace*{\fill} - ChatGPT}

\section{Berechenbarkeit} \textbf{Konvention: } Sprechen wir von einer $e \in \mathbb{N}_0$ oder $(e_1, \cdots, e_n) \in \mathbb{N}_0^n$ wobei $n \in \mathbb{N}$ als Eingabe für eine TM oder Ausgabe einer TM, so bedetet dies, dass die Eingabe bzw. Ausgabe $bin(e)$ bzw $(bin(e_1), \cdots, bin(e_n))$ ist. Dies erlaubt es über partiell berechnenbare Funktionene $\Phi: \mathbb{N}_0^n \leadsto \mathbb{N}_0$ wobei $n\in \mathbb{N}$ zu sprechen und $L \subseteq \mathbb{N}_0$ als Sprache über $\{0, 1\}$ aufzufassen.

\subsection{Definition (Code)} Wir betrachten die Funktion code (mit geeignetem Definitionsbereich) und Zielmenge $\{0, 1\}^*$, für die folgendes gilt. Zunächst gelte \[code(L) = 10 \qquad code (S) = 00 \qquad code(R) = 01\] Für eine Instruktion $ I = (q, a, q', a', B) \in \mathbb{N}_0 \times \{0, 1\} \to \mathbb{N}_0 \times \{0, 1\} \times \{L, S, R\}$ einer normierten TM sei \[code (I) = 0^{|bin(q)|} 1 bin (q) a 0^{|bin(q')|} 1 bin (q') a' code (B)\] Für eine endliche Menge $ \Delta \subseteq \mathbb{N}_0 \times \{0, 1\} \to \mathbb{N}_0 \times \{0, 1\}\times \{L, S, R\}$ von Instruktionen einer normierten TM und $i \in [|\Delta|]$ sein $code_i(\Delta)$ dann ein längenlexikographische Ordnung i-te Wort in $\{code(I): I \in \Delta\}$ und sei \[ code (\Delta) = code_1(\Delta), \cdots, code_{|\Delta|}(\Delta)\] Für eine normierte TM $M = (\{0, \cdots, n\}, \{0, 1\}, \{\Box, 0, 1\}, \Delta, 0, \{0\})$ sei \[ code (M) = 0^{|bin(n)|} 1 bin (n) code (\Delta)\] der \textbf{Code} von $M$. Relevant ist hierbei dass es eine geeignete effektive Codierung von Turingmachinen durch Binärwörter gibt, so dass folgendes gilt 
\begin{itemize}
  \item Jede normierte TM hat einen Code 
  \item Keine zwei verschiedene normierten TMs haben den gleichen Code.
  \item Die Sprache der Codes von Turingmachinen ist entscheidbar
  \item Codes können eine geeignete Repräsentation der durch sie codierten TMs umgewandelt werden, die es insbesondere erlauben die codierten TMs effekiv zu simulieren.
  \item geignete Repräsentationen von TMs können effektiv in ihre Codes umgewandet werden.
\end{itemize}

\subsection{Definition (standardaufzählung)} Sei $\hat{w_0}, \hat{w_1}, \cdots$ die Aufzählung aller Codes normierter TMs in längenlexikographischer Ordnung. Für $e \in \mathbb{N}_0$ sei $M_e$ die durch $\hat{w_e}$ codierte TM und für $n \in \mathbb{N}$ sei $\Phi_e^n : \mathbb{N}_0^n \rightarrow \mathbb{N}_0$ die von $M_e$ berechnete n-äre partielle Funktion. Für $n \in \mathbb{N}$ heißt die Folge $(\Phi_e^n)_e\in \mathbb{N}$ \textbf{standardaufzählung} der n-ären partiell berechenbaren Funktion. Für $n \in \mathbb{N} $ und eine partiell berechenbare n-äre Funktion $\varphi: \mathbb{N}_0^n \rightarrow \mathbb{N}_0$ heißt jede zahl $e \in \mathbb{N}_0$ mit $\Phi_e^n = \varphi$ \textbf{Index} von $\varphi$.

\subsection*{Konvention: } Ergibt sich n aus dem Kontext, so schreiben wir auch $\Phi_e$ statt $\Phi_e^(n)$

\subsection{Bemerkung} Für $n \in \mathbb{N}$ und eine partielle berechnbare n-äre partielle Funktion $\Phi : \mathbb{N}_0^n \rightarrow \mathbb{N}_0$ gibt es unendlich viele Indizes von $\varphi$.

\subsection{Definition (U)} Es bezeichnet U die normierte TM, die bei Eingabe $(e, x_1, \cdots, x_n) \in \mathbb{N}_0^{n+1}$ wobei $n \in \mathbb{N}$ die normierte TM $\mathcal{M}_e$ bei Eingabe $(x_1, \cdots, x_n)$ simuliert und falls diese terminiert die Ausgabe der Simulierten ausgibt.

\subsection{Definition (Universell)} Eine DTM U heißt \textbf{Universell}, wenn es für alle $n \in \mathbb{N}$ und alle partiell berechenbaren Funktionen $\varphi : \mathbb{N}_0^n \leadsto \mathbb{N}_0$ eine $e \in \mathbb{N}$, so dass \[U(e, x_1, \cdots, x_n) = \varphi(x_1, \cdots, x_n)\] $\forall x_1, \cdots, x_n \in \mathbb{N}_0$ gilt.

\subsection{Bemerkung} Die TM U ist universell, denn für $e \in \mathbb{N}_0$, $n \in \mathbb{N}$ und $x_1, \cdots, x_n \in \mathbb{N}_0$ gilt \[U(e,x_1, \cdots, x_n) = \Phi_e(x_1, \cdots, x_n)\]\\ \[(x, y) \mapsto x^y\] \[y \mapsto 2^y\] \[(x_1, \cdots, x_m, y_1, \cdots, y_n) \mapsto \varphi(x_1, \cdots, x_m, y_1, \cdots, y_n) partiell berechenbar\] \[\leadsto (y_1,\cdots, y_m) \mapsto \varphi(x_1, \cdots, x_m, y_1, \cdots, y_n) partiell berechenbar \]

\subsection{Satz ($s_n^m$ - Theorem)} $\forall m, n \in \mathbb{N}$ existiert eine berechenbare Funktion $s_n^m : \mathbb{N}_0^{m+1} \to \mathbb{N}_0$ mit \[\Phi_e^{m+1}(x_1\cdots, x_m, y_1, \cdots, y_n) = \Phi_{s_n^m(e, x_1, \cdots, x_m)}^n (y_1, \cdots, y_n)\] $\forall e, x_1, \cdots, x_m, y_1, \cdots, y_n \in \mathbb{N}_0$

\subsubsection*{Beweis: } Fixiere $m \in \mathbb{N}$. Betrachte die DTM S , die bei Eingabe $(e, x_1, \cdots, x_m) \in \mathbb{N}_0^{m+1}$ wie folgt verfährt??.
\begin{itemize}
  \item Zunächst bestimmt S den Code von $\mathcal{M}_e$ 
  \item der Code von $\mathcal{M_e}$wird dann in einen Code einer normierten TM $\mathcal{M}$ umgewandet, die zunächst $x_1\Box\cdots\Box x_m\Box$ neben die Eingabe schreibt, dan den Kopf auf das erste Feld des beschriebenen Bandteilsbewegt und dann wie $\mathcal{M_e}$ arbeitet.
  \item Es wird bestimmt an welcher Stelle der Standardaufzählung der Code von auftaucht und diese Stelle wird ausgegeben.
\end{itemize}
Sei $s_n^m$ die von S berechnete $(m+1)$-äre partielle Funktion. Dann ist $s_n^m$ eine Funktion wie gewünscht. Es gibt überabzählbar viele Binärsprachen, denn: Betrachte Aufzählung von Binärsprachen $L_1, L_2, \cdots$ 
\begin{table}[ht]
  \centering
  \renewcommand{\arraystretch}{2} % Adjust the value to increase or decrease the cell size
  \begin{tabular}{c c c c c}
    \tikzmarknode{L0-0}{$\mathds{1}_{L_0}(0)$} & $\mathds{1}_{L_0}(1)$ & $\mathds{1}_{L_0}(2)$ & $\mathds{1}_{L_0}(3)$ \\
    $\mathds{1}_{L_1}(0)$ & $\mathds{1}_{L_1}(1)$ & $\mathds{1}_{L_1}(2)$ & $\mathds{1}_{L_1}(3)$ \\
    $\mathds{1}_{L_2}(0)$ & $\mathds{1}_{L_2}(1)$ & \tikzmarknode{L2-2}{$\mathds{1}_{L_2}(2)$} & $\mathds{1}_{L_2}(3)$ \\
  \end{tabular}
  \captionsetup{labelformat=empty, justification=centering, skip=10pt}
  \caption{Standardaufzählung}
\end{table}

\begin{tikzpicture}[overlay, remember picture, blue, >=stealth]
  \draw [->, thick] ([yshift=1ex]L0-0.south) -- ([yshift=-1ex]L2-2.north);
\end{tikzpicture}

$L$ mit $\mathds{1}_L(i)$ = 
$\begin{cases}
    0, & \text{wenn } \mathds{1}_{L_i}(i) = 1 \\
    1, & \text{wenn } \mathds{1}_{L_i}(i) = 0 \\
\end{cases}$

\subsection{Definition (diagonales Halteproblem)} Die Menge $H_diag := \{e \in \mathbb{N}_0 : \Phi_e (e) \downarrow\}$ heißt \textbf{diagonales Halteproblem}.

\subsection{Proposition} Das diagonale Halteproblem ist rekursiv aufzählbar.
\subsubsection*{Beweis: }Die DTM, die bei Eingabe $e \in \mathbb{N}_0$ wie $U$ bei Eingabe $(e, e)$ arbeitet, aber bei terminieren $1$ statt der Ausgabe von $U$ ausgibt berechnet die partielle charachteristische Funktion von $H_{diag}$. Die partielle Funktion $x_{H_{diag}}$ ist also partiell berechenbar. Die partielle Funktion $x_{H_{diag}^c}$ ist nicht partiell berechenbar, dann: Betrachte Standardaufzählung

\begin{table}[ht]
  \centering
  \renewcommand{\arraystretch}{2} % Adjust the value to increase or decrease the cell size
  \begin{tabular}{c c c c c}
    \tikzmarknode{L0-0}{$\Phi_{L_0}(0)$} & $\Phi_{L_0}(1)$ & $\Phi_{L_0}(2)$ & $\Phi_{L_0}(3)$ \\
    $\Phi_{L_1}(0)$ & $\Phi_{L_1}(1)$ & $\Phi_{L_1}(2)$ & $\Phi_{L_1}(3)$ \\
    $\Phi_{L_2}(0)$ & $\Phi_{L_2}(1)$ & \tikzmarknode{L2-2}{$\Phi_{L_2}(2)$} & $\Phi_{L_2}(3)$ \\
  \end{tabular}
  \captionsetup{labelformat=empty, justification=centering, skip=10pt}
  \caption{Standardaufzählung}
\end{table}

\begin{tikzpicture}[overlay, remember picture, blue, >=stealth]
  \draw [->, thick] ([yshift=1ex]L0-0.south) -- ([yshift=-1ex]L2-2.north);
\end{tikzpicture}

\end{document}
