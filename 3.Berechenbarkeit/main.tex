\documentclass[a4paper,11pt]{article}

%\setlength{\headheight}{22.62503pt}

% Use packages to set margins, fonts, and spacing
\usepackage[margin=2.5cm,headheight=22.28003pt,top=2.5cm]{geometry}
\usepackage{amssymb}
\usepackage{mathptmx}
\usepackage{setspace}
\usepackage{amsmath}
\usepackage{mathptmx}
\usepackage{graphicx} % for graphics
\usepackage{lipsum} % for sample text

%\usepackage{fontspec}
%\setmainfont{TeX-Gyre-Schola/texgyreschola-regular.otf}
\newcommand{\coversection}[3]{
  \newpage
  \thispagestyle{empty}
  \begin{center}
    \vspace*{\fill}
    \includegraphics[width=0.5\linewidth]{#1}
    \vspace*{0.5cm} % Anpassung des Abstands
    \par
    \Large\textbf{#2}
    \par\vspace{0.5cm} % Anpassung des Abstands
    \begin{quote}
      \itshape\small\raggedleft #3
    \end{quote}
    \par\vspace{\fill}
  \end{center}
  \newpage
}

\begin{document}
\coversection{predic.png}{Berechenbarkeit}{Predictability is like knowing the path a river takes. The river starts at its source and flows down to the sea. Along the way, it may turn, twist, and divide, but it always follows the path of least resistance due to gravity. Knowing the terrain allows us to predict where the river will go.\\ \hspace*{\fill} - ChatGPT}

\section{Berechenbarkeit} \textbf{Konvention: } Sprechen wir von einer $e \in \mathbb{N}_0$ oder $(e_1, \cdots, e_n) \in \mathbb{N}_0^n$ wobei $n \in \mathbb{N}$ als Eingabe für eine TM oder Ausgabe einer TM, so bedetet dies, dass die Eingabe bzw. Ausgabe $bin(e)$ bzw $(bin(e_1), \cdots, bin(e_n))$ ist. Dies erlaubt es über partiell berechnenbare Funktionene $\Phi: \mathbb{N}_0^n \leadsto \mathbb{N}_0$ wobei $n\in N$ zu sprechen und $L \subseteq \mathbb{N}_0$ als Sprache über $\{0, 1\}$ aufzufassen.

\subsection{Definition (Code)} Wir betrachten die Funktion code (mit geeignetem Definitionsbereich) und Zielmenge $\{0, 1\}^*$, für die folgendes gilt. Zunächst gelte \[code(L) = 10 \qquad code (S) = 00 \qquad code(R) = 01\] Für eine Instruktion \[ I = (q, a, q', a', B) \in \mathbb{N}_0 \times \{0, 1\} \times \mathbb{N}_0 \times \{0, 1\} \times \{L, S, R\}\] einer normierten TM sei \[code (I) = 0^{|bin(q)|} 1 bin (q) a 0^{|bin(q')|} 1 bin (q') a' code (B)\] Für eine endlcihe Menge \[ \Delta \subseteq \mathbb{N}_0 \times \{0, 1\} \times \mathbb{N}_0 \times \{0, 1\}\times \{L, S, R\}\] von Instruktionen einer normierten TM und $i \in [|\Delta|]$ sein $code_i(\Delta)$ dan ein längenlexikographische Ordnung i-te Wort in $\{code(I): I \in \Delta\}$ und sei \[ code (\Delta) = code_1(\Delta), \cdots, code_{|\Delta|}(\Delta)\] Für eine normierte TM $M = (\{0, \cdots, n\}, \{0, 1\}, \{\Box, 0, 1\}, \Delta, 0, \{0\})$ sei \[ code (M) = 0^{|bin(n)|} 1 bin (n) code (\Delta)\] der \textbf{Code} von M. Relevant ist hierbei dass es eine geeignete effekttive Codierung von Turingmachinen durch Binärwörter gibt, so dass folgendes gilt 
\begin{itemize}
  \item Jede normierte TM hat einen Code 
  \item Keine zwei verschiedene normierten TMs haben den gleichen Code.
  \item Die Sorache der Codes von Turingmachinen ist entscheidbar
  \item Codes können ein geeignete Repräsentation der durch sie codierten TMs umgewandelt werden, die es insbesondere erlauben die codierten TMs effekiv zu simulieren.
  \item geignete Repräsentationen von TMs können effektiv in ihre Codes umgewandet werden.
\end{itemize}

\subsection{Definition (standardaufzählung)} Sei $\hat{w_0}, \hat{w_1}, \cdots$ die Aufzählung aller Codes normierter TMs in längenlexikographischer Ordnung. Für $e \in \mathbb{N}_0$ sei $M_e$ die durch $\hat{w_e}$ codierte TM und für $n \in \mathbb{N}$ sei $\Phi_e^n : \mathbb{N}_0^n \rightarrow \mathbb{N}_0$ die von $M_e$ berechnete n-äre partielle Funktion. Für $n \in \mathbb{N}$ heißt die Folge $(\Phi_e^n)_e\in \mathbb{N}$ \textbf{standardaufzählung} der n-ären partiell berechenbaren Funktion. Für $n \in \mathbb{N} $ und eine partiell berechenbare n-äre Funktion $\varphi: \mathbb{N}_0^n \rightarrow \mathbb{N}_0$ heißt jede zahl $e \in \mathbb{N}_0$ mit $\Phi_e^n = \varphi$ \textbf{Index} von $\varphi$.
\end{document}
