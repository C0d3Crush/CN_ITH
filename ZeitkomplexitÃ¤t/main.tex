\coversection{Automaten/finit1.png}{Zeitkomplexität}{\\ \hspace*{\fill} - ChatGPT}
\paragraph*{Komplexität} 
    Einschrkung von TM bezüglich Zeit (Anzahl Rechenschritte einer TM) und Platz (Anzahl besuchte Felder einer TM)
    \begin{itemize}
        \item Worst case
        \item Average case
        \item Best case
    \end{itemize}

\mysubsection{Definition (Rechenzeit)}
    Sei \(M = (Q, \Sigma, \Gamma \Delta, s, F)\) eine DTM. Es bezeichne \(time_M:\Sigma^* \to \mathbb{N}_0\) die partielle Funktion mit \(dom(time_M) = dom (\varphi_M)\), so dass \(time_M(w)\) für alle \(w \in dom (\varphi_M)\) die länge der Rechnung von M zur Eingabe w ist. Die \textbf{rechenzeit} von M zur Eingabe \(w \in dom (\varphi_M)\) ist \(time_M(w)\). \\
    Da im Folgenden nahezu alle Aussagen nur im Hinblick auf die Betrachtung Eingabelänge \(\to \infty\) sinnvoll sind, lassen wir in der nächsten Definition \textbf{endlich} viele Ausnahmen zu.

\mysubsection{Definition (zeitbeschränkt)}
    Eine \textbf{Zeitschranke} ist eine berechenabre Funktion \(t: \mathbb{N}_0 \to \mathbb{N}_0\) mit \(t(n)\leq n \quad \forall n \in \mathbb{N}_0\).\\ Sei \(t: \mathbb{N}\to \mathbb{R}_{\leq 0}\) eine Funktion. Eine TM M ist \textbf{t-zeitbeschränkt}, wenn M total ist und es ein \(n_0 \in \mathbb{N}_0\) gibt, so dass \(\forall w \in \Sigma^{\leq n_0}\) alle Rechnungen von M zur Eingabe w höchstens Länge \(t(|w|)\) haben.\vspace{0.5cm}\\Ist n ein Variablensymbol und t ein Term, der eine Funktion \(f: n \mapsto t\) festlegt, so verwendet man auch die Zeichnung t-zeitbeschränkt an Stelle von f-zeitbeschränkt. Man spricht also zum Beispiel von \(n^2\)-zeitbeschränkt der \(3x^3\)-zeitbeschränkt oder auch von \(f(n)\)-zeitbeschränkt für eine Funktion \(f: \mathbb{N}_0 \to \mathbb{N}_0\). Betrachten wie eine Funktion \(t: \mathbb{N}_0 \to \mathbb{R}_{\leq 0}\) als Zeitschranke, so ist damit die Fukntion \(t': \mathbb{N}_0 \to \mathbb{N}_0\), \(n \mapsto \lfloor t(n)\rfloor\) gemeint.

\mysubsection{Satz (lineare Beschleunigung)}
    Sei \(\tilde{c} > 1\), \(\epsilon > 0\) und \(t: \mathbb{N}_0 \to \mathbb{R}_{\leq 0}\) mit \(t(n) \leq (1+ \epsilon)n \quad \forall n \in \mathbb{N}_0\). Ist M eine \(\tilde{c}\cdot t(n)\)-zeitbeschränkt k-DTM, \(k \leq 2\), so gibt es auch eine \(t(n)\)-zeitbeschränkte k-TM M' mit L(M) = L(M').
    \paragraph*{Beweisskizze:}    
        Die Idee ist es sei \(c \leq 2\) wir wollen c Felder in einen Vektor der Länge c zusammenfassen und damit das Bandalphabet vergrößern und andererseits ungefähr "c Schritte auf einmal machen".
        Mit einem neuen Bandalphabet 
        \[
            ((\Gamma \cup \{\underline{a}: a \in \Gamma\})^c/\{\Box\}^c) \cup \{\Box\} \cup \Sigma
        \]
        dabei übernimmt \(\Box\) die Rolle von \(\{\Box\}^c\) und die Markierung ist für den Kopf der urspünglichen TM. 
        Konstruktion einer neuen TM M', die schneller ist als eine gegebene TM M.
        \begin{itemize}
            \item [1.Schritt:] Transformation der Eingabe in die "komprimierte"\ Form auf dem 2.Band (|w|+1 Schritte)
            \item [2.Schritt:] Simulation von M bandweise (1. und 2. Band vertauscht). Wir benutzen Vektoren der Länge c um c Felder von M darzusetellen. Zeichen mit einer Markierung \_ , markieren den Köpfen den simulierten TM M.
            \item [3.Schritt:] Die TM M' speichert in ihren Zuständen die Beiden Felder rechts und links des Feldes wo die simulierten Köpfe von M sind. 
            \begin{itemize}
                \item Simulation von c Schritten von M.
                \item Update der Felder, die von M verändert wurden 6 Schritte notwendig
            \end{itemize}
        \end{itemize}
        Insgesamt macht M' also \(|w|+ 1 + 6\cdot \frac{\tilde{c}t(|w|)}{c} + 10 \approx \frac{6\tilde{c}}{c} t(|w|) = t(|w|)\) für \(c := 6\tilde{c}\). In gewissem Sinne ist die obige Konstruktion invers zum \hyperref[subsec:2.18]{Alphabetwechel des bandalphabets bei der Konstruktion normierter TM}. Entsprechend steigt die Laufzeit auch nur um einen konstanten Faktor wenn wir die TM normieren 

\mysubsection{Satz(Alphabetwechel)}
    Ist M eine \(t(n)\)- zeitbeschränkte k-TM M mit Eingabealphabet \{0, 1\}, so gilt es eine Konstante \(c \leq 1\) und eine \(ct(n)\)-zeitbeschränkte k-TM M' mit Eingabealphabet \{0, 1\}, Bandalphabet \(\{\Box, 0, 1\}\) und L(M') = L(M).
    \medskip
    \(\leadsto\) Konstanten spielen nur eine untergeordnete Rolle für uns.

\mysubsection{Definition(Landau-Symbole)*}
    Für eine Funktion \(f: \mathbb{N}_0 \to \mathbb{R}_{\leq 0}\) sei
    \begin{itemize}
        \item [] \(o(f) = \{g: \mathbb{N}_0 \to \mathbb{R}_{\leq 0} : \forall \epsilon > 0 : \exists n_0 \in \mathbb{N}_0 : \forall n \geq n_0 : g(n) \leq \epsilon f(n)\}\)
        \item [] \(O(f) = \{g: \mathbb{N}_0 \to \mathbb{R}_{\leq 0} : \exists c > 0 : \exists n_0 \in \mathbb{N}_0 : \forall n \geq n_0 : g(n) \leq c f(n)\}\)
        \item [] \(\theta(f) = \{g: \mathbb{N}_0 \to \mathbb{R}_{\leq 0} : \exists c, \epsilon > 0 : \exists n_0 \in \mathbb{N}_0 : \forall n > n_0 : \epsilon f(n) \leq g(n) \leq cf(n)\}\)
        \item [] \(\Omega (f) = \{g: \mathbb{N}_0 \to \mathbb{R}_{\leq 0} : \exists \epsilon > 0 : \exists n_0 \in \mathbb{N}_0 : \forall n \geq n_0 : g(n) \geq \epsilon f(n)\}\)
        \item [] \(w(f) = \{g: \mathbb{N}_0 \to \mathbb{R}_{\leq 0} : \forall c > 0 : \exists n_0 \in \mathbb{N}_0 : \forall n \geq n_0 : g(n) \leq c f(n)\}\)
    \end{itemize}

\mysubsection{Bemerkung(Eigenschaften Landau-Symbole)*}
    Seien \(f, g : \mathbb{N}_0 \to \mathbb{R}_{\leq 0}\) Funktion dann gilt:
    \begin{itemize}
        \item [(i)] \(g \in o(f) \Leftrightarrow f \in w(g)\)
        \item [(ii)] \(g \in O(f) \Leftrightarrow f \in \Omega(g)\)
        \item [(iii)] \(g \in \theta(f) \Leftrightarrow ( g \in O(f) \text{ und }g \in \Omega(f))\)
    \end{itemize}

\mysubsection{Bemerkung(Eigenschaften Landau-Symbole)*}
    Seien \(f, g : \mathbb{N}_0 \to \mathbb{R}_+\) Funktion dann gilt: 
    \begin{itemize}
        \item [(i)] \(g \in o(f) \Leftrightarrow \underset{n \to \infty}{\lim} \frac{g(n)}{f(n)} = 0\)
        \item [(ii)] \(g \in O(f) \Leftrightarrow \underset{n \to \infty}{\lim sub} \frac{g(n)}{f(n)} < \infty\)
        \item [(iii)] \(g \in \Omega(f) \Leftrightarrow \underset{n \to \infty}{\lim inf} \frac{g(n)}{f(n)} > 0\)
        \item [(iv)]\(g \in w(f) \Leftrightarrow \underset{n \to \infty}{\lim} \frac{g(n)}{f(n)} = \infty\)
    \end{itemize}
    Bei diesen Klassen ist es üblich \("\ = "\) statt \(" \in "\) zu verwenden, also beispielsweise \(g = O(f)\) statt \(g \in O(f)\) zu schreiben.
    \medskip
    Unsere Konstruktion für lineare Beschleunigungbedurfte mehrere Bänder. Tatsächlich kann es einen Unterschied machen ob ein oder mehrere Bänder zur Verfügung stehen. Als Beispiel betrachten wir Binärpalindrome:\\ Für \(w \in \{0, 1\}^*\). Sei \(w^R = w(|w| \cdots w(1))\). Die Sprache der Binärpalindrome ist folglich \(\{w \in \{0, 1\}^* : w = w^R \}\).

\mysubsection{Proposition(Binärpalindrome erkennen)*}
    Es gibt eine \(t(n)\)- zeitbeschränkte DTM M mit \(L(M) = \{w \in \{0, 1\}^* : w = w^R\}\) und \(t(n) = O(n)\).
    \begin{proof}
        Eine 2-DTM M die die Eingabe von links nach rechts durchläuft und dabei die Eingabe auf das 2.Band kopiert, dann auf dem ersten Band den Kopf zurück auf die Ausgangsposition bringt, dann gleichzeitig die Eingabe vpn links nach rehcts und die Kopie symbolweise vergeicht und genau dann akzeptiert, wenn alle Vergleiche positiv waren, dann erkennt M die Binärpalindrome.
    \end{proof}

\mysubsection{Satz (Zeitbeschränkung einer Turingmaschine)*}
    Ist M eine \(t(n)\)-zeitbeschränkte 1-DTM mit \(L(M) = \{w \in \{0, 1\}^* : w = w^R\}\), so ist \(t(n) = \Omega (n^2)\)
    \begin{proof}
        Sei M eine \(t(n)\)-zeitbeschränkter 1-DTM mit Eingabealphabet \{0, 1\}, \(L := \{ww^R : w \in \{0, 1\}^*\} \subseteq L(M)\) und \(t(n) \not = \Omega (n^2)\). Wir zeigen \(L \not = L(M)\). Aus der Existenz von M folgt die Existenz einer \((2t(n) + 1)\)-zeitbeschränkte TM M' mit Zustandsmenge Q wobei \(|Q \leq 2|\), Eingabealphabet \{0, 1\} und L(M) = L(M'), so dass der Kopd von M' am Ende jeder Rechnung auf dem Feld steht auf dem er am Anfang stand. Wir betrachten die Felder auf dem Band von M' als beginnend mit dem Feld auf dem der Kopf zu Rechungsbeginn steht mit dem natürlichen Zahlen 1, 2, 3, ...\\ Für jede Rechung R von M' sie \(C_i (R)\) die Folge \(q_1, \cdots, q_l\) von Zuständen von M', so dass M' während der Rechung R insgesamt l und die Grenze zwischen i und i+1 überschritt und so dass \(q_j\) für \(j \in [l]\) der Zustand ist, in dem sich M' unmittelbar vor dem Auführen der Anweisungen befindet, die das j-te übertreten auslöst. Die Folgen \(C_i(R)\) werden Crossing-Sequenz von M' gennant.\\ Die wesentliche Eigenschaft dieser Crossing-Sequenzen, die wir benutzen, ist wie folgt:\\ Ist R eine akzeptierte Rechung von M' zur Eingabe u und S eine Rechung zur Eingabe v und sind \(i \in [|u|]\) und \(j \in [|v|]\) mit \(C_i(R) = C_j(S)\), so gilt 
        \[
            u(1) \cdots u(i) v(j+1) \cdots v(|v|) \in L(M')   
        \]
        Wir nutzen die kurze laufzeit von M' nun mittels Schubfachprinzip gleiche Crossing-Sequenzen zu Rechnungen von unterschiedlichen Eingaben zu finden, um zu folgern, dass M' ein Wort akzeptiert, das nicht in L liegt. \\ Sei \(\epsilon := \frac{1}{300 \log_2 |Q|}\). Aus \(t(n) \not = \Omega(n^2)\) folgt die Existenz eines und mit \(\epsilon n \leq 1\). Sei \(n' := \frac{n}{4}\).\\ Für jedes Wort \(L' := \{w' 0^{\frac{n}{2} w'^R : w' \in \{0, 1\}^{=\frac{n}{4}}}\} \subseteq L\) betrachten wir die kürzeste Crossing-Sequenz \(C_w \in \{C_{n'}(R), \cdots, C_{n-n'}(R)\}\) einer akzeptierten Rechung R von M' zur Engabe w. Für jede Rechung von M' zu einer Eingabe \(w \in \{0, 1\}^{= n}\) gilt 
        \[
            \sum\limits_{i=n'}^{n-n'}|C_i(R)| \leq 2t(n)+1
        \]
        wobei |C| die länge von C bezeichnet. Für jedes Wort \(w \in \{0, 1\}^{=n}\) folgt damit 
        \[
            |C_w| \leq \frac{2t(n)+1}{n-2n' + 1} \leq \frac{4cn^2}{\frac{n}{2} + 1} \leq 16\epsilon u    
        \]
        Die Anzahl der Crossing-Sequenzen von M' der länge \(16\epsilon n -1\) ist höchstens 
        \[
            (|Q| + 1)^{16\epsilon n} \leq \frac{1}{2} (2|Q|)^{16 \epsilon n} \leq \frac{1}{2} |Q|^{32 \epsilon n} \leq 2^{n'-1} 
        \]
        \(mit \Rightarrow |Q| = 2^{\log_2 |Q|}\) \\ Die Anzahl der Wörter in L' ist aber \(2^{n'}\). Es gibt also voneinander verschiedene Wörter in L' ist und \(i \in [|u|]\) und \(j \in [|v|]\) mit \(C_i(R) = C_j(S)\) wobei R die Rechung von u ist und S die Rechung von v in M'. Es gibt also einen Präfix u' von u mit \(|u'| \leq n'\) und einen Suffix v' von v mit \(|v'| \leq n'\), sodass \(u' v' \in L(M') = L(M)\) gilt. Da u und v verschieden sind gilt \(u' v' \not \in L\). 
        \medskip
        Im Sinne von \hyperref[subsec:7.9]{Satz 7.9} macht es zwar einen Unterschied ob TM mit einem oder mit mehreren Bändern betrachtet werden, der Anstieg der Laufzeit ist bei Simulation mehrerer Bänder auf einem band mittels Spurentechnik aber nur quadratisch.
    \end{proof}

\mysubsection{Satz(Reduktion auf 1-Band-TM)*}
    Ist M eine \(t(n)\)- zeitbeschränkte TM, so gibt es eine \(t'(n)\)- zeitbeschränkte 1-TM mit \(t'(n) = O((t(n))^2)\), die dieselbe Sprache erkennt. Während der laufzeitanstieg bei der Reduktion auf ein Band quadratisch sein kann, ist die Reduktion von k auf zwei Bänder effizienter möglich 

\mysubsection{Satz(Codekonstruktion und Simulation)*}
    Ist für alle deterministischen TM \(M = (\{0, \cdots, n\}, \{0, 1\}, \{0, 1, \Box\}, \Delta, 0, \{0\})\) und für alle Wörter \(w \in \{0, 1\}^*\) das Wort code (M, w) ein geeigneter Codefür (M, w), so gibt es eine 2-Dtm N, so dass folgendes gilt:

    \begin{itemize}
        \item [(i)] \(\forall\) DTM U wie oben und \(\forall w \in \{0, 1\}^*\) akzeptiert U den Code code(M, w) genau dann, wenn M das Binärwort w akzeptiert.
        \item [(ii)] Für alle zeitschranken t, alle t-zeitbeschränkten DTM M wie oben, gibt es ein \(n_0\), sodass \(\forall w \in \{0, 1\}^{\leq n_0}\) eine Konstante \(c \in \mathbb{N}\) gibt, sodass die Rechnung von U zur Eingabe code(M, w) Länge höchstens \(ct(|w|)\log(t(|w|))\) hat.
    \end{itemize}

    \begin{proof}
        \textbf{Idee: }
            \begin{enumerate}
                \item Verwende \hyperref[subsec:2.18]{Spurentechnick} um die womöglich vielen Bänder von M auf einem Band zu simulieren.
                \item Inteligentes Speichermanagement: Simulierte Köpfe bleiben in der Mitte simuliertes Band "bewegt" sich + geschicktes Lücken lassen im Speicher.
            \end{enumerate}
            Bild für den Fall, dass M eine Band hat
            \[ABBHIEREINFÜGEN\]
            Genauer verfügt U bei Eingabe \(code(M,w)\) wie folgt:
            \begin{itemize}
                \item Zu Beinn werden einige Initisierungsschritte ausgeführt, die es erlauben die k Bänder von M in jeweils einer Spur auf den ersten Band von U geeignet zu simulieren, wobei auf dem zweiten Band eine Repräsentition von M erstellt wird. Es wird aber sichergestellt, beispielsweise mittels mehrerer Spuren auf dem zweiten Band, dass das zweite Band noch verwendet werden kann um mit un der Abschnittslänge linearem Aufwand Abschnitte auf dem ersten NBand verschienen zu können.
                \item Während der Simulation von M wird jede Spur auf dem ersten Band von U in Abschnitte unterteilt 
                \begin{itemize}
                    \item Die Felder auf dem ersten Band von U auf dem der Kopf zu Beginn steht bildet für alle Spren \(i \in [k]\) den aus einem Feld bestehenden Abschnitt \(H^{(i)}\). Die Simulation von M wird so mittels angepasster Spurentechnik durchgeführt, dass die Simulierte Kopfposition auf jeder Spur \(z \in [k]\) das Feld im Abschnitt \(H^{(i)}\) ist und auf keiner Spur wird dieses Feld am Ende der Simulation eines Schrittes von M mit \# Beschrieben sein.
                    \item Auf jeder Spur \(i \in [k]\) ist unmittelbar links von \(H^{(i)}\) der Abschnitt \(L_1^{(i)}\) der Länge 2 und für jedes \(j \leq 1\) unmittelbar links von \(L_j^{(i)}\) der Abschnitt \(L_{j+1}^{(i)}\) der länge \(2^{j+1}\). 
                    \item Analog ist auf jeder Spur \(i \in [k]\) unmittenbar rechts von \(H^{(i)}\) der Abschnitt \(R_1^{(i)}\) der Abschnitt \(R_{j+1}^{(i)}\) der länge \(2^{j+1}\).
                    \item Jder Abschnitt auf einer Spur \(i \in [k]\), außer \(H^{(i)}\), wird beim ersten Verwenden in dem Sinne als halb leer initialisiert, dass er zur Hälfte mit den die Blank-Symbole von M repräsentierenden Symbolen beschrieben.
                    \item Zu jeden Zeitpunkt stehen die Lückensymbole \# in einem Abschnitt am weitesten links im Abschnitt.
                    \item Ein Abschnitt \(L_j^{(i)}\) oder \(R_j^{(i)}\) heißt \textbf{leer}, wenn alle Symbole in diesem Abschnitt \# sind, \textbf{halb leer} wenn die Hälfte der Symbole in diesem Abschnitt \# oder er nicht initial ist und \textbf{voll} wenn kein Symbol in diesem Abschnitt \# ist.
                    \item Die Darstellung der durch die Spuren repräsentierten bändervon M ist dabei so zu verstehen, dass die sich wie bei derSpurentechnick üblichen Darstellung der Bänder ergeben wenn die Felder mit den Lückensymbolen \# ignoriert werden.
                    \item Vor Simulationsbegin wird w als Eingabe an die simulierte Maschine M übergeben ohne dabei die Lückensymbole \# zu überschreiben. 
                \end{itemize}
                \item Werden nun die Bandbewegungen geeignet durch die Spurenmanipulationen realisiert, so kann U die Turingmaschine M simulieren, da den Abschnitt \(H^{(i)}\) die zur Simulation nötigen Informationen über die gegenwärtige Konfiguration von M entnommen werden können.
                \item Bandbewegungen nach links (die Kopfbewegung nach rechts sinnvoll) werden durch manipulationen der entsprechenden Spur wie folgt simuliert:
                \begin{itemize}
                    \item Für das minimale \(j_0 \leq 1\), so dass \(R_{j_0}^{(i)}\) nicht leer ist, ersetzt das erste Symbol der rechten Hälfte, falls \(R_{j_0}^{(i)}\) halb voll ist, und der linken Hälfte, falls \(R_{j_0}^{(i)}\) voll ist, das Symbol im Abschnitt \(H^{(i)}\) und die \(2^{j_0 - 1}-1\) weitere Symbole Symbole der Hälfte von \(R_{j_0}^{(i)}\) werden im unveränderter Reihenfolge auf die rechten Hälften von \(R_1^{(i)}, \cdots, R_{j_0-1}^{(i)}\) geschrieben.
                    \item Die von \# verschiedenen Symbole der Abschnitte \(L_1^{(i)}, \cdots, L_{j_0^{(i)}}\) werden im unveränderter Reihenfolge auf den Abschnitt \(L_{j_0}^{(i)}\) und die rechten Hälften der Abschnitte \(L_1^{(i)}, \cdots, L_{j_0 -1}^{(i)}\) geschrieben und die Symbole in den linken Hälften der Abschnitte \(L_1^{(i)}, \cdots, L_{j_0 -1}^{(i)}\) werden durch \# überschrieben.
                    \item Schließlich wird das nun ersetzte Symbol, das zuvor im Abschnitt \(H^{(i)}\) stand auf die rechte Hälfte des Abschnitts \(L_1^{(i)}\) geschrieben.
                \end{itemize}
                \item Bandbewegungen nach rehcts werden analog simuliert.
                \item Die Bandbewegungen lassen sich eindeutig in dieser Weise simulieren da für alle \( j \leq 1\) zu jede, Zeitpunkt für die Abschnitte \(L_j^{(i)}\) und \(R_j^{(i)}\) folgendes gilt: Entweder sind beide Abschnitte halb voll oder einer ist leer und der andere voll.
                \item Beisliel hier!
                \item Die Laufzeit von U ergibt sich im Wesentlchen aus der Simulation der Bandbewegungen. Bei einer Bandbewegung ist die Anzahl der bewtrachteten Felder \(O(\sum\limits_{j = 1}^{j_0} 2^j) = O(2^{j_0})\) für ein \(j_0 \leq 1\). Dank des zweiten Bandes lassen solche Bandbewegungen im Zeit \(O(2^{j_0})\) durchführen. Ist t die Laufzeit von M zur Eingabe w, so ist eine solche Bandbewegung höchstens \(\frac{t}{2^{j_0}}\) und durchzuführen. Als Laufzeit von U ergibt sich so bis auf konstante Faktoren \(\sum\limits_{j_0 = 1}^{\log_2 t} \frac{t}{2^{j_0}} = t\log_2 t\).
                \begin{itemize}
                    \item \(\frac{t}{2^{j_0}}\) \(\to\) häufigkeit
                    \item \(2^{j_0}\) \(\to \) Kosten
                \end{itemize}
            \end{itemize}
    \end{proof}
\newpage
\mysubsection{Satz (Universelle TM-Simulation mit Zeitbeschränkung)*}
    Ist für alle TM \(M = (\{0, \cdots, n\}, \{0, 1\}, \{\Box, 0, 1\}, \Delta, 0, \{0\})\) und alle Wörter \(w \in \{0, 1\}^*\) das wort \(code(M, w)\) ein geeigneter Code für (M, w), so gibt es eine 2-TM U, so dass folgendes gilt.
    \begin{itemize}
        \item [(i)] \(\forall\) TM M wie oben und \(\forall w \in \{0, 1\}^*\) akzeptiert U den Code \(code(M, w)\) genau dann, wenn M das Wort w akzeptiert.
        \item [(ii)] Für alle Zeitschranken t, \(\forall\) t - zeitbeschränkten TM M wie oben gibt es ein \(u_0\) und c, so dass \(\forall w \in \{0, 1\}^{\geq n_0}\) jede Rechnung von U zur Eingabe \(code(M, w)\) Länge höchstens \(ct(|w|)\) hat, falls U eine Rechnung von M zur Eingabe w simuliert, die von M akzeptiert wird.
    \end{itemize}
    \begin{proof}
        Die Idee ist die Bänder ist die Bänder einer TM zu Simulieren. Genauer verfährt U bei Eingabe \(code (M, w)\) wie folgt:
        \begin{itemize}
            \item Zu Beginn werden einige Initialierungsschritte ausgeführt, die es erlauben die k Bänder von M nacheinander auf dem ersten Band von U zu simulieren, wobei auf einer Spur auf dem ersten Band eine repräsentation von M erzeugt wird, die während der Simulation so verschoben wird, das sie vor jedem zu simulierenden Schritt unmittelbbar rechts von Kopf auf dem ersten Band steht. 
            \item Die k-TM M wird mit w als Eingabe simuliert, indem nacheinander für jedes Band \(i \in [k]\) alle Kopfbewegungen auf Band i bis zum Rechnungsende simuliert werden. Es werden also zunächst alle Kopfbewegungen auf dem ersten, dann auf dem zweiten Band usw. simuliert.
            \begin{itemize}
                \item Dies erfordert im Sinne der Arbeitsweise von M Kenntnis der in jedem Schritt gelesenen Bandsymbol auf allen anderen Bändern und der gewählten Instruktion, weshalb diese zu Beginn der Simulation des ersten Bandes für die nicht simulierten Bänder nicht deterministisch geraten werden und auf dem zweiten Band gespeichert werden.
                \item Bei der Simulation der weiteren Bänder wird dann geprüft, ob die tatsächlich dort auftretenden Bandsymbole zu den geratenen Bandsymbole und Instruktion passen. Ist dies nicht der Fall \\ \(\rightarrow\) Terminierung mit Nichtakzeptanz.\\ Andernfalls war die Simulation erfolgreich und die Eingabe wird genau dann akzeptiert, wenn die Simulation die Simulierte Eingabe akzeptiert hat.
            \end{itemize}
        \end{itemize} 
        Nach Konstruktion sieht man leicht, dass es ein \(c, n_0 \in \mathbb{N}\) gibt, so dass U höchstens \(ct(|w|)\) Schritte benötigt falls \(|w| \geq n_0\).
    \end{proof}
\mysubsection{Definition (Zeitbeschränkte Funktionen und Sprachen)*}
    Für eine Menge T von Funktionen von \(\mathbb{N}_0 \to \mathbb{R}_{\geq 0}\) definieren wir
    \begin{itemize}
        \item \textbf{DTIME}(T) := \(\{L(M) \subseteq \{0, 1\}^* : \exists t \in T\) ist M eine t-zeitbeschränkte DTM\}
        \item \textbf{FTIME}(T) := \(\{\varphi(M) : \mathbb{N}_0 \to \mathbb{N}_0 : \exists t \in T\) und M ist eine t-zeitbeschränkte DTM\}
        \item \textbf{NTIME}(T) := \(\{L(M) \subseteq \{0, 1\}^* : \exists t \in T\) ist M eine t-zeitbeschränkte TM\}
    \end{itemize}

\mysubsection{Definition (Zeitkonsturierbar)}
    Eine Zeitschranke t ist genau dann \textbf{Zeitkonsturierbar}, wenn es eine DTM M mit Eingabealphabet \{1\} gibt, so dass \(time_M(1^n) = t(n)\) gibt.

\mysubsection{Bemerkung (Polynomielle Zeitkonstruierbarkeit)*}
    \begin{itemize}
        \item [(i)] Ist p ein Polynom in einer variable über \(\mathbb{Z}\) mit \(p(n) \geq n+1 \quad \forall n \in \mathbb{N}_0\), so ist \(t: \mathbb{N}_0 \to \mathbb{N}_0\), \(n \mapsto p(n)\) Zeitkonstruierbar.
        \item [(ii)] Ist t Zeitkonsturierbar, so ist auch \(T: \mathbb{N}_0 \to \mathbb{N}_0\), \(n \mapsto 2^{t(n)}\) Zeitkonsturierbar.
    \end{itemize}

\mysubsection{Satz (Zeithirarchiesatz dür deterministische TM)}
    Sei t eine Zeitschranke und T eine Zeitkonsturierbare Zeitschranke mit \(t(n) \log t(n) = o(T(n))\). Dann gilt \textbf{DTIME}(t(n))\(\not \subseteq\) \textbf{DTIME}(T(n)).
    \begin{proof}
        Sei \(M_0, M_1, \cdots\) eine geeignete effektive Aufzählung deterministische TM der Form \((\{0, \cdots, n\}, \{0,1\}, \{\Box, 0, 1\}, \Delta, 0, \{0\})\), so dass \(\forall\) DTM M dieser Form es unendlich viele \(e \in \mathbb{N}_0\) mit \(M = M_e\) gibt.\\Wir betrachen eine DTM U, die bei Eingabe \(1^e\) die DTM \(M_e\) bei eingabe \(1^e\) simuliert und genau dann akzeptiert, wenn die Simulation terminiert und nicht akzeptiert. \\ Nach \hyperref[subsec:7.11]{Satz 7.11} können wir U so wählen, dass es \(\forall c \geq 1\) und alle \(ct(n) \log t(n)\) - zeitbeschränkter DTM M der obigen Form ein \(e_{c_1 M}\) mit \(M = M_{e_{c_1}M}\), so dass die Rechnung von U zur Eingabe \(1^e_{c_1 M}\) höchstens Lämge \(T(e_M, c)\) hat.\\Da T zeitbeschränktist, ist es möglich U so zu modifizieren, dass U die Simulation immer nach \(T(e)\) Schritten abbricht, wobei wir bei Abbruch nie akzeptieren.\\Dann ist U offenbar \(T(n)\) - zeitbeschränkt, es gilt also \(L(U) \in \textbf{DTIME}(T(n))\). \\Wir zeigen nun, dass keine \(t(n)\) - zeitbeschränkte DTM M existiert mit \(L(M) = L(U)\)\\Ist M eine \(t(n)\) - zeitbeschränkte DTM, so existiert nach \hyperref[subsec:7.4]{Satz 7.4} ein \(c \geq 1\) und eine \(ct(n)\) - zeitbeschränkte DTM M' der obigen Form mit \(L(M) = L(M')\).\\Die TM U bricht dann die Simulation der Eingabe \(1^{e_{c_1 M'}}\) nicht ab, es gilt also \(L(M_{e_{c_1 M'}}) \not = L(U)\) und damit wegen \(L(M_{e_{c_1 M'}}) = L(M') = L(M)\) also \(L(U) \not = L(M)\).
    \end{proof}

\mysubsection{Proposition(Zeitbeschränkte TM-Äquivalenz)*}
    Für alle \(t:\mathbb{N}_0 \to \mathbb{R}_{\geq 0}\) und t - zeitbeschränkte TM M ibt es eine DTM M' mit \(L(M) = L(M')\), sodass M' \(2^{O(t)}\) - zeitbeschränkt ist.
    \begin{proof}
        \(\forall\) M wie oben gibt es \(c \in \mathbb{N}_0\) sodass es bei Eingabe w höchstens \(c^{t(|w|)}\) viele rechnungen zur Eingabe w gibt. \\
        \(\leadsto\) Brute-Force
    \end{proof}

\mysubsection{Satz (Zeithirarchiesatz für nichtdeterministische TM)}
    Sei t eine Zeitschranke und T eine zeitkonsturierbare Zeitschranke mit \(t(n+1) = o(T(n))\). Dann gilt \(\textbf{NTIME}(t(n)) \not \subseteq \textbf{NTIME}(T(n))\). 
    \begin{proof}
        Sei \(M_0, M_1, \cdots\) eine geeignete Aufzählung von TM der Form \(M = (\{0, \cdots, m\}, \{0, 1\}, \{\Box, 0, 1\}, \Delta, 0, \{1\})\), so dass \(\forall\) TM M dieser Form unendlich viele \(e \in \mathbb{N}\) mit \(M = M_e\) gibt.\\
        Sei \(l_0 := 0\). Sei \(l_e\) für \(e \geq 1\) deterministisch durch \(l_e := 2^{T(l_{e-1})}\).\\
        Für \(e \in \mathbb{N}_0\), sei \(I_e := (l_e, l_{e+1}) \cap \mathbb{N}_0\).\\
        Wir betrachten eine TM U, die bei Eingabe \(1^x\) wie folgt verfährt.
        \begin{itemize}
            \item Es wird \(e\in\mathbb{N}_0\) mit \(x \in I_e\) bestimmt.
            \item Gilt \(x < l_{e+1}\), wird \(M_e\) bei Eingabe \(1^{x+1}\) simuliert und dann genau akzeptiert, wenn die Simulation terminiert und akzeptiert.
            \item Gilt \(x = l_{e+1}\), so werden alle Rechnungen von \(M_e\) bei Eingabe \(1^{l_e +1}\) simuliert und es wird genau dann akzeptiert, wenn \(M_e\) nicht akzeptiert. 
        \end{itemize}
        Da T Zeitkonsturierbar ist, lässt sich gegeben x, \(I_e\) (bzw. e) bestimmen. Nach \hyperref[subsec:7.12]{Satz 7.12} können wir daher U so wählen, dass es \(\forall x \leq 1\) und \(\forall\) \(ct(n)\) - zeitbeschränkter TM M der obigen Form ein \(e_{c,M}\) mit \(M_{e_{c,M}} = M\) gibt, so dass alle rechnungen von U zur Eingabe \(l^x\) mit \(x \in I_{e_{c, M}} / \{l_{e_{c,M + 1}}\}\) höchstens Länge T(x) haben (falls sie eine akzeptierte Rechnung simmulieren) und so dass die Rechnung von U zur Eingabe \(1^{l_{e_{c,M + 1}}}\) hächstens \(2^{T(l_{e_{c,M}})} = l_{e_{c,M + 1}} \leq T(l_{e_{c, M +1}})\) hat.\\
        Da T Zeitkonsturierbar ist, ist es möglich U so zu modulieren, dass U die Simulation immer nach T(x) Schritten abbricht, wobei U dann bei abgebrochener Simulation wie akzeptiert.\\
        Dann ist U offenbar T(n) - Zeitbeschränkt, es gilt also \(L(U) \in \textbf{NTIME}(T)\).\\
        Wir zeigen nun, dass keine t(n) - zeitbeschränkte TM M mit L(U) = L(M) existiert.\\
        Ist M eine t(n) - zeitbeschränkte TM mit L(M) = L(U), so existiert nach \hyperref[subsec:7.4]{Satz 7.4} ein \(x \leq 1\) und eine ct(n) - zeitbeschränkte TM M' der obigen Form mit 
        \[
            L(M') = L(M) = L(U)
        \]
        Die TM U bricht dann die simulation bei allen Eingaben \(1^x\) mit \(x \in I_{e_{c, M'}}\) nicht ab, \(\forall x \in I_{e_{c, M'}} / \{l_{e_{c, M' +1}}\}\) gilt also 
        \[
            1^x \in L(M') \Leftrightarrow 1^x /in L(U) \Leftrightarrow 1^{x+1} \in L(M')
        \]
        Folgendes gilt 
        \[
            1^{l_{e_{c,M' + 1}}} \in L_{M'} \Leftrightarrow 1^{l_{e_{x, M' + 1}}} \in L(M') \Leftrightarrow ...
        \]
        Dies ist im Wiederspruch zu L(M') = L(M).
    \end{proof}
    Im Folgenden fixieren wir eine Aufzählung \(M_0, M_1, \cdots\) aller normierten DTM.

\mysubsection{Definition(abstraktes Komplexitätsmaß)}
    Eine partielle Funktion \(\phi : \mathbb{N}_0 \times \{0, 1\}^* \leadsto \mathbb{N}_0\) heißt \textbf{abstraktes Komplexitätsmaß} falls folgendes gilt:
    \begin{itemize}
        \item [(i)] \(\phi(e,x) \downarrow \Leftrightarrow M_e(x) \downarrow \quad \forall e \in \mathbb{N}_0, x \in \{0, 1\}^*\)
        \item [(ii)] Die Relation \(\{(e, x, t) \in \mathbb{N}_0 \times \{0, 1\}^* \times \mathbb{N}_0 : \phi(e,x) = t\}\) ist im Sinne entscheidbar, dass die Funktion \(\psi : \mathbb{N}_0 \times \{0, 1\}^* \times \mathbb{N}_0 \to \{0, 1\}\) mit \(\psi(e, x, t) = 1 \Leftrightarrow \phi (e,x) = t \quad \forall e, x, t\) berechenbar ist.
    \end{itemize}

\mysubsection{Definition(T-Beschränkte TM und DCOM)*}
    Sei \(t: \mathbb{N}_0 \to \mathbb{N}_0\) eine Funktion für \(e \in \mathbb{N}_0\) wird die TM \(M_e\) als \textbf{t - \(\phi\) - beschränkt} bezeichnet, wenn \(M_e\) total ist und es ein \(n_0 \in \mathbb{N}_0\) gibt, sodass \(\phi(e, x) \leq t(|x|) \quad \forall x \in \{0, 1\}^{\geq n_0}\) gilt. Wir setzen:
    \[
        \textbf{DCOM}^{\phi}(t) := \{L(M) \subseteq \{0, 1\}^* : M ist t - \phi - beschränkt\}  
    \] 
    Beispielsweise ist Zuordnung \(e, x \mapsto time_{M_e} (x)\) ein abstraktes Komplexitätsmaß.

\mysubsection{Satz (Lückensatz)}
    Sei \(\phi\) ein abstracktes Komplexitätsmaß und sei \(f, g : \mathbb{N}_0 \to \mathbb{N}_0\) streng monoton wachende Funktion. Dann gibt es eine streng monoton wachende berechenbare Funktion \(t : \mathbb{N}_0 \to \mathbb{N}_0\) mit \(f(n) \leq t(n) \quad \forall n \in \mathbb{N}_0\) und \(\textbf{DCOMP}^{\phi}(t) = \textbf{DCOMP}(g \circ t)\).
    \begin{proof}
        Wir konstruieren t so, dass es \(\forall e \in \mathbb{N}_0\) ein \(n_0 \in \mathbb{N}_0\) gibt, so dass \(\phi(e, x) \not \in [t(|x|) + 1, g(t(|x|))] \quad \forall x \in \{0, 1\}^{\geq n_0}\) gilt.
        \[SKIZZE HIER \]
        Dafür genügt es t(n) für \(n \in \mathbb{N}_0\) so zu wählen, dass \(\phi (e,x) \not \in [t(n) + 1, g(t(n))] \quad \forall e < n\) und \(x \in \{0, 1\}^{= n}\) gilt.\\
        \(\forall n \in \mathbb{N}_0\) ist \(\{\phi(e,x) : e < n, x \in \{0, 1\}^{=n}\}\) eine endliche Menge.\\
        \(\Rightarrow\)(hier nocht limits eintragen!!!) effektive Bestimmung einer abstrakten Schranke möglich.\\
        \(\Rightarrow\) t(n) groß genug wählen genügt
        \paragraph{Genauer:}
            \(t(n) := min \{m \in \mathbb{N}_0 : t(n-1)<m \wedge f(n) < m \wedge \forall e < n : \forall x \in \{0, 1\}^{=n} : \phi(e,x) \in [m+1, g(m)]\}\).
            \(\Rightarrow\) t ist berechenbar, streng monoton wachend und \(f(n) \subseteq t(n)\)\\
            Sei \(L \in \textbf{DCOM}^{\phi}(g /circ t)\). Sei \(e \in \mathbb{N}_0\) so dass \(M_e\) eine \((g \circ t)\) - \(\phi\) - zeitbeschränkte TM ist mit \(L(M_e) = L\). Wir zeigen nun, dass \(M_e\) t - \(\phi\) - beschränkt ist. Sei \(x \in \{0, 1\}^{\geq e + 1}\). Es genügt zu zeigen , dass \(\phi (e, x) \leq t (|x|)\). Nach Konstruktion von t gilt \(\phi (e, x) \not \in [t(|x|) + 1, g(t(|x|))]\) und nach Wahl von e gilt \(phi (e, x) \leq g ((t(|x|)))\). Es folgt also \(\phi(e,x) \leq t(|x|)\).
    \end{proof}